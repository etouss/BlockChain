\documentclass{article}

\newcommand{\bchain}{\text{bchain}}

\newcommand{\body}{K}

% \let\bfseriesbis=\bfseries \def\bfseries{\sffamily\bfseriesbis}
%
%
% \newenvironment{point}[1]%
% {\subsection*{#1}}%
% {}
%
% \setlength{\parskip}{0.3\baselineskip}


%% USEFUL packages
\usepackage{mypackages}

%% USEFUL macros
%% Macros Juan

\newcommand{\paths}{\text{PATHS}}

%% Macros comments
\newcommand{\tover}[1]{\textcolor{red}{#1}}
\newcommand{\td}[1]{\textcolor{blue}{[TODO: #1]}}

%% Macros logics
\newcommand{\NN}{\mathbb{N}}
\newcommand{\ZZ}{\mathbb{Z}}
\newcommand{\MM}{\mathbb{M}}
\newcommand{\SE}{\mathbb{S}}
\newcommand{\BB}{\mathbb{B}}
\newcommand{\RR}{\mathbb{R}}
\newcommand{\cF}{\mathcal{F}}
\newcommand{\cI}{\mathcal{I}}
\newcommand{\QFBILIA}{\textsf{QFBILIA}}
\newcommand{\nnf}{\textsf{f}}

\newcommand{\ite}{\textsf{ite}}
\newcommand{\limp}{\Rightarrow}
\newcommand{\flc}{\rightarrow}

\newcommand{\bagone}[1]{\llbracket #1 \rrbracket}
\newcommand{\bsingle}{\textsf{bag}}
\newcommand{\bplus}{\oplus}
\newcommand{\bminus}{\ominus}

%% Macros tools
\newcommand{\spen}{\textsc{spen}}
\newcommand{\zzz}{\textsc{Z3}}

%% Environments
\newtheorem*{myrem}{Remark} %% based on amsthm
\newtheorem{mydef}{Definition}
\newtheorem{myex}{Example}
\newtheorem*{mynota}{Notation}
\newtheorem{myprop}{Proposition}
\newtheorem{mylem}{Lemma}
\newtheorem*{mylem*}{Lemma}
\newtheorem*{myprop*}{Proposition}
\newtheorem*{comp}{Efficiency Study}

\newenvironment{point}[1]
{\subsection*{#1}}%
{}

\newcommand{\flist}{\text{\sc FList}}
\newcommand{\set}{\text{\sc Set}}
\newcommand{\fset}{\text{\sc FSet}}
\newcommand{\B}{\text{\bf B}}
\newcommand{\G}{\mathcal{G}}
\newcommand{\K}{\mathcal{K}}
\newcommand{\LOG}{\text{\sc Log}}
\newcommand{\length}{\text{\rm length}}
\newcommand{\BK}{\text{\sc BK}}
%\newcommand{\cP}{\mathcal{P}}
%\newcommand{\cV}{\mathcal{V}}
%\newcommand{\cC}{\mathcal{C}}
%\newcommand{\cS}{\mathcal{S}}
%\newcommand{\cA}{\mathcal{A}}
%\newcommand{\cR}{\mathcal{R}}
%\newcommand{\cQ}{\mathcal{Q}}
\newcommand{\cG}{\mathcal{G}}
\newcommand{\cT}{\mathcal{T}}
\newcommand{\pr}{\mathbf{Pr}}
\newcommand{\Dyck}{\mathcal{D}}
\newcommand{\expected}{\mathbf{E}}
\newcommand{\bv}{\mathbf{v}}
\newcommand{\bV}{\mathbf{V}}
\newcommand{\bs}{\mathbf{s}}
\newcommand{\bw}{\mathbf{w}}
\newcommand{\ba}{\mathbf{a}}
\newcommand{\bq}{\mathbf{q}}
\newcommand{\bx}{\mathbf{x}}
\newcommand{\by}{\mathbf{y}}


\newcommand{\marcelo}[1]{{\color{red} {\bf Marcelo: #1}}}
\newcommand{\etienne}[1]{{\color{blue} {\bf Etienne: #1}}}
\newcommand{\juan}[1]{{\color{brown} {\bf Juan: #1}}}
\newcommand{\domagoj}[1]{{\color{green} {\bf Domagoj: #1}}}
\newcommand{\francisco}[1]{{\color{magenta} {\bf Francisco: #1}}}
\newcommand{\martin}[1]{{\color{orange} {\bf Martin: #1}}}

\newcommand{\quot}[1]{#1/\!\equiv}

\newcommand{\body}{q}
\newcommand{\bchain}{\text{bc}}

\newcommand{\owner}{\text{\rm owner}}
\newcommand{\pred}{\text{\rm pred}}
\newcommand{\mine}{\text{\rm mine}}
\newcommand{\suc}{\text{\rm succ}}


\newcommand{\bP}{\mathbf{P}}
\newcommand{\bB}{\mathbf{B}}
\newcommand{\bA}{\mathbf{A}}
\newcommand{\bR}{\mathbf{R}}
\newcommand{\bS}{\mathbf{S}}
\newcommand{\bH}{\mathbf{H}}
\newcommand{\bQ}{\mathbf{Q}}

\newcommand{\df}{\text{\rm default}}
\newcommand{\gf}{\text{\rm gen\_fork}}
\newcommand{\mfork}{\text{\rm fork($m$)}}
\newcommand{\last}{\text{\rm last}}
\newcommand{\best}{\text{\rm best}}
\newcommand{\cho}{\text{\rm choose}}

\newcommand{\ie}{i.e.$\!$ }

\newcommand{\longest}{{\text{\rm longest}}}

\newcommand{\subbody}{{\text{\rm sub-body}}}








%% Title
% \title{}
% \author{}
% \date{}

\begin{document}

% \sloppy
% \maketitle

\title{BlockChain Framework}

%\author{Etienne Toussaint, Leonid Libkin \\
%	Laboratory for Foundations of Computer Science, The University of Edinburgh }

%\date{21st August 2016}

%\maketitle

\pagestyle{empty} %
\thispagestyle{empty}

\section{Blockchain}

\medskip
\noindent
\textbf{Lists and blocks}

Given a set $S$, let $\set(S)$ be the set of sets of elements of $S$ and $\flist(S)$ be the set of all finite lists of elements of $S$. Given $L \in \flist(S)$, we use notation $\length(L)$ to refer to the number of elements in $L$, notation $L[i]$ to refer to the $i$-th element in $L$, where $i \in \{1, \ldots, \length(L)\}$, and notation $L[i,j]$ to refer to the sublist $[L[i], \ldots, L[j]]$ of $L$, where $i,j \in \{1, \ldots, \length(L)\}$ and $i \leq j$. Notice that $\length(L) = 0$ if and only if $L$ is the empty list $[\ ]$. Finally, we say that a list $L_1$ is a prefix of a list $L_2$ if $L_1$ is the empty list, or $1 \leq \length(L_1) \leq \length(L_2)$ and $L_1 = L_2[1, \length(L_1)]$. 

We assume a fixed alphabet $\Sigma$, and that blocks contain the following information: 
\begin{itemize}
\item A word in $\Sigma^*$ that serves as the id of the block. 
\item The id of the previous block. 
\item Another word serving as an identifier of the owner of the block. 
\end{itemize}


\medskip
\noindent
\textbf{Body of Knowledge and Blockchain}

+++ explain how blockchain works, nodes have ids and previous block hash so one mines blocks after other blocks +++

Let us fix a \emph{genesis block}, identified as $\epsilon$ and that does not contain neither a previous block id or an owner id. 

We represent all knowledge in a blockchain system as a tree $\body$ rooted in the genesis block: the set of nodes of of this tree are all blocks that have been mined and there is an edge from block $b_1$ to block $b_2$ if $b_2$ was mined from $b_1$. We denote these trees as \emph{body of knowledges}. 

+++ explain the notion of the blockchain ++++  


Given a body of knowledge $\body$, we say that the blockchain of $\body$ is the list formed from the longest path from the root to the tree, 
if such a path is unique. If two or more different paths are tied for the longest, then we say that the blockchain in $\body$ does not exists. 
We use the notation $\bchain(\body)$ as a function that returns the blockchain of $\body$, if it exists, or the empty list otherwise. 


\section{Mining game}

+++ explain a bit on how blockchain uses miners to validate, and that whomever wins a block receives a mining reward. 
Include the notion that our blocks  +++

The mining game is played by a set $\cP = \{1, \ldots, m\}$ of players, and the set of states of our game consists of all possible body of knowledges in which all blocks except for the genesis block are owned by one of the players in $\cP$. 

On each step, miners looking to maximise their rewards choose a block in the current body of knowledge, and attempt to mine from this block. Thus, in each turn, each of the players race to put the next block in the body of knowledge, and only one of them succeeds. The probability of succeeding is directly related to the comparative amount of hash power available to this player, the more hash power the likely it is that she will mine the next block before the rest of the players. Once a player puts a block, this block is added to the current state, obtaining a different body of knowedge, and the game continues from this new state. 

In order to formally define our game, let us denote by $\BLOCKS(\cP)$ the set of all possible blocks owned by any player in $\cP$, and 
$\BK(\cP)$ the set of all body of knowledges constructed from blocks in $\BLOCKS(\cP)$, so that the states of our game are precisely 
$\BK(\cP)$. 

A strategy for a player $p$ is a function $\BK(\cP) \rightarrow \BLOCKS(\cP)$ that assigns to each body of knowledge a block of it 
where player $p$ wishes to mine next. 


%\input fichesynth.tex


%\input abstract.tex


%\input semantics.tex
%\input preliminaries.tex
%\input fragmentsimple.tex
%\input fragmentorder.tex

%\input introduction.tex
%\input block.tex
%
An attempt to give the most abstract possible definition of a blockchain.

\section{BlockChain}

\begin{mydef}
	We call Validated Rules a function noted $V$:
	$$V : (\Sigma^*)^\times \rightarrow \mathscr{P}(\Sigma^*)$$
\end{mydef}

\begin{myrem}
	Intuitively $V$ is a function taking a list of block in input and returning the set of block which are valid.
\end{myrem}

\begin{mydef}
	We call $(G,V)$ validated chain,where $G \in (\Sigma^*)^\times$, noted $log_{G,V}$ a function 
	$$log_{G,V}: \mathbb{N} \rightarrow \Sigma^*$$
	such that:
	\begin{align*}
	&G(0) \in V(\emptyset) \\
	&\forall i \in \llbracket 0;|G|\rrbracket, log(i) = G(i) \\
	&\forall i \in \mathbb{N}^+ , log(i) \in V(log_{G,V}(0),,,log_{G,V}(i-1)) 
	\end{align*}
\end{mydef}

\begin{myrem}
	$G$ is the list of genesis block to startup the system. $log_{G,V}$ would be an infinite chain that is valid regarding $V$.
\end{myrem}

\begin{mynota}
	$$\forall i , log_{G,V}(i)^- = (log_{G,V}(0),,,log_{G,V}(i))$$
	$$LOG_{G,V} = \{ f | \textit{ f is a (G,V) validated chain}\}$$
\end{mynota}

\begin{myrem}
	We introduce $LOG_{G,V}$ which really complicated but is actually necessary to deal with fork and consensus later.
\end{myrem}

\begin{mydef}
	A set of validated chain $LOG_{G,V}$ is said to be infinite if:
	$$\forall log_{G,V} \in LOG_{G,V}, \forall i \in \mathbb{N} , \forall b  \in V(log_{G,V}(i)-), V(log_{G,V}(i)-,b)\neq \emptyset $$
\end{mydef}

\begin{myrem}
	Infinite here is used in a sense that whatever instance of a $G,V$ validated-chain we are dealing with we will always be able to complete it.
\end{myrem}

\begin{mydef}
	A set of validated chain $LOG_{G,V}$ is said secured if :
	$$\forall log_{G,V} \in LOG_{G,V}, \forall i \in \mathbb{N}^+ , \forall b,b' \in \Sigma, b\neq b' \implies V(log_{G,V}(i)^-,b)\cap V(log_{G,V}(i)^-,b') = \emptyset$$
\end{mydef}

\begin{myrem}
	Intuitively in order to be secured $V(,,b)$ should depend on $b$ as bitcoin include previous hash block. 
\end{myrem}

\begin{mydef}
	We call  alive Set of player a tuple $(P,K_P)$ where $P$ is a set of player and $K_P$ a function:
	$$K_P : P\times T \rightarrow \mathscr{P}(\Sigma^*)$$
	such that: 
	\begin{align*}
	&\forall p \in P, \forall t,t' \in T; t\leq t' \implies K_P(p,t) \subseteq K_P(p,t') \\
	&\exists p \in P, \forall t \in T, \exists t'>t, K_P(p,t) \subset K_P(p,t') \\
	\end{align*}
\end{mydef}

\begin{myrem}
	$K_p$ represent the knowledge of $p$ this knowledge should not decrease. And to be considered alive at least one player should be trying to increase is knowledge (mining).
\end{myrem}

\begin{mydef}
	We call an alive set of validated chain a tuple $(LOG_{G,V},P,K_P)$ where $LOG_{G,V}$ is an set of infinite validated chain and $P,K_P$ an alive set of player.
\end{mydef}

\begin{myprop*}
	Let $(LOG_{G,V},P,K_P)$ an alive set of validated chain then:
	$$\forall log_{G,V} \in LOG_{G,V}, \forall i \in \mathbb{N}, \exists p \in P, \exists t \in T , V(log_{G,V}(i)^-) \cap K_P(p,t) \neq \emptyset $$
\end{myprop*}
\begin{myrem}
	To be honest i am not sure as we are dealing with infinite number. I may have to trick things here. I want to ensure the fact that the chain will eventually move forward.
\end{myrem}

\begin{mydef}
	A block chain protocol is a function noted $P_{G,V}$:
	$$P_{G,V} : (LOG_{G,V}\times \mathbb{N})  \times (LOG_{G,V}\times \mathbb{N}) \times T \rightarrow (LOG_{G,V}\times \mathbb{N}) $$
	such that :
		\begin{align*}
		&\forall log_{G,V},log_{G,V}' \in LOG_{G,V}, \forall n,n' \in \mathbb{N} , \forall t \in T; P_{G,V}(log_{G,V},n,log_{G,V}',n',t) = (log_{G,V},n) \lor (log_{G,V},n')		\end{align*}
\end{mydef}
\begin{myrem}
	$P_{G,V}$ can be seen as the rule in case of fork and new block. 
\end{myrem}

\begin{mydef}
	Considering an alive set of validated chain $(LOG_{G,V},P,K_P)$ and a block chain protocol $P_{G,V}$. We denote $S_{t,p}$ where $t\in T$ and $p\in P$ the set of tuple :
	$$ S_{t,p} = \{log_{G,V}(N)^- | log_{G,V} \in LOG_{G,V} \land log_{G,V}(N)^- \subseteq K_P(p,t)\} $$
	
	We call BlockChain at time $t\in T$ for user $p \in P$ noted $BC_{t,p}$ the tuple:
	\begin{align*}
	&BC_{t,p} \in S_{t,p} \\
	& \forall log \in S_{t,p}, P_{G,V}((log,|log|),(BC_{t,p},|BC_{t,p}|),t) = (BC_{t,p},|BC_{t,p}|) \\
	\end{align*}
	
\end{mydef}
\begin{myrem}
	Intuitively the blockchain for a user $p$ at a time $t$ is the best chain he fully knows regarding the protocol function and the validity at time $t$ (time-stamping).
\end{myrem}

At the end a specific block chain seems to be definable through her validation rules, a consensus protocol and each player knowledge function. Which seems to fit puzzle and rules.

%\input old.tex
%%!TEX root = main.tex
\section{Etienne 's modification space}


%\input debt.tex
%\input blockchain.tex
%\input openproblem.tex

%\input fragmentimpl.tex

%\input conclusion.tex

%\input acknow.tex


%% USEFUL Bibliography
%\clearpage
%\bibliographystyle{plain}
%\bibliography{dp-mset}

%\input appendix.tex

\end{document}
