%!TEX root = main.tex

\section{A Game-theoretic Formalization of Cryptocurrency Mining}
\label{sec-formalization}

The mining game is played by a set $\bP = \{0, 1, , \ldots, m-1\}$ of players, with $m \geq 2$.
In this game, each player has some reward depending on the number of blocks she owns. Every block must point to a previous block, except for the first block which is called the {\em genesis block}. Thus, the game defines a tree of blocks. Each block is put by one player, called the {\em owner} of this block. Each such tree is called a {\em state of the game}, or just {\em state}, and it represents the knowledge that each player has about the blocks that have been mined thus far.

The key question for each player is, then, where do I put my next block? In bitcoin, miners are only allowed to spend their reward if their blocks belongs to the \emph{blockchain}, which is simply the longest chain of blocks in the current state. Thus, players face essentially two possibilities: to put their blocks right after the end of the blockchain, or try to \emph{fork}, betting that a smaller chain will eventually become the blockchain. As the likelihood of mining the next block is directly related to the comparative hash power of a player, it makes sense to model mining as an infinite stochastic game, in which the probability of executing the action of a player $p$ is given by her comparative hash power.

In what follows, we first define the components of the infinite stochastic games considered in this paper, and the notions of strategy, utility and equilibrium for such games. Our formalization is similar to others presented in the literature \cite{mininggames:2016}, except for the way in which miners are rewarded and the way in which these rewards are accumulated in the utility function. Hence, we analyse them in more detail in Section \ref{sec-pay-uti}, emphasizing the properties of these two elements that are fundamental for our formalization.

% Let us now turn to the formal definition of the game.

\subsection{The definition of the game}
\subsubsection{Blocks, states and the notion of blockchain}\label{sub:states}

In a game played by $m$ players, a block is defined as a string $b$ over the alphabet $\{0,1,\ldots$, $m-1\}$. We denote by $\bB$ the set of all blocks, that is, $\bB = \{0,1,\ldots , m-1\}^*$. Each block apart from $\varepsilon$ has a unique owner, defined by the function $\owner: (\bB \smallsetminus \{\varepsilon\}) \rightarrow \{0,1, \ldots ,m-1\}$ such that $\owner(b)$ is equal to the last symbol of $b$. As in \cite{mininggames:2016}, a state of the game is defined as a tree of blocks. More precisely, a state of the game, or just state, is a finite and nonempty set of blocks $q \subseteq \bB$ that is prefix closed. That is, $q$ is a set of strings over the alphabet $\{0,1,\ldots, m-1\}$ such that if $b\in q$, then every prefix of $b$ (including the empty word $\varepsilon$) also belongs to $q$. Note that a prefix closed subset of $\bB$ uniquely defines a tree with $\varepsilon$ as the root.
%
The intuition here is that each element of $q$ corresponds to a block that was put into the state $q$ by some player. The genesis block corresponds to $\varepsilon$. When a player $p$ decides to mine on top of a block $b$, she puts another block into the state defined by the string $b\cdot p$, where we use notation $b_1 \cdot b_2$ for the concatenation of two strings $b_1$ and $b_2$.
%
Notice that with this terminology, given $b_1, b_2 \in q$, we have that $b_2$ is a descendant of $b_1$ in $q$ if $b_1$ is a prefix of $b_2$, which is denoted by $b_1 \preceq b_2$. Moreover, a path in $q$ is a nonempty set $\pi$ of blocks from $q$ for which there exist blocks $b_1, b_2$ such that $\pi = \{ b \mid b_1 \preceq b$ and $b \preceq b_2\}$; in particular, $b_2$ is a descendant of $b_1$ and $\pi$ is said to be a path from $b_1$ to $b_2$.
Finally, let $\bQ$ be the set of all possible states in a game played by $m$ players, and for a state $q \in \bQ$, let $|q|$ be its size, measured as the cardinality of the set $q$ of strings (or blocks).

The {\em blockchain} of a state $q$, denoted by $\bchain(q)$, is the path $\pi$ in $q$ of largest length, in the case that this path is unique.
%, in which case we denote it by $\bchain(q)$ \francisco{``in which case'' and ``in this case'' is redundant}. 
If two or more different paths in $q$ are tied for the longest, then we say that the blockchain in $q$ does not exist, and we assume that $\bchain(q)$ is not defined (so that $\bchain(\cdot)$ is a partial function). Notice that if $\bchain(q) = \pi$, then $\pi$ has to be a path in $q$ from the genesis block $\varepsilon$ to a leaf of $q$.
%If the blockchain of $q$ is defined, say $b = \bchain(q)$, then a block $b' \in q$ is said to belong to the blockchain of $q$ if $b' \in \epath(b)$, that is, if $b'$ is a block in the path from the root $\varepsilon$ to $b$ in the state $q$.

\begin{example}\label{ex-mining}
The following picture shows a state $q$ of the game assuming that $\bP =\{0,1\}$:
\begin{center}
\begin{tikzpicture}[->,>=stealth',auto,thick, scale = 0.55,state/.style={circle,inner sep=2pt}]
	\node [state] at (-0.2,0) (r) {$\varepsilon$};
	\node [state] at (1.5,0.75) (n0) {$0$};
	\node [state] at (1.5,-0.75) (n1) {$1$};
	\node [state] at (3.3,-0.75) (n11) {$11$};
	\node [state] at (5.4,0) (n110) {$110$};
	\node [state] at (5.4,-1.5) (n111) {$111$};
	\node [state] at (7.7,-1.5) (n1111) {$1111$};
	\node [state] at (10.2,-1.5) (n11110) {$11110$};

	\path[->]
	(r) edge (n1)
	(r) edge (n0)
	(n1) edge (n11)
	(n11) edge (n110)
	(n11) edge (n111)
	(n111) edge (n1111)
	(n1111) edge (n11110);
\end{tikzpicture}
\end{center}
In this case, we have that $q = \{\varepsilon, 0, 1, 11, 110, 111, 1111, 11110\}$, so $q$ is a finite and prefix-closed subset of $\bB = \{0,1\}^*$. The owner of each block $b \in q \smallsetminus \{ \varepsilon\}$ is given by the the last symbol of $b$; for instance, we have that $\owner(11) = 1$ and $\owner(11110) = 0$. Moreover, the longest path in $q$ is $\pi = \{\varepsilon, 1, 11, 111, 1111, 11110\}$, so that the blockchain of $q$ is $\pi$ (in symbols, $\bchain(q) = \pi$).
%Thus, we have that the blocks that belong to the blockchain of $q$ are $\varepsilon$, $1$, $11$, $111$, $1111$, $11110$, since these are precisely the blocks in $q$ in the path from $\varepsilon$ to $11110$ (in symbols, $\epath(\bchain(q)) = \{\varepsilon, 1, 11, 111, 1111, 11110\}$). Notice that the information about these blocks is important as the reward of each player depends on the blocks that belong to her in the blockchain. For instance, if the reward for each block is a constant $c$, then we have that the reward of player 1 is $4 c$ in $q$, as four blocks in the blockchain of $q$ belong to this player.
Finally, 
%notice that 
$|q| = 8$, as $q$ is a set consisting of eight blocks (including the genesis block $\varepsilon$). 

	Assume now that $q'$ is the following state of the game:
	\begin{center}
\begin{tikzpicture}[->,>=stealth',auto,thick, scale = 0.55,state/.style={circle,inner sep=2pt}]
	\node [state] at (-0.2,0) (r) {$\varepsilon$};
	\node [state] at (1.5,0.75) (n0) {$0$};
	\node [state] at (1.5,-0.75) (n1) {$1$};
	\node [state] at (3.3,-0.75) (n11) {$11$};
	\node [state] at (5.4,0) (n110) {$110$};
	\node [state] at (5.4,-1.5) (n111) {$111$};

	\path[->]
	(r) edge (n1)
	(r) edge (n0)
	(n1) edge (n11)
	(n11) edge (n110)
	(n11) edge (n111);
\end{tikzpicture}
\end{center}
%In this case, 
We have that $\bchain(q')$ is not defined since the paths $\pi_1 = \{\varepsilon, 1, 11, 110\}$ and $\pi_2 = \{\varepsilon, 1, 11, 111\}$ are tied for the longest path in $q'$. \qed
\end{example}


Real-life Bitcoin blocks also contain transactions that indicate movement of money in the system, and thus there are
several different blocks that a player $p$ can use to extend the current state when mining upon a block $b$ (e.g, depending on the ordering of transactions, or the nonce being used to announce the block). Since we are interested %primarily 
in miners behaviour, we just focus on the owner of the block following $b$, and do not consider the possibility of two different blocks belonging to $p$ being added on top of $b$. 
%Alternatively, if we consider the Bitcoin protocol, we could say that all the different blocks that $p$ can put on top of $b$ are considered as~equivalent.
%\juan{moving this to the reward section}
%\etienne{We may have to specify that it is under the assumption that there are no fees or that they are negligible.} \marcelo{Etienne is right about this, depending on the transactions included in the block the reward could be different. I don't think we should talk about reward here.}

\subsubsection{Actions of a miner}\label{sub:actions}
On each step, each miner
% looking to maximize their rewards 
chooses a block in the current state, and attempts to mine on top of this block. Thus, in each turn, each of the players race to place the next block in the state, and only one of them succeeds. The probability of succeeding is directly related to the comparative amount of hash power available to this player, the more hash power the likely it is that she will mine the next block before the rest of the players. Once a player places a block, this block is added to the current state, obtaining a different state from which the game continues.

Let $p \in \bP$. Given a block $b \in \bB$ and a state $q \in \bQ$, we denote by $\mine(p,b,q)$ an action played in the mining game in which player $p$ mines on top of block $b$. Such an action $\mine(p,b,q)$ is considered to be valid if $b \in q$ and $b\cdot p \not\in q$. The set of valid actions for player $p$ is collected in the set:
\begin{multline*}
\bA_p \ = \ \{ \mine(p,b,q) \mid b \in \bB, q \in \bQ \text{ and}\\\mine(p,b,q) \text{ is a valid action}\}.
\end{multline*}
Moreover, given $a \in \bA_p$ with $a = \mine(p,b,q)$, the result of applying $a$ to $q$, denoted by $a(q)$, is defined as the state $q \cup \{b \cdot p\}$. Finally, we denote by $\bA$ the set of combined actions for the $m$ players, that is, $\bA = \bA_0 \times \bA_1 \times \cdots \times \bA_{m-1}$.

\subsubsection{The pay-off of a miner}
%The next and last component of our game is the reward for miners: how much do they earn for each mined block. 
Most cryptocurrencies based on blockchain follow this miner's payment rules:
%, and Bitcoin in particular. 
%
%\noindent
(1) Miners receive a one-time reward per each block they mine. For example, in Bitcoin, each miner will receive a pre-set reward when she mines a block (12.5BTC at the time of writing).
%
%\noindent
(2) The only blocks that are valid are those in the blockchain; if a block is not in the blockchain then the reward given for mining this block cannot be spent. To prevent drastic changes on what are valid blocks, the Bitcoin protocol enforces that a block reward can only be spent when there are more than one hundred blocks on top of it. 

So how should the reward function look like? The first naive idea is to give miners the block reward as soon as they put a block at the top of the current blockchain. This provides some incentive to extend the blockchain, however, it does not completely protect the system from forks that (potentially malicious) miners might want to initiate. To illustrate this, consider the state $q'$ in Example \ref{ex-mining}, where we have two blocks (110 and 111) competing to be in the blockchain, and consider 111 to be the last block added to reach $q'$. If player 0 already cashed in the reward for the block 110 (since this was awarded immediately), she has no further incentive to mine on top of this block, as the block 1110 would give her an equal reward. This can then allow a malicious miner (1 in this example) to do a local fork (e.g. the block 111), where for instance some transactions are being blocked, and does not provide an incentive for other miners to protect the other branch competing to be the blockchain. The way that the Bitcoin protocol goes around this is by paying for blocks which are buried deep into the blockchain.
% (they have been confirmed one hundred times). 
This now gives a strong incentive to player 0 to mine on top of the block 110, since she cannot cash in her reward immediately after mining this block, but only once it has been extended many times.

In Section \ref{sec-pay-uti}, 
%we discuss the difficulty of defining a reward function which follow these rules, and 
we explain how our
%therefore propose a
 pay-off model 
 %that
 mimics 
 %those incentives. 
 the rules and incentives mentioned in the previous paragraphs. For now we assume, for each player $p \in \bP$, the existence of a reward function $r_p : \bQ \to \mathbb{R}$ such that the reward of $p$ in a state $q$ is given by $r_p(q)$. Moreover, the combined reward function of the game is $\bR = (r_0, r_1, \ldots, r_{m-1})$. 

\subsubsection{Transition probability function}\label{sub:transi}

As a last component of the game, we assume that $\pr : \bQ \times \bA \times \bQ \to [0,1]$ is a transition probability function satisfying that for every state $q \in \bQ$ and combined action $\ba = (a_0, a_1, \ldots, a_{m-1})$ in $\bA$:
\begin{eqnarray*}\label{eq-prop}
\sum_{p=0}^{m-1} \pr(q, \ba, a_p(q)) & = & 1.
\end{eqnarray*}
Notice that if $p_1$ and $p_2$ are two different players, then for every action $a_1 \in \bA_{p_1}$, every action $a_2 \in \bA_{p_2}$ and every state $q \in \bQ$, it holds that $a_1(q) \neq a_2(q)$. Thus, we can think of $\pr(q, \ba, a_p(q))$ as the probability that player $p$ places the next block, which will generate the state $a_p(q)$. As we have mentioned, such a probability is directly related with the hash power of player $p$, the more hash power the likely it is that action $a_p$ is executed and $p$ mines the next block before the rest of the players. In what follows, we assume that the hash power of each player does not change during the mining game, which is captured by the following condition:
%\begin{itemize}
%\item Given $q, q' \in \bQ$ and $\ba, \ba' \in \bA$ with $\ba = (a_0, a_1$, $\ldots$, $a_{m-1})$ and $\ba' = (a'_0, a'_1, \ldots, a'_{m-1})$, for every player $p \in \bP$ it holds that $\pr(q, \ba, a_p(q)) = \pr(q', \ba', a'_p(q'))$.
%\end{itemize}
%Thus, we assume from now on that this condition is satisfied. In particular, 
for each player $p \in \bP$, we 
%assume that 
have that $\pr(q, \ba, a_p(q)) = h_p$ for every $q \in \bQ$ and $\ba \in \bA$ with $\ba = (a_0, a_1, \ldots, a_{m-1})$. We refer to such a fixed value $h_p$ 
%and we refer to $h_p$ 
as the hash power of player $p$.
%Moreover, we define $\bH = (h_0, h_1, \ldots, h_{m-1})$ as the hash power distribution, and we replace $\pr$ by $\bH$ in the definition of an an infinite stochastic game, so that $\Gamma = (\bP,\bA,\bQ,\bR,\bH)$.
Moreover, we assume that $h_p > 0$ for every player $p \in \bP$, as if this not the case then $p$ can just be removed from the mining game.


\subsubsection{The mining game: definition, strategy, utility and equilibrium}
Putting together the components defined in the previous sections, we have that 
%Summing up, from now on we consider 
an infinite stochastic game is a tuple $(\bP,\bQ,\bA,\bR,\pr)$, where $\bP$ is the set of players, $\bQ$ is the set of states, $\bA$ is the set of combined actions, $\bR$ is the combined pay-off function 
and $\pr$ is the transition probability function. 

%The pay-off function of the game denoted $\bR = (r_0,r_1 \cdots, r_{m-1})$, with $r_p : \bQ \mapsto \mathbb{R^+}$ is the pay-off function of the player $p$, represent the reward that each player receive at any state of the game. Before presenting our model for the pay-off function we have to talked about the purpose of using stochastic game as a model.

We have defined a game that can capture miners' interactions. A fundamental component of such a game is the strategy that each player decides to take, which combined determine the utility of each player. In particular, miners take actions and decide about strategies trying to maximize their utility. In this sense, a 
%stationary 
equilibrium of the game is a fundamental piece of information about miners' behavior, because such an equilibrium is a combination of players' strategies where no miner has an incentive to perform a different action. The notions of strategy, utility and stationary equilibrium are the last components of our game-theoretical characterization of mining in Bitcoin, and they are defined next. 

A Markov strategy (or just strategy) for a player $p \in \bP$ is a function $s : \bQ \rightarrow \bA_p$.
We define $\bS_p$ as the set of all strategies for player $p$, and $\bS = \bS_0 \times \bS_{1} \times \cdots \times \bS_{m-1}$ as the set of combined strategies for the game (recall 
%that we are assuming 
that $\bP = \{0, \ldots, m-1\}$ is the set of players).
%As usual, we are interested in understanding which strategies are better than others, which we capture by the notions of utility and equilibrium. 
To define the notions of utility and equilibrium, we need some additional notation.
Let $\bs = (s_0, \ldots, s_{m-1})$ be a combined strategy.
% in $\bS$. 
Then given $q \in \bQ$, define $\bs(q)$ as the combined action $(s_0(q), \ldots, s_{m-1}(q))$. Moreover, given an initial state $q_0 \in \bQ$,
the probability of reaching state $q \in \bQ$, denoted by $\pr^{\bs}(q \mid q_0)$, is defined as 0 if $q_0 \not\subseteq q$ (that is, if $q$ is not reachable from $q_0$), and otherwise it is recursively defined as follows: if $q = q_0$, then $\pr^{\bs}(q \mid q_0) = 1$; otherwise, we have that $|q| - |q_0| = k$, with $k \geq 1$, and
$$
\pr^{\bs}(q \mid q_0) =
\sum_{\substack{q' \in \bQ \,:\\ q_0 \subseteq q' \text{ and } |q'| - |q_0| = k-1}}
 \pr^{\bs}(q' \mid q_0) \cdot \pr(q', \bs(q'), q).
 $$
In this definition, if for a player $p$ we have that $s_p(q') = a$ and $a(q') = q$, then $\pr(q', \bs(q'), q) = h_p$. Otherwise, we have that $\pr(q', \bs(q'), q) = 0$. Notice that this is well defined, since there can be at most one player $p$ whose action in the state $q'$ leads us to the state $q$.
For the sake of presentation, the probability of reaching a state $q$ from the initial state $\{\varepsilon\}$ (consisting only of the genesis block $\varepsilon$) is denoted by $\pr^{\bs}(q)$, instead of $\pr^{\bs}(q \mid \{\varepsilon\})$. 
%Hence, consistently with the simplification described before, we can replace the transition probability function $\pr$ by the hash power distribution $\bH$ when computing $\pr^{\bs}(q \mid q_0)$.
%It should be noticed that 
The framework just described corresponds to a Markov chain; in particular, the probability of reaching a state from an initial state is defined in the standard way for Markov chains~\cite{MU05}. However, we are not interested in the steady distribution for such a Markov chain and, thus, we do not explore this connection in this paper.

We finally have all the necessary ingredients to define the utility of a player in a mining game given a particular strategy. As is common
when looking at personal utilities, we define it as the summation of the expected rewards, and choose
to impose a discount for future rewards using a factor $\beta \in (0,1)$.

\begin{mydef}
\label{def-utility}
The $\beta$--discounted utility of player $p$ for the strategy $\bs$ from the state $q_0$ in
the mining game, denoted by $u_p(\bs \mid q_0)$, is defined as:
\begin{eqnarray*}
u_p(\bs \mid q_0) & = & (1 - \beta) \cdot \sum_{q \in \bQ \,:\, q_0 \subseteq q} \beta^{|q|-|q_0|} \cdot r_p(q) \cdot \pr^{\bs}(q \mid q_0).
\end{eqnarray*}
\end{mydef}
Notice that the value $u_p(\bs \mid q_0)$ may not be defined if the series $\sum_{q \in \bQ \,:\, q_0 \subseteq q} \beta^{|q|-|q_0|} \cdot r_p(q) \cdot \pr^{\bs}(q \mid q_0)$ diverges. To avoid this problem, from now on we assume that for every pay-off function $\bR = (r_0, \ldots, r_{m-1})$, there exists a polynomial $P$ such that $|r_p(q)| \leq P(|q|)$ for every player $p \in \bP$ and state $q \in \bQ$. Under this simple yet general condition, which is satisfied by the pay-off functions considered in this paper and in other game-theoretical formalizations of Bitcoin mining \cite{mininggames:2016}, it is possible to prove that $u_p(\bs \mid q_0)$ is a real number. Moreover, as for the definition of the probability of reaching a state from the initial state $\{\varepsilon\}$, we use notation $u_p(\bs)$ for the $\beta$--discounted utility of player $p$ for the strategy $\bs$ from $\{\varepsilon\}$, instead of $u_p(\bs \mid \{\varepsilon\})$. 
%(see Appendix \ref{sec-conver} for a proof of this property).

As a last ingredient in our formalization, we need to introduce the notion stationary equilibrium.
Given a player $p \in \bP$, a combined strategy $\bs \in \bS$, with $\bs = (s_0,s_1, \ldots, s_{m-1})$, and a strategy $s$ for player $p$ ($s \in \bS_p$), we denote by $(\bs_{-p}, s)$ the strategy $(s_0, s_1, \ldots s_{p-1},s,s_{p+1}, \ldots, s_{m-1})$.
\begin{mydef}
A strategy $\bs$ is a $\beta$--discounted stationary equilibrium from the state $q_0$ in the %infinite
mining game if for every player $p \in \bP$ and every strategy $s$ for player $p$ $(s \in\bS_p)$, it holds that:
\begin{eqnarray*}u_p(\bs \mid q_0) & \geq & u_p ((\bs_{-p},s) \mid q_0).
\end{eqnarray*}
\end{mydef}

\subsection{On the pay-off and utility of a miner}
\label{sec-pay-uti}
%If we assume that long forks are impossible, a reward function which give miners their complete payoff as soon as a block is burried deep enough in the blockchain can be build \cite{mininggames:2016}. However the goal of a game-theoretical formalisation is to prove assumption such as: \textit{long forks are impossible}. Therefore long fork is a situation that our model should not only allow, but also reward in agreement with the cryptocurrency's payment system. And what does the rules state? When a long fork happens each block buried deep enough in the new blockchain generate a one-time payment to their miner, while the blocks from the previous longest chain become invalid. However this rule can not be enforce by a stochastic game's reward function. Indeed, consider a situation where two distinct chains are competing to be the longest, and finally settle. One of the chain has become the blockchain, hence according to the rule, only its blocks which had not generate payment before should give reward to their owner. But due to the memoryless nature of reward functions in stochastic games, there is no possibility to distinguish already and never paid blocks from each others.

%Hence we 
We design our pay-off model to mimic the incentives of the payment system of Bitcoin and other cryptocurrencies. More precisely, given a player $p$ and a state $q$, for every block $b \in q$ assume that the reward obtained by $p$ for the block $b$ in $q$ is given by $r_p(b,q)$, so that $r_p(q) = \sum_{b \in q} r_p(b,q)$. This decomposition can be done in a natural and straightforward way for the pay-off functions considered in this paper and in other game-theoretical formalizations of Bitcoin mining \cite{mininggames:2016}. 
%the reward of a player $p$ in a state $q$, denoted $r_p(q)$ is defined as $r_p(q)= \sum_{b \in \bchain(q)} r_p(b,q)$, where $r_p(b,q) > 0$ is the reward that player $p$ receives for a block $b$ she owns in the blockchain of the state $q$, and equals zero if $\bchain(q)$ is not defined. 
Then to enforce the fact that the block reward for the block $b$ is not granted immediately, we pay in Definition \ref{def-utility} a 
%($\beta$-discounted) 
portion of $r_p(b,q)$, for each state $q$ where $b$ is in. In other words, if a miner owns a block, then she will be rewarded for this block in every state where this block is part of the blockchain, in which case $r_p(b,q) > 0$. %This type of reward functions is general enough to express other game-theoretical formalizations of Bitcoin mining \cite{mininggames:2016}.

This means that we might pay the miner infinitely many times for a single block. A natural question is then whether we overpay for the blocks. This is where the discount factors in our definition of utility come into play.  More precisely, we pay a portion of block $b$'s reward each time it is included in the current blockchain. In other words, 
when a player mines a new block, she will receive the full amount for this block only if she manages to maintain the block in the blockchain up to infinity. Otherwise, if this block 
ceases to be in the blockchain, we only pay a fraction of the full amount. Formally, given a combined strategy $\bs$, we can define the utility of a block $b$ for a player $p$, denoted by $u_p^b(\bs)$,  as follows:
\begin{eqnarray*}
u_p^b(\bs) & =  & (1 - \beta) \cdot  \sum_{q \in \bQ \,:\, b \in \bchain(q)} \beta^{|q|-1} \cdot  r_p(q,b) \cdot \pr^{\bs}(q).
\end{eqnarray*}
For simplicity, here we assume that the game starts in the genesis block $\varepsilon$, and not in an arbitrary state $q_0$. The discount factor in this case is $\beta^{|q|-1}$, since $|\{\varepsilon\}|= 1$.  


To see that we pay the correct amount for each block, assume that there is a maximum value for the reward of a block $b$ for player $p$, which is denoted by $M_p(b)$. Thus, we have that there exists $q_1 \in \bQ$ such that $b \in q_1$ and $M_p(b) = r_p(b,q_1)$, and for every $q_2 \in \bQ$ such that $b \in q_2$, it holds that $r_p(b,q_2) \leq M_p(b)$. Again, such an assumption is satisfied by most currently circulating cryptocurrencies, by the pay-off functions considered in this paper, and by other game-theoretical formalizations of Bitcoin mining \cite{mininggames:2016}. Then we have that:
\begin{myprop}\label{prop-ub-block}
For every player $p \in \bP$, block $b \in \bB$ and combined strategy $\bs \in \bS$, it holds that: \ \ $u^b_p(\bs)  \leq   \beta^{|b|} \cdot M_p(b)$.
%\begin{eqnarray*}
%u^b_p(\bs) & \leq &  \beta^{|b|} \cdot M_p(b).
%\end{eqnarray*}
\end{myprop}
Thus, the utility obtained by player $p$ for a block $b$ is at most $\beta^{|b|} \cdot M_p(b)$, that is, the maximum reward that she can obtained for the block $b$ in a state multiplied by the discount factor $\beta^{|b|}$, where $|b|$ is the minimum number of steps that has to be performed to reach a state containing $b$ from the initial state $\{\varepsilon\}$. 
Moreover, a miner can only aspire to get the maximum utility for a block $b$ if once $b$ is included in the blockchain, it stays in the blockchain in every future state. This tell us that our framework puts a strong incentive for each player in maintaining her blocks in the blockchain.

%Of course, one could ask why not simply reward a portion of the block's reward until it is buried hundred blocks deep into the blockchain? One can think of the Bitcoin's protocol working in this way, and it is also studied in other game theoretic formalizations of the protocol \cite{??}. Apart from being able to reason about games in a more elegant way, the main motivation for this approach is that it allows us to prove assumptions such as that arbitrarily long forks are virtually impossible.  


\begin{comment}
\subsection{On the pay-off and utility of a miner}\label{sec-pay-ut}
Blockchain protocols enforce that miners receive a reward once and only once for each block they own
in the blockchain. However due to the nature of pay-off functions in stochastic games such rule can not be enforced. In fact, as the pay-off function does not rely on the history of states, for any pay-off function there exists a sequence of plays such that a miner has never received a reward for a block belonging to the blockchain, or he has received it several times. \marcelo{I think this should be replaced by a less technical explanation, something easier to understand. In particular, it is not clear how this is obtained.}

In this section, we show how this constraint is simulated in our framework by combining the notions of pay-off function and $\beta$--discounted utility introduced in the previous section. \etienne{We should change enforced by something just a bit weaker and it would be perfect.} \marcelo{I used simulated instead of enforced, is it better?}

In our framework, a miner is paid in every state according to the blocks she has in the blockchain. Such a pay-off includes not only the final block in the blockchain if she won the mining race, but also the blocks she could have put in the blockchain in some previous states. This property can be formalized as follows.
Given a player $p$ and a state $q$, for every block $b \in q$ assume that the reward obtained by $p$ for the block $b$ in the state $q$ is given by $r_p(b,q)$, so that $r_p(q) = \sum_{b \in q} r_p(b,q)$. This decomposition can be done in a natural and straightforward way for the pay-off functions considered in this paper and in other game-theoretical formalizations of Bitcoin mining \cite{mininggames:2016}. Then the aforementioned property is satisfied by all these pay-off functions in the following sense: given a player $p$ and a state $q$, it holds that $r_p(q,b) > 0$ for every block $b$ such that $b \in \bchain(q)$ and $\owner(q) = p$. 
In this way, given that $r_p(q) = \sum_{b \in q} r_p(b,q)$, our formalization puts a strong incentive for each player in maintaining her blocks in the blockchain, but without ruling out the possibility that in some cases deviating from the blockchain can be a better alternative for a player. This is the first desirable property of our formalization.

A natural question at this point is whether the property mentioned in the previous paragraph implies that a miner is rewarded multiple times for the same block (once for each state where the block belongs to the blockchain), given the definition of the utility function. In what follows, we formally shows that this is not the case in the sense that the reward for a block can be paid at most once, which is the second desirable property of our formalization.

%Given a player $p$ and a state $q$, for every block $b \in q$ assume that the reward obtained by $p$ for the block $b$ in the state $q$ is given by $r_p(b,q)$, so that $r_p(q) = \sum_{b \in q} r_p(b,q)$. This decomposition can be done in a natural and straightforward way for the pay-off functions considered in this paper and in other game-theoretical formalizations of Bitcoin mining \cite{mininggames:2016}. We propose a pay-off where on every states, we pay miners for each of the blocks they already have in the blockchain, plus the new block they will potentially mine. Therefore for any player $p$ and any state $q$ we have that $r_p(q,b) \neq 0$ if $q \in bc(q)$ and $\owner(q) = p$. And the pay-off function verify $r_p(q) = \sum_{b \in q} r_p(b,q)$.
%This model clearly puts a strong incentive in maintaining blocks in the blockchain and does not nullify the incentive to fork. The main concern is that if we consider a sequence of plays we end-up giving reward to a miner multiple times for the same block (once for each state where the block belongs to the blockchain), but to analyse the pay-off of a sequence of plays one should focus on the utility. 

Definition~\ref{def-utility} corresponds to the usual notion of average discounted utility \marcelo{A citation is needed here}. In particular, if the starting point of the game is a state $q_0$, then for every state $q$ such that $q_0 \subseteq q$, the pay-off of a player $p$ in $q$ is $\beta^{|q|-|q_0|} \cdot r_p(q)$, where $|q|-|q_0|$ is the number of steps that have to be performed to reach $q$ from $q_0$ so the discount factor $\beta^{|q|-|q_0|}$ has to be applied. The uncertainty about reaching state $q$ from $q_0$ is taking into consideration by including the term $\pr^{\bs}(q \mid q_0)$, which tell us that the expected pay-off of player $p$ for the state $q$ is $\beta^{|q|-|q_0|} \cdot r_p(q) \cdot \pr^{\bs}(q \mid q_0)$.
%
%As a last comment on the definition of utility, 
%
But notice that although this definition of expected pay-off is the natural one,
a block $b$ can be included in an infinite number of states $q$ such that $\pr^{\bs}(q \mid q_0) > 0$, which is in contradiction with the aforementioned blockhain protocol's rule that a miner receives a reward once and only once for each block she owns in the blockchain. To solve this issue, the term $(1 - \beta)$ is included in the definition of utility, as shown next.
% But the reward obtained by a player $p$ for this block $b$ should not be added more than once, which is the reason to include the term $(1 - \beta)$ in the definition of utility. Let us formalize this claim in more precise terms.
Given a combined strategy $\bs$, we can naturally define the utility of a block $b$ for a player $p$, denoted by $u_p^b(\bs)$, as follows:
\begin{eqnarray*}
u_p^b(\bs) & = & (1 - \beta) \cdot \sum_{q \in \bQ \,:\, b \in q} \beta^{|q|-1} \cdot r_p(q,b) \cdot \pr^{\bs}(q).
\end{eqnarray*}
For the sake of readability, we assume that the game is starting from the initial state $\{\varepsilon\}$ that consists only of the genesis block. Notice that $|\{\varepsilon\}| = 1$, so that the discount factor for a state $q$ is $\beta^{|q|-1}$. Now assume that there is a maximum value for the reward of a block $b$ for player $p$, which is denoted by $M_p(b)$. Thus, we have that there exists $q_1 \in \bQ$ such that $b \in q_1$ and $M_p(b) = r_p(b,q_1)$, and for every $q_2 \in \bQ$ such that $b \in q_2$, it holds that $r_p(b,q_2) \leq M_p(b)$. Again, such an assumption is satisfied by the pay-off functions considered in this paper and in other game-theoretical formalizations of Bitcoin mining \cite{mininggames:2016}. Then we have that:
\begin{myprop}\label{prop-ub-block}
For every player $p \in \bP$, block $b \in \bB$ and combined strategy $\bs \in \bS$, it holds that:
\begin{eqnarray*}
u^b_p(\bs) & \leq & \beta^{|b|} \cdot M_p(b).
\end{eqnarray*}
\end{myprop}
Thus, the utility obtained by player $p$ for a block $b$ is at most $\beta^{|b|} \cdot M_p(b)$, that is, the maximum reward that she can obtained for the block $b$ in a state multiplied by the discount factor $\beta^{|b|}$, where $|b|$ is the minimum number of steps that has to be performed to reach a state containing $b$ from the initial state $\{\varepsilon\}$. 
Moreover, a miner can only aspire to get the maximum utility for a block $b$ if once $b$ is included in the blockchain, it stays in the blockchain in every future state. This again tell us that our framework puts a strong incentive for each player in maintaining her blocks in the blockchain.
%Which make sense as it means the miner get maximal value if he can spend is money whenever he wants to.


\iffalse
Given a player $p$ and a combined strategy $\bs$, we have that: 
\begin{eqnarray*}
u_p(\bs \mid \{\varepsilon\}) & = & \sum_{b \in \bB} u^b_p(\bs \mid \{\varepsilon\}).
\end{eqnarray*}
Therefore, we know from Proposition \ref{prop-ub-block} that the reward of $p$ for a block $b$ is not accumulated more than once in the utility of $p$ for the combined strategy $\bs$. In fact, we obtained as a corollary of Proposition \ref{prop-ub-block} that:
\begin{mycor}\label{cor-ub-ut}
For every player $p \in \bP$ and combined strategy $\bs \in \bS$, it holds that:
\begin{eqnarray*}
u_p(\bs \mid \{\varepsilon\}) & \leq & \sum_{b \in \bB} \beta^{|b|} \cdot M_p(b).
\end{eqnarray*}
\end{mycor}
\fi

\end{comment}
