\begin{abstract}
In the consensus protocol used in most cryptocurrencies, participants are expected to perform actions for which they receive a reward. These actions are colloquially known as mining, and protocols must ensure that mining is reward-compatible: participants are expected to behave correctly if they want to maximise their gains. Yet studying incentives behind mining actions in cryptocurrencies has proven a difficult task, leaving us without a clear understanding of optimal miner behaviour.

Our goal is to understand under which circumstances are miners incentivised to fork the blockchain by mining on top of blocks which are not at the tail of the current chain. In other word we study what are the conditions for which the default mining behaviour is better than any branching strategy. We model mining in cryptocurrencies as a stochastic game in which players always try to produce new blocks, and have to immediately disclose them to other. The probability of a player producing a new block depends on the hash power of the player, that is, the proportion of the computational power he controls compared to the power of the entire network. We work under the assumption that a miner's pay-off increase as long as his blocks are part of the blockchain. Players then look to maximise their utility in the long run, and their best strategy depends on their hash power and the discount applied to the reward of each new block being mined. We show that  if no discount is offered, then miners do not have any incentive to fork, no matter how high their hash power is. On the other hand, when working with discounts similar to those in Bitcoin, miners can have an incentive to fork; however the minimal proportion of hash power for which it happens is close to half. 
\end{abstract}
