%!TEX root = main.tex

\subsection{Proof of Theorem \ref{thm-dificil}}

\label{sec-evaluation-G}
In this section, we develop analytical expressions for the utilities when
considering the strategy $\gup^k_\ell$ against a default player. In fact, it is
possible to express $u(\gup^k_\ell)$ as a rational function (\ie, a quotient of
two polynomials). In particular, these functions can be evaluated to any degree
of precision, which allowed us to establish the results of section
\ref{sec-dec}.


There are two miners (or, more generally, pools of miners), labelled Miner 0
and Miner 1. Miner 0 follows the \cdf{} strategy, that is, they will always
mine on top of the longest chain, regardless of where their past blocks are. On
the other hand, Miner 1 plays with the $\gup^k_\ell$ strategy: given a portion
of at most $k$ blocks not owned by her at the end of the blockchain, she forks
at the beginning of this chain (we refer to $k$ as the disadvantage). When
forking, she establishes a give-up length $\ell$ such that she will give up the
fork if the main branch achieves to append $\ell$ blocks, orphaning all mined
blocks on the fork. This scenario captures the fact that, with reasonable hash
power, a pool does not want to fork too far behind, and will give up the fork
if it does not prove successful, before reaching a hopeless situation.



%A_1 & = & \frac{\alpha \cdot \beta \cdot h}{1 - \alpha}\\
%A_2 & = & \alpha \cdot \beta \cdot h\\

\begin{mythm}
Miner 1, with hash power $h$, fixes a disadvantage $k$, a give-up length
$\ell$, and plays with $\gup^k_\ell$. Miner 0 plays with $\cdf$. 
The utility of miner 1 is given by
$$
u_1(\gup^k_\ell)(h) = \frac{\Phi_{k,\ell}(h)}{1-\Gamma_{k,\ell}(h)},
$$
where $\Phi_{k,\ell},\Gamma_{k,\ell}$ are polynomials in $h$ (with coefficients depending on $k,\ell,\alpha$ and $\beta$).
\end{mythm}
In the proof of this theorem, we develop precise expressions for
$\Phi_{k,\ell}, \Gamma_{k,\ell}$. These are combinations of two sets of
polynomials we call Trapezoidal and Pentagonal Dyck polynomials ($T_{n,m},
P_{n,m}$, see \ref{sec-trapezoid-pentagon}). In what follows, let $x = \beta^2
\cdot h \cdot (1 - h)$ for the ease of notation.
\begin{proof}
The game begins with both players mining on $\varepsilon$, and assume that
Miner 0 first appends $j$ blocks. Here, five types of states can arise:
\begin{itemize}
    \item[(0)] (Case $j=0$) Miner 1 immediately appends a block $b$ on top of
        $\varepsilon$ and now both players will mine on top of $b$.
    \item[(a)] After miner 0 appends $j\leq k$ blocks on top of $\varepsilon$,
        miner 1 appends one block on top of $\varepsilon$ and starts the fork.
        Miner 1 wins the fork, gaining at most $j+\ell+1$ blocks.
    \item[(b)] After miner 0 appends $j > k$ blocks on top of $\varepsilon$,
        miner 1 appends one block at position $j-k$. Miner 1 wins the fork,
        gaining at most $k+\ell+1$ blocks.  \item[(c)] After miner 0 appends
        $j\leq k$ blocks on top of $\varepsilon$, miner 1 appends one block on
        top of $\varepsilon$ and starts the fork. Miner 0's branch achieves
        length $\ell+1$ and miner 1 gives up.  \item[(d)] After miner 0 appends
$j > k$ blocks on top of $\varepsilon$, miner 1 appends one block at position
$j-k$. Miner 0's branch achieves length $\ell+1$ and miner 1 gives up.
\end{itemize}

A key observation is that after one of this cases, both players will mine over
the same block, and reboot the strategies as if this block were $\varepsilon$.
This recursive behaviour allows us to compute the utility of single forks in
the same spirit as in \ref{thm:always_fork}, and then use proper shifting to
establish an equation in $u_1(\gup^k_\ell)(h)$. In other words, we have

$$u_1(\gup^k_\ell)(h) = u_0(h) + \sum_{j=1}^k (u_{a,j}(h) + u_{c,j}(h)) +
\sum_{j=k+1}^\infty (u_{b,j}(h) + u_{d,j}(h)),$$ where each term corresponds to
the total utility contributed by the cases above. In what follows, we develop
expressions for each of these terms.

\begin{subsubsection}{Case (0)} Every state $q$ in this case is of the form
    $1\cdot w$, where $w\in \bstring$, and thus the utility amounts to
\begin{align*}
u_0(h) &= (1-\beta) \cdot \sum_{w\in \bstring}\beta^{1+|w|} (\alpha + \alpha\cdot r_1(w))\cdot h \cdot \pr(w).\\
    &= (1-\beta) \cdot\frac{\alpha\cdot \beta \cdot h}{1-\beta} + \alpha\cdot\beta\cdot h \cdot(1-\beta)\cdot\sum_{w\in \bstring}\beta^{|w|} r_1(w) \cdot \pr(w)\\
    &= \alpha\cdot \beta \cdot h + \alpha\cdot\beta\cdot h\cdot u_1(\gup^k_\ell)(h),
\end{align*}
\end{subsubsection}


\begin{subsubsection}{Case (a)}
These states are all of the form 
$$q = 0^j\cdot 1\cdot d\cdot 1\cdot w, d\in \Dyck_{j-1}, |d|+j \leq 2\ell, w\in \bstring,$$ 
where $\Dyck_{j-1}$ are Dyck words with disadvantage $j-1$. 

\begin{figure}[ht!]
\begin{tikzpicture}[->,>=stealth',auto,thick, scale = 1.0,state/.style={circle,inner sep=2pt}]

    % The graph
    \node [state] at (0,0) (R) {$\varepsilon$};
    \node [state] at (1,0) (1) {$|$};
    \node [state] at (2,0) (2) {$|$};
    \node [state] at (3,0) (3) {$|$};
    \node [state] at (4,0) (4) {$|$};
    \node [state] at (5,0) (5) {$|$};
    \node [state] at (6,0) (6) {$|$};
    \node [state] at (7,0) (7) {$|$};
    \node [state] at (8,0) (8) {$|$};

    \node [state] at (0.9,-1) (ghost) {$$};
    
    \node [state] at (1,-1) (one1) {$|$};
    \node [state] at (2,-1) (one2) {$|$};
    \node [state] at (3,-1) (one3) {$|$};
    \node [state] at (4,-1) (one4) {$|$};
    \node [state] at (5,-1) (one5) {$|$};
    \node [state] at (6,-1) (one6) {$|$};
    \node [state] at (7,-1) (one7) {$|$};
    \node [state] at (8,-1) (one8) {$|$};
    \node [state] at (9,-1) (one9) {$|$};
    \node [state] at (10,-1) (one10) {$w$};
    
    \node [state] at (6.1,-1) (ghost2) {$$};
    
    % Graph branches

    \path[->]
    (1,-1) edge (9.5,-1);
    

    \path[-]
    (R) edge (1,-1)
    (ghost) edge (ghost2)
    (R) edge (8,0);

    
    \path[<->,dashed]
    (1,.5) edge node {$j$} (5,.5)
    (6,.5) edge node {$r$} (8,.5);
\end{tikzpicture} 
\caption{(Case a) Miner 1 forked at $\varepsilon$ successfully, since $r+j \leq \ell$.}
\end{figure}


We have
\begin{align*}
    \scriptsize    \frac{u_{a,j}(h)}{1-\beta} &\scriptsize =\sum_{\begin{array}{c}\scriptscriptstyle w\in \bstring\\ \scriptscriptstyle d\in \Dyck_{j-1}\\ \scriptscriptstyle |d| \leq 2\ell-j\end{array}}\beta^{|d|+j+1+|w|}\cdot \bigg(\alpha^{\frac{|d|+j}{2}+1} \cdot r(w)+\sum_{i=0}^{\frac{|d|+j}{2}}\alpha^{i+1} \bigg)\cdot h^{\frac{|d|+j}{2}+ 1}\cdot (1-h)^{\frac{|d|+j}{2}}\cdot \pr(w).
\end{align*}
Simplifying this expression yields
\begin{align*}
u_{a,j}(h) &= 
\frac{\alpha\cdot\beta\cdot h}{1-\alpha}\cdot x^j \cdot 
\bigg(\sum_{d\in \Dyck_{j-1}, |d|\leq 2l - j}x^{\frac{|d|-j}{2}}\cdot (1-\alpha^{\frac{|d|+j}{2}+1})\bigg)\\
& 
+\alpha\cdot\beta\cdot h \cdot(\alpha\cdot x)^j\cdot \bigg(\sum_{d\in \Dyck_{j-1}, |d|\leq 2l - j} (\alpha\cdot x)^{\frac{|d|-j}{2}}\bigg)\cdot\bigg(\sum_{w\in\bstring} \beta^{|w|}\cdot r_1(w)\cdot \pr(w)\bigg)\\
    &= \frac{\alpha\cdot\beta\cdot h}{1-\alpha}\cdot x^j\big(T_{j-1,\ell-j}(x)-\alpha^{j+1}\cdot T_{j-1,\ell-j}(\alpha\cdot x)) \;+\\
    &+ \alpha\cdot\beta\cdot h\cdot(\alpha \cdot x)^j\cdot T_{j-1,\ell-j}(\alpha\cdot x)\cdot u_1(\gup^k_\ell)(h),
\end{align*}
where $T_{n,m}$ is defined on section \ref{sec-trapezoid-pentagon}.

\end{subsubsection}

\begin{subsubsection}{Case (b)}
These states are of the form 
$$q = 0^j\cdot 1\cdot d\cdot 1\cdot w, d\in \Dyck_{k-1}, |d|+k \leq 2\ell, w\in \bstring.$$ 

\begin{figure}[ht!]

\begin{tikzpicture}[->,>=stealth',auto,thick, scale = 1.0,state/.style={circle,inner sep=2pt}]

    % The graph
    \node [state] at (-3,0) (R) {$\varepsilon$};
    \node [state] at (-2,0) (-2) {$|$};
    \node [state] at (-1,0) (-1) {$|$};
    \node [state] at (0,0) (0) {$|$};
    \node [state] at (1,0) (1) {$|$};
    \node [state] at (2,0) (2) {$|$};
    \node [state] at (3,0) (3) {$|$};
    \node [state] at (4,0) (4) {$|$};
    \node [state] at (5,0) (5) {$|$};
    \node [state] at (6,0) (6) {$|$};
    \node [state] at (7,0) (7) {$|$};
    \node [state] at (8,0) (8) {$|$};

    \node [state] at (0.9,-1.5) (ghost) {$$};
    
    \node [state] at (1,-1.5) (one1) {$|$};
    \node [state] at (2,-1.5) (one2) {$|$};
    \node [state] at (3,-1.5) (one3) {$|$};
    \node [state] at (4,-1.5) (one4) {$|$};
    \node [state] at (5,-1.5) (one5) {$|$};
    \node [state] at (6,-1.5) (one6) {$|$};
    \node [state] at (7,-1.5) (one7) {$|$};
    \node [state] at (8,-1.5) (one8) {$|$};
    \node [state] at (9,-1.5) (one9) {$|$};
    \node [state] at (10,-1.5) (one10) {$w$};
    
    \node [state] at (6.1,-1.5) (ghost2) {$$};
    
    % Graph branches

    \path[->]
    (1,-1.5) edge (9.5,-1.5);
    

    \path[-]
    (0,0) edge (1,-1.5)
    (R) edge (8,0);

    
    \path[<->,dashed]
    (-2,.5) edge node {$j$} (5,.5)
    (1,-.5) edge node[below] {$k$} (5,-.5)
    (6,.5) edge node {$r$} (8,.5);
\end{tikzpicture} 
\caption{(Case b) Miner 1 forked successfully, since $r+k \leq \ell$.}

\end{figure}

Analogously as before, we have
\begin{align*}
u_{b,j}(h) &= (1-\beta) \cdot
 \sum_{
 \mbox{\scriptsize
 $\begin{array}{c}
 w\in \bstring\\
  d\in \Dyck_{k-1}\\
  |d| \leq 2\ell-k
  \end{array}$
  }
 }\beta^{j+|d|+k+1+|w|}\cdot \bigg(\alpha^{j-k}\cdot \bigg(\sum_{i=0}^{(|d|+k)/2}\alpha^{i+1}\bigg) + \alpha^{(|d|-k)/2 + j + 1} \cdot r(w)\bigg) \cdot h^{(|d|+k)/2 + 1}\cdot (1-h)^{(|d|-k)/2 + j}\cdot \pr(w).
\end{align*}
Developing and simplifying this expression as the latter case yields
\begin{align*}
u_{b,j}(h) &= \frac{\alpha\cdot\beta\cdot h}{1-\alpha}\cdot \alpha^{j-k}\cdot \beta^{j+k}\cdot h^{k}\cdot (1-h)^{j}\big(T_{k-1,\ell-k}(x)-\alpha^{k+1}\cdot T_{k-1,\ell-k}(\alpha\cdot x)\big) \\
&+ \alpha\cdot\beta\cdot h\cdot \alpha^j\cdot\beta^{j+k}\cdot h^{k}\cdot(1-h)^{j}\cdot T_{k-1,\ell-k}(\alpha\cdot x)\cdot  u_1(\gup^k_\ell)(h).
\end{align*}

\end{subsubsection}


\begin{subsubsection}{Case (c)}
These states are of the form 
$$q = 0^j\cdot 1\cdot e\cdot w, e\in \Pent_{j-1,\ell-j+1}, w\in \bstring,$$ 
where $\Pent_{n,m}$ are all the binary strings with $m$ zeroes such that in every prefix, the number of ones minus the number of zeroes does not exceed $n$. The number of such strings is analyzed in section \ref{sec-trapezoid-pentagon}. Let us denote by $H(s)$ the Hamming weight of a binary string $s$. 

\begin{figure}[ht!]
\begin{tikzpicture}[->,>=stealth',auto,thick, scale = 1.0,state/.style={circle,inner sep=2pt}]

    % The graph
    \node [state] at (0,0) (R) {$\varepsilon$};
    \node [state] at (1,0) (1) {$|$};
    \node [state] at (2,0) (2) {$|$};
    \node [state] at (3,0) (3) {$|$};
    \node [state] at (4,0) (4) {$|$};
    \node [state] at (5,0) (5) {$|$};
    \node [state] at (6,0) (6) {$|$};
    \node [state] at (7,0) (7) {$|$};
    \node [state] at (8,0) (8) {$|$};
    \node [state] at (9,0) (9) {$|$};
    \node [state] at (10,0) (10) {$|$};
    \node [state] at (11,0) (11) {$w$};
    \node [state] at (0.9,-1) (ghost) {$$};
    \node [state] at (1,-1) (one1) {$|$};
    \node [state] at (2,-1) (one2) {$|$};
    \node [state] at (3,-1) (one3) {$|$};
    \node [state] at (4,-1) (one4) {$|$};
    \node [state] at (5,-1) (one5) {$|$};
    \node [state] at (6,-1) (one6) {$|$};
    \node [state] at (6.1,-1) (ghost2) {$$};
    
    % Graph branches

    \path[->]
    (R) edge (11);
    

    \path[-]
    (R) edge (1,-1)
    (ghost) edge (ghost2);
    
    \path[<->,dashed]
    (1,.5) edge node {$j$} (5,.5)
    (6,.5) edge node {$\ell-j+1$} (10,.5)
    (2,-1.5) edge node [below] {$r=H(e)$}(6,-1.5);
\end{tikzpicture} 
\caption{(Case c) Miner 1 forked at $\varepsilon$ but gave up, since the upper branch attained $\ell$ blocks.}

\end{figure}

We have
\begin{align*}
u_{c,j}(h) &= (1-\beta) \cdot \sum_{w\in \bstring, e\in \Pent_{j-1, \ell-j-1}} \beta^{\ell+H(e)+|w|+2}\cdot \alpha^{\ell+1}\cdot r_1(w) \cdot h^{H(e)+1}\cdot(1-h)^{\ell+1}\cdot \pr(w).
\end{align*}
This yields the following:
\begin{align*}
u_{c,j}(h) &= \alpha\cdot x\cdot (\alpha\cdot\beta\cdot(1-h))^\ell\cdot u_1(\gup^k_\ell)(h) \cdot  \sum_{e\in\Pent_{j-1,\ell-j-1}}(\beta\cdot h)^{H(e)}\\
        &= \alpha\cdot x\cdot (\alpha\cdot\beta\cdot(1-h))^\ell \cdot P_{\ell,j}(\beta\cdot h) \cdot u_1(\gup^k_\ell)(h).
\end{align*}


\end{subsubsection}


\begin{subsubsection}{Case (d)}
These states are of the form $q = 0^j\cdot 1\cdot e\cdot w, e\in \Pent_{k-1,\ell-k+1}, w\in \bstring.$

\begin{figure}[ht!]
\begin{tikzpicture}[->,>=stealth',auto,thick, scale = 1.0,state/.style={circle,inner sep=2pt}]

    % The graph
    \node [state] at (-3,0) (R) {$\varepsilon$};
    \node [state] at (-2,0) (-2) {$|$};
    \node [state] at (-1,0) (-1) {$|$};
    \node [state] at (0,0) (0) {$|$};
    \node [state] at (1,0) (1) {$|$};
    \node [state] at (2,0) (2) {$|$};
    \node [state] at (3,0) (3) {$|$};
    \node [state] at (4,0) (4) {$|$};
    \node [state] at (5,0) (5) {$|$};
    \node [state] at (6,0) (6) {$|$};
    \node [state] at (7,0) (7) {$|$};
    \node [state] at (8,0) (8) {$|$};
    \node [state] at (9,0) (9) {$|$};
    \node [state] at (10,0) (10) {$|$};
    \node [state] at (11,0) (11) {$w$};
    \node [state] at (0.9,-1.5) (ghost) {$$};
    \node [state] at (1,-1.5) (one1) {$|$};
    \node [state] at (2,-1.5) (one2) {$|$};
    \node [state] at (3,-1.5) (one3) {$|$};
    \node [state] at (4,-1.5) (one4) {$|$};
    \node [state] at (5,-1.5) (one5) {$|$};
    \node [state] at (6,-1.5) (one6) {$|$};
    \node [state] at (6.1,-1.5) (ghost2) {$$};
    
    % Graph branches

    \path[->]
    (R) edge (11);
    

    \path[-]
    (0,0) edge (1,-1.5)
    (ghost) edge (ghost2);
    
    \path[<->,dashed]
    (-2,.5) edge node {$j$} (6,.5)
    (1,-.5) edge node [below] {$k$} (6,-.5)
    (7,.5) edge node {$\ell-k+1$} (10,.5)
    (2,-2) edge node [below] {$r=H(e)$}(6,-2);
\end{tikzpicture} 
\caption{(Case d) Miner 1 forked at some block, but gave up, since the upper branch attained $\ell$ blocks.}
\end{figure}


As in case (c), we compute
\begin{align*}
u_{d,j}(h) &= (1-\beta) \cdot \sum_{w\in \bstring, e\in \Pent_{k-1, \ell-k-1}} \beta^{\ell+j-k+H(e)+|w|+2}\cdot \alpha^{\ell+j-k+1}\cdot r_1(w) \cdot h^{H(e)+1}\cdot(1-h)^{\ell+j-k+1}\cdot \pr(w)\\
        &= \alpha\cdot x\cdot(\alpha\cdot\beta\cdot(1-h))^{j+\ell-k}\cdot P_{\ell,k}(\beta\cdot h) \cdot u_1(\gup^k_\ell)(h).
\end{align*}



\end{subsubsection}

Putting all of these expressions together and summing over $j$ gives an equation for $u_1(\gup^k_\ell)(h)$, that solves to the claimed rational function. More precisely, we have
$$u_1(\gup^k_\ell)(h) = u_0 + \sum_{j=1}^k (u_{a,j} + u_{c,j}) + \sum_{j=k+1}^\infty (u_{b,j} + u_{d,j}) = \Phi_{k,\ell} + \Gamma_{k,\ell}\cdot u_1(\gup^k_\ell)(h),$$
where $\Phi$ is the total contribution of one fork (successful or not) and $\Gamma$ is the total shifting factor. First note that $u_{b,j}+u_{d,j}$ is a geometric sequence in $j$ with a ratio in $[0,1)$, therefore the summation converges. In particular, $\Phi_{k,\ell}$ and $\Gamma_{k,\ell}$ are polynomials in $h$ whose coefficients depend in $\alpha,\beta,\ell,k$.
\end{proof}

\subsection{Trapezoidal and Pentagonal Dyck Polynomials}
\label{sec-trapezoid-pentagon}

In this section, we define two sets of polynomials that control the combinatorial nature of our game. They arise as generating polynomials of the sequences of states of fixed length. 

Recall that $\Dyck_{a,2b}$ is the set of binary strings of length $2b+a$ such that the number of ones equals the number of zeroes plus $a$, and such that for every prefix, the number of ones minus the number of zeroes is at most $a$. Now, define $\sigma_{a,b}:=|\Dyck_{a,2b}|$. Interpreting each 0 as a unitary $\uparrow$ step in $\ZZ^2$ and each 1 as a unitary $\rightarrow$ step, we have that $\sigma_{a,b}$ is the total number of $\{\uparrow,\rightarrow\}$ paths from $(0,0)$ to $(a,a+b)$ that lie within the trapezoid ${(0,0),(a,0),(a,a+b),(0,b)}$. We enumerate them in proposition \ref{enum-sigma}.

Also, from section \ref{sec-evaluation-G}, the set $\Pent_{a,b}$ consists of all binary strings with $b$ zeroes such that in every prefix, the number of ones minus the number of zeroes does not exceed $a$. Also, define $\Pent_{a,b}^r$ the number of such strings which have exactly $r$ ones, and define $\theta_{a,b}^r:= |\Pent_{a,b}^r|$. As before, note that $\theta_{a,b}^r$ is the total number of $\{\uparrow,\rightarrow\}$ paths in $\ZZ^2$ from $(0,0)$ to $(r,b)$ that lie within the pentagon $\{(0,0),(0,a),(r,r-a), (r,b), (0,b)\}$ (this is a rectangle if $r<a$). We develop a linear recurrence for $\theta_{a,b}^r$ that solves in terms of $\sigma_{a,b}$ in proposition \ref{enum-theta}. 

\begin{mydef}
For positive integers $n,m$, let
$$
\left\{
\begin{array}{lll}
T_{n,m}(x) &=& \displaystyle \sum_{i = 0}^{m} \sigma_{n,i}\cdot x^i\\
P_{n,m}(x) &=& \displaystyle \sum_{i = 0}^{m-1} \theta_{n-1,m-n+1}^i\cdot x^i
\end{array}
\right.
$$
\end{mydef}

This polynomials arise when evaluating utility for states with two branches. These states can be enumerated by length of the fork, and sums like the following arise (recall that $\Dyck_{n,2m}$ are the Dyck draws with disadvantage $n+2m$):
$$
\sum_{d\in \Dyck_{n}} x^{(|d|-n)/2} = \sum_{i=0}^m \sum_{d\in \Dyck_{n,2i}} x^{(|d|-n)/2} = \sum_{i=0}^m x^i\cdot  \bigg(\sum_{d\in\Dyck_{n,2i}} 1\bigg) = \sum_{i=0}^m x^i\cdot |\Dyck_{n,2i}| = T_{n,m}(x).
$$

For the sake of completeness, we compute closed forms for $\sigma_{a,b}$ and $\theta_{a,b}^r$ in the following section.



 \subsection{Trapezoidal and Pentagonal Dyck paths}
 \label{appendix-trapezoid}
 
 \subsubsection{Counting paths in a trapezoid.} For nonnegative integers $a,b$, let $\mathcal L_{a,b}$ be the trapezoid in $\ZZ^2$ whose vertices are
 $\{(0,0),(a,0),(a+b,b),(0,b)\}.$ Also, let $\mathcal{T}_{a,b}$ be set of one-step north-east paths from $(0,0)$ to $(a+b,b)$ that stay inside $\mathcal L_{a,b}$, and  $\sigma_{a,b}$ the amount of such paths. Finally, let $C_n$ denote the $n$-th Catalan number. 
 \begin{myprop}
    \label{sigma-recurrence}
    The sequence $\sigma:\NN^2\to \NN$ verifies the following recurrence:
    $$\begin{cases}
    \sigma_{a,b}=\sum_{i=0}^b\sigma_{a-1,i}\cdot C_{b-i}& \mbox{ for }a\geq 1,b\in \NN,\\
    \sigma_{x,0}=1 & \mbox{ for }x\in \NN,\\
    \sigma_{0,y}=C_y & \mbox{ for }y\in \NN.
    \end{cases}$$
 \end{myprop}
 \begin{proof}
    To prove this, we write $\mathcal T_{a,b}$ as a union of disjoint sets. As a first remark, note that every path in $\mathcal T_{a,b}$ touches the line $l=\overline{(a,0)(a+b,b)}$ at least once. For $i\in \{0,\dots,b\}$, let $P_i$ be the point $(a+i,i)\in l$ and $\mathcal T_{a,b}^{(i)}$ all paths in $\mathcal T_{a,b}$ that touch the line $l$ for the first time at $P_i$. We have the disjoint union
    $$\mathcal T_{a,b}=\bigcup_{i=0}^b \mathcal T_{a,b}^{(i)}.$$
    Also, note that a path in $\mathcal T_{a,b}$ touches $l$ for the first time in $P_i$ if and only if it passed through the point $P_i-(1,0)$ without exiting $\mathcal T_{a-1,i}$, followed by an east step and any path from $P_i$ to $(a+b,b)$. This yields
    \begin{eqnarray*}
        \sigma_{a,b}=\sum_{i=0}^{b}\mbox{(number of paths from $(0,0)$ to $P_i-(0,1))$}\cdot \mbox{(number of paths from $P_i$ to $(a+b,b))$}
    \end{eqnarray*}
    Note that the amount of paths from $P_i$ to $(a+b,b)$ is $C_{b-i}$, since both points belong to $l$, proving that $\sigma_{a,b}$ verifies the recurrence equation. The border cases $\sigma_{0,\cdot},\sigma_{\cdot,0}$ are straightforward to prove. 
 \end{proof}
 
 Now let us define the sequence of generating functions with respect to the second variable of $\sigma_{\cdot,\cdot}$ as follows:
 $$\begin{array}{cl}
 \phi_a:&\RR\to \RR\\
 &x\mapsto \displaystyle \sum_{j=0}^{\infty}\sigma_{a,j}\cdot x^j
 \end{array}$$
 
 \begin{myprop}
    For $x\in [0,1/4]$ and $a\in \NN$ 
    $$\phi_a(x)=c(x)^{a+1},$$
    where $c(x):=\frac{1-\sqrt{1-4x}}{2x}$ is the generating function of the Catalan numbers.
 \end{myprop}
 \begin{proof}
    We have $\sigma_{a,\cdot}=\sigma_{a-1,\cdot}\star C_\cdot$, where $\star$ is the convolution operator, therefore $\phi_a(x)=\phi_{a-1}(x)\cdot c(x)$. The result follows noting that $\phi_0(x)=c(x)$.
 \end{proof}
 
Extracting the sequence from the Taylor series of $\phi_a(x)$ around 0 and proving by induction gives
 \begin{myprop}
    \label{enum-sigma}
    For $(a,b)\in\NN^2$,
    $$\sigma_{a,b}=\frac{(a+b)\cdot(a+2b)!}{b!\cdot(a+b+1)!}=\frac{a+1}{a+b+1}\cdot {a+2b\choose a+b}.$$
 \end{myprop}


\begin{subsubsection}{Counting paths in a pentagon.} Let $\mathcal{P}_{a,b}^r$ be the set of north-east unitary-step paths from $(0,0)$ to $(r,b)$ that do not exit the pentagon $(0,0),(a,0),(r,r-a),(r,b),(0,b)$, and $\theta_{a,b}^r$ the amount of such paths. Also, let $\lambda$ be the diagonal line $\overline{(a,0)(r,r-a)}$. 
\begin{myprop}
    The sequence $\theta:\NN^3\to \NN$ verifies the following recurrence relation
    $$
    \displaystyle \theta_{a,b}^r = \theta_{a-1,b}^r + \sum_{i=0}^{r-a}\sigma_{a-1,i}\cdot \sigma_{a+b-r,r-a-i}.
    $$
 \end{myprop}
 \begin{proof}
 As in the proof of \ref{sigma-recurrence}, we write $\mathcal{P}_{a,b}^r$ as a disjoint union:

 \begin{align*}
 \mathcal{P}_{a,b}^r &= (\mbox{paths that do not touch }\lambda)\cup (\mbox{paths that touch }\lambda)\\
                     &= (\mbox{paths that do not touch }\lambda)\cup \bigcup_{y\in \lambda} (\mbox{paths that touch }\lambda\mbox{ for the first time at $y\in \lambda$})\\
                     &= (\mbox{paths that do not touch }\lambda)\cup \bigcup_{y\in \lambda} (\mbox{paths from }(0,0)\mbox{ to }y-(1,0))\cdot (1,0)\cdot (\mbox{paths from }y\mbox{ to }(r,b))\\
                     &= (\mbox{paths that do not touch }\lambda)\cup \bigcup_{i = 0}^{r-a} (\mbox{paths from }(0,0)\mbox{ to }(a+i,i)-(1,0))\cdot (1,0)\cdot (\mbox{paths from }(a+i,i)\mbox{ to }(r,b)),
 \end{align*}
 where in the last equation, the symbol $\cdot$ stands for concatenation of paths. The recurrence follows noting that the paths that do not touch $\lambda$ are exactly $\mathcal{P}_{a-1,b}^r$, the paths from $(0,0)$ to $(a+i-1,i)$ are $\mathcal{T}_{a-1,i}$ and the paths from $(a+i,i)$ to $(r,b)$ are exactly $\mathcal{T}_{a+b-r,r-a-i}$ \end{proof}


 \begin{myprop}
 \label{enum-theta}
 For positive integers $a,b,r$ it holds that
 $$
 \begin{cases}
 \displaystyle\mbox{If }r\leq a,\quad \theta_{a,b}^r = {r+b\choose b},\\
 \displaystyle\mbox{If }a<r,\quad \theta_{a,b}^r = \sigma_{|b-r|,\min(r,b)} + \sum_{i=\max(1,r-b+1)}^a  \sum_{j=0}^{r-i}\sigma_{i-1,j}\cdot \sigma_{i+b-r,r-i-j}.\\
 %\displaystyle\mbox{If }a<r<b,\quad \theta_{a,b}^r = \sigma_{b-r,r} + \sum_{i=1}^a  \sum_{j=0}^{r-i}\sigma_{i-1,j}\cdot \sigma_{i+b-r,r-i-j},\\
 %\displaystyle\mbox{If }b<r\mbox{ and }a<r,\quad \theta_{a,b}^r = \sigma_{r-b,b} + \sum_{i=r-b+1}^a  \sum_{j=0}^{r-i}\sigma_{i-1,j}\cdot \sigma_{i+b-r,r-i-j},\\
 \end{cases}
 $$
 \end{myprop}
 \begin{proof}
    If $r\leq a$, the polygon is simply a rectangle from $(0,0)$ to $(r,b)$. If $a<r\leq b$, then the result yields adding the recurrence relation from the base case $a=0$ and noting that $\theta_{0,b}^r = \sigma_{b-r,r}$. If $b<r$, then the base case is $a=r-b$ and $\theta_{r-b,b}^r = \sigma_{r-b,b}$.
 \end{proof}
\end{subsubsection}
