\section{Concluding remarks}
\label{sec-con-r}

Our model of mining via a stochastic game allows for an intuitive representation of miners' actions as strategies, 
and gives us a way of understanding the rational behaviour of miners looking to accumulate cryptocurrency wealth. As it is the first model to provide payoff to miners for every branching strategy we can validate the commonly accepted assumption that long fork are not a viable strategy.
In this respect, we would like to identify strategies that are a Nash equilibrium for the case of decreasing rewards. However, this has proven 
to be a difficult task. In particular, one can show that the default strategy can never be part of such an 
equilibrium, no matter how small the hash power is for one of the players, if the strategy of another player involves forks of any length. 
This means that one must look for much more complex strategies to find such an equilibrium. 

One of the advantages of our model is its generality: it can be adapted to specify more complex 
actions, study other forms of reward and include cooperation between miners. For example, 
we are currently looking at strategies that involve withholding 
a mined block to the rest of the network, for which we need a slight extension of the notions of action and state. 
It would be very interesting how this model and previous work combine into a model where miner's behavior can deviate depending on both their short-and long-term 
goals. We would also like to study incentives under different models of cooperation between miners, and 
also other forms of equilibria in a dynamic setting. 

