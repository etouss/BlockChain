%!TEX root = main.tex

\section{Introduction}

The Bitcoin Protocol \cite{Bitcoin,DBLP:books/daglib/0040621,NC17}, or Nakamoto Protocol, introduces a novel decentralized network-consensus mechanism that is trustless and open for anyone connected to the Internet. To support such an open and dynamic topology, the protocol requires an underlying currency (a so-called \emph{cryptocurrency} \cite{NC17}) to encourage/discourage participants to/from taking certain actions. The largest network running this protocol at the time of writing is the Bitcoin network, and its underlying cryptocurrency is Bitcoin (BTC). Following the success of Bitcoin, several other cryptocurrencies have been created. Some of them are simple replicas of Bitcoin with slight modifications of the protocol parameters (e.g. Litecoin~\cite{Litecoin} or Bitcoin Cash~\cite{Bcash}), while some of them introduce interesting new modifications on top of the protocol to provide further functionalities (e.g. Ethereum~\cite{Ethereum,E17} or Monero~\cite{Monero}).

Due to the success and popularity of cryptocurrencies, there has been a growing body of research on the topic~\cite{mininggames:2016,optimalselfishmining2017,instabilitywithoutreward:2016,selfishmining2014,stop_selfish_mining2014,eclipseattacks2015,LBSZR15,LJG15,stubborn_mining:2016,economics_of_mining2013,ZGR17,ABLZ17,MHG18,SZWTK18}, giving us better clarity of their underlying protocols. However, due to the fact that the multitude of actors and incentives involved in cryptocurrencies make them rather hard to formalize and study rigorously, we are still lacking the complete understanding of how they work. 

We focus on 






Therefore, our goal in this paper is to construct a realistic model of a cryptocurrency protocol, and study its dynamics. In doing so, we focus on the action of cryptocurrency \emph{mining}, as this will allow us to measure the impact of incentives on the actors participating in the protocol. We model cryptocurrency mining as an infinite stochastic game, and we study what are the best mining strategies based on different choices for mining rewards. The model itself tries to stay as close as possible to reality, and consider 
%details 
fundamental properties such as the deflationary character of mining rewards in cryptocurrencies, and the economical discounts that different strategies incur for monetary gains that happen far in the future.
%, and the fact that cryptocurrency mining is a discrete process. 
%In order 
To provide context for our contributions, we first briefly explain how the Bitcoin protocol~works.


%	IMPORTANT INCLUDE IN CONTRIBUTIONS
%In contrast to other previous work, our model is general enough to consider any possible mining strategy, and we study crypto-mining assuming a closed world with no option to ``cash out" and leave the market, as we believe that this scenario gives us the cleanest possible model. 


%	IMPORTANT INCLUDE IN RELATED WORK
%A good body of research pursuing this objective has been presented recently~\cite{mininggames:2016,optimalselfishmining2017,instabilitywithoutreward:2016,selfishmining2014,stop_selfish_mining2014,eclipseattacks2015,LBSZR15,LJG15,stubborn_mining:2016,economics_of_mining2013,ZGR17,ABLZ17,MHG18,SZWTK18}, yet some concession and simplification regarding the model had to be made. 



\smallskip
\noindent
{\bf The Bitcoin protocol.} The objective of the Bitcoin protocol is to generate consensus on a data structure that is replicated amongst all nodes in a trustless 
and decentralized peer-to-peer network. The data structure used in the protocol is an append-only record of transactions. New transactions are spread across the network using peer-to-peer communication until one node is \emph{allowed} (we will explain this in detail later) to assemble a group of transactions into a new \emph{block}, and present this block as a candidate to extend the data structure. Apart from the set of transactions, the block will contain a pointer to some previous block. The newly formed block is then spread throughout the network, and the whole process is repeated. Since every block points to a previous block, a tree of blocks is naturally formed. The consensus data structure is generally defined as the longest branch of such a tree, also known as the \emph{blockchain}.

%To support the dynamics described above, the protocol requires an underlying currency to encourage actors in the network to take certain actions. The most important action is the generation of new blocks: whenever a participant forms a new block, she receives a certain amount of currency. This is known as the block reward, and is the way in which new currency is created. In Bitcoin, this amount was originally 50BTC and halves approximately every four years (the current reward is 12.5BTC). The currency generated in a block is considered valid only if the block belongs to the blockchain. This is important, because block rewards may cease to be valid currency if the block is no longer part of the blockchain, even if the block was part of it when it was created. Therefore, whenever a node forms a new block, it is encouraged to place this block in a part of the tree with a high probability of becoming part of the blockchain. Actually, the protocol states that new blocks should always be appended on top of a block with maximal distance to the root of the tree, although participants are not obliged to follow this rule.

To provide an incentive for the nodes in the network to keep  generating new blocks (and to check that transactions they include in a block are correct according to the protocol rules), the protocol requires an underlying currency. This currency is then used to reward generation of new blocks: whenever a participant forms a new block, she receives a certain amount of currency. This is known as the block reward, and is the way in which new currency is created. In Bitcoin, this amount was originally 50BTC and halves approximately every four years (the current reward is 12.5BTC). The currency generated in a block is considered valid only if the block belongs to the blockchain. This is important, because block rewards may cease to be valid currency if the block is no longer part of the blockchain, even if the block was part of it when it was created. Therefore, whenever a node forms a new block, it is encouraged to place this block in a part of the tree with a high probability of becoming part of the blockchain. Actually, the protocol states that new blocks should always be appended on top of a block with maximal distance to the root of the tree, although participants are not obliged to follow this rule.

Since blocks give a reward, nodes will naturally want to generate blocks\footnote{In order to encourage including all the transactions that are correct according to the cryptocurrency rules, the miners also receive a small {\em transaction fee} for each transaction they include in a block. As in currently used cryptocurrencies fees rarely exceed 10\% of the mining reward \cite{TotalMiningRevenue,TotalMiningFees}, we focus on block rewards when gauging the miner's economic motivation for participating in the protocol.}. If we expect the currency to have any value, generating new blocks must then be hard. Under the proof-of-work framework, participants generating new blocks are required to solve some computationally hard problem per each new block. In Bitcoin, a block is \emph{valid} in the protocol if its hash value, when interpreted as a number, is less than a certain threshold. Since hash functions are unpredictable, the only way to generate a valid block is to try with several different blocks, until one of them has a hash value below the established threshold. This is known as \emph{mining}, and the number of (valid and invalid) blocks per second that a miner can hash is referred to as her \emph{hash power}. Nodes who participate in the generation of blocks are called \emph{miners}. 

Mining rewards introduce a competition for generating new blocks, and to ensure that one's blocks form part of the blockchain. This can be naturally viewed as a non-cooperative game in which the miners take certain actions in order to maximize their benefit, thus defining different strategies that describe their behaviour. A natural question
that a miner would ask in this scenario is what is her optimal strategy, given that she controls some fixed amount of hash power. On the other hand, the protocol designers might wonder how to set the parameters of the protocol in such a way that the default behaviour defines a Nash equilibrium (i.e. a set of strategies where no miner has an incentive to perform a different action). Studying these types of questions is the main objective of this paper.

\smallskip
\noindent{\bf Contributions.} The first contribution of our work is a realistic model of cryptocurrency mining. We model mining as an infinite stochastic game in which miners are expected to maximize their long-term utility. One of the benefits of our approach is that using  a few basic design parameters we can in fact accommodate different cryptocurrencies, and not focus solely on e.g. Bitcoin. These parameters also allow us to account for fundamental factors such as deflation, or discount in the block reward. Another benefit of our proposal is the fact that we can reason about all possible mining strategies, without the need to focus on a specific subclass.

The second contribution of our work is a set of results about optimal behaviour of miners in different cryptocurrencies. In particular, we study mining under the assumption that block rewards are constant (as it will eventually be in cryptocurrencies with tail-emission such as Monero or Ethereum), or when they decrease over time (e.g. Bitcoin). In the first scenario, we show that the default scenario of always mining on the latest block of the blockchain is indeed a Nash equilibrium. However, this is not the case for the second scenario, and in fact we 
prove that strategies that involve mining on a block that is not at the tail of the blockchain (known as \emph{forking}) can indeed give higher utility in some cases. Thus we study 
what is the best strategy for miners when assuming everyone else is playing the default strategy. The choice of strategy depends on the hash power, the rate at which 
block rewards decrease over time, and the usual financial discount rate. We show that our model confirms the commonly held belief that players should start deviating from the default strategy 
when they approach 50\% of the network's hash power (also known as 51\% attack). However, in this case the strategy may not be as simple as ``always try to fork'', but rather be a combination 
of appropriately choosing when to fork, and when to keep on extending the current blockchain. 


%In this paper we present a framework to represent the miners' incomes in a blockchain protocol and study is realisticness. Then we study the incentives of the miners under two instances of the framework. When we assume that the block creation fee is constant, we prove that the default behaviour is not only a Nash equilibrium, but the one which maximises the income of every miners. Under the assumption that the block creation fee decrease overtime, we computed the utility of a miner for several strategies \etienne{A word on the fact that we have close form for the utility ? Is it useful ? }. We considered a set-up where all the player expect one is playing default, and prove that there exists a thresholds of the hash power ($\leq 0.x$ for Bitcoin \etienne{value when we assume alpha and beta for bitcoin, note that we dont have the actual threshold just an upper bound !}) above which default is not an optimal strategy.


\smallskip
\noindent
{\bf Related Work.} There are a number of studies that approach mining from a game-theoretical point of view \cite{economics_of_mining2013,selfishmining2014,optimalselfishmining2017,biais2018blockchain,mininggames:2016,instabilitywithoutreward:2016}. The main difference with our work is in the choice of the reward function \cite{economics_of_mining2013,selfishmining2014,optimalselfishmining2017,biais2018blockchain,instabilitywithoutreward:2016} these papers use, or restricting the space of viable mining strategies \cite{economics_of_mining2013,selfishmining2014,optimalselfishmining2017}. Perhaps the study which is closest in spirit to ours is the one by Kiayias et al. in \cite{mininggames:2016}. 
Here the authors also focus on the hash power thresholds for which the default strategy is not an optimal strategy any more. Albeit the authors consider a range of strategies, their model is still somewhat more restrictive than ours. Overall, the underlying characteristic of our model is that the block emission cadence is discrete and not continuous, that the reward function can be fine tuned depending on the discounts and deflationary parameters, and that we do not restrict the space of strategies that the miners are allowed to use.

There are also recent work regarding miners' strategies in multi-cryptocurrency markets \cite{dhamal2018stochastic,spiegelman2018game}. The main difference with our work is that we focus on a single cryptocurrency in a closed world setting (e.g. we do not reason about  the exchange rate of the studied cryptocurrency with the US dollar or another cryptocurrency). Finally, there are a number of papers on network properties of the Bitcoin protocol, as well as technical considerations regarding its security and privacy (see e.g. the survey by Conti et al. \cite{conti2018survey}). One interesting result here is that the network's specificity of the protocol could give participants an incentive to deviate from default behaviour~\cite{bitcoin_attacks_2013,ddos_attacks2014,empirical_dos_attacks2014}.  


%More specifically,  Kroll et al. \cite{economics_of_mining2013} focus on a subset of strategies they call \emph{monotonic}, and prove a 
%Nash equilibria for these strategies when assuming a constant payoff model. Eyal and Sirer~\cite{selfishmining2014}  and later Sapirshtein et al. in \cite{optimalselfishmining2017} studied a different strategy known as \emph{selfish mining}, in which miners may withhold some of their newly created blocks. Their main result is that, assuming that all other miners are following the default strategy and that block rewards remain constant, a miner with strictly less than 50\% of the network's hash power can increase his income by not always revealing block immediately (thus proving that the default strategy is not a Nash equilibrium). Carlsten et al. \cite{instabilitywithoutreward:2016} studied the \emph{tail} behavior of Bitcoin in which the block reward becomes negligible compared to the mining fees, proving that in such situation miners have further incentives to deviate from the default strategy. 
%%Those works differs from our's in that they consider a model with constant block's reward \cite{selfishmining2014,optimalselfishmining2017} or really volatile block's reward \cite{instabilitywithoutreward:2016}. 
%A formalisation considering a utility where miners must define a fixed cash-in window as they start the game has been studied in \cite{biais2018blockchain}, and the authors proved the existence of a non-default nash equilibrium in this set-up.
%Perhaps the study which is closest in spirit to ours is the one by Kiayias et al. in \cite{mininggames:2016}. %later on extended in \cite{biais2018blockchain}. 
%Here the authors also focus on the hash power thresholds for which the default strategy is not an optimal strategy any more. Albeit the authors consider a range of strategies, their model is still somewhat more restrictive than ours. %not able to consider the same as our. %\francisco{I do not understand this last phrase}

\smallskip
\noindent
{\bf Organization of the paper.} We present our game-theoretic formalization of cryptocurrency mining in Section \ref{sec-formalization}. Our results for constant and decreasing rewards are presented in Sections \ref{sec-const_rew} and~\ref{sec-dec}, respectively. Finally, some concluding remarks are given in Section~\ref{sec-con-r}. We provide the main details of the proofs of the results in the paper; however, due to the lack of space, complete proofs of all results are provided in the full version of this paper, which can be found at \url{https://anonymous.4open.science/repository/08eff11e-78b1-4836-837a-cff08348a8c8}. Besides, independent Python and C++ scripts used in the paper for the utility evaluation and plots generation can also be found in this repository.




