%!TEX root = main.tex

An attempt to give the most abstract possible definition of a blockchain.

\section{BlockChain}

\subsection{Lists and their validation}

Given a set $S$, let $\set(S)$ be the set of all sets of elements of $S$, and $\flist(S)$ be the set of all finite lists of elements of $S$. Given $L \in \flist(S)$, we use notation $\length(L)$ to refer to the number of elements in $L$, notation $L[i]$ to refer to the $i$-th element in $L$, where $i \in \{1, \ldots, \length(L)\}$, and notation $L[i,j]$ to refer to the sublist $[L[i], \ldots, L[j]]$ of $L$, where $i,j \in \{1, \ldots, \length(L)\}$ and $i \leq j$. Notice that $\length(L) = 0$ if and only if $L$ is the empty list $[\ ]$. 

From now on, assume that $\Sigma$ is a finite alphabet, and that $\B \subseteq \Sigma^*$ is the set of all possible blocks.  
%MARCELO: Do we really need the following?
%Moreover we extend the definition of $\subseteq$ such that :
%\begin{eqnarray*}
%	\forall S \in \fset(B), \forall L \in \flist(B), L \subseteq S \Leftrightarrow \forall i \in \{1, \ldots, |L|\}, L[i] \in S
%\end{eqnarray*}

\begin{mydef}
A validation rule is a function $V : \flist(\B) \to \set(\B)$
\end{mydef}
Intuitively, $V$ is a function taking a finite list $L$ of blocks as input, and returning the set of blocks that could be added to $L$ to produce a valid blockchain.

A list $G \in \flist(\B)$  is said to be a genesis list of $V$ if $\length(G) \geq 1$, $G[1] \in V([\ ])$ and $G[i+1] \in V(G[1,i])$, for every $i \in \{1, \ldots, \length(G)-1\}$. That is, $G$ is a genesis list if $G$ is a non-empty valid blockchain. 
\begin{mydef}
Let $V$ be a validation rule and $G$ be a genesis list. Then a list $L \in \flist(\B)$ is valid with respect to $(G,V)$ if:
\begin{enumerate}
\item $\length(G) \leq \length(L)$ and $G = L[1,\length(G)]$.

\item $L[i+1] \in V(L[1,i])$, for every $i \in \{\length(G), \ldots, \length(L)-1\}$.
\end{enumerate}
\end{mydef}
The role of $G$ in this definition is to provide the blocks to startup the system. Let $\LOG(G,V)$ be the set of valid lists with respect to $(G,V)$.

%\begin{myrem}
%	We introduce $LOG_{G,V}$ which really complicated but is actually necessary to deal with fork and consensus later.
%\end{myrem}
%
%\begin{mydef}
%	A set of validated chain $LOG_{G,V}$ is said to be infinite if:
%	$$\forall log_{G,V} \in LOG_{G,V}, \forall i \in \mathbb{N} , \forall b  \in V(log_{G,V}(i)-), V(log_{G,V}(i)-,b)\neq \emptyset $$
%\end{mydef}
%
%\begin{myrem}
%	Infinite here is used in a sense that whatever instance of a $G,V$ validated-chain we are dealing with we will always be able to complete it.
%\end{myrem}

Two lists $L_1, L_2 \in \flist(\B)$ are said to disagree in the last element if one of the following conditions holds: (1) $\length(L_1) = 0$ and $\length(L_2) > 0$, (2) $\length(L_1) > 0$ and $\length(L_2) = 0$, or (3) $\length(L_1) > 0$, $\length(L_2) > 0$ and $L_1[\length(L_1)] \neq L_2[\length(L_2)]$.
\begin{mydef}
Let $V$ be a validation rule and $G$ be a genesis list of $V$. Then $\LOG(G,V)$ is safe if for every pair $L_1, L_2 \in \flist(\B)$ that disagree in the last element, it holds that $V(L_1) \cap V(L_2) = \emptyset$.
\end{mydef}


\subsection{Knowledge}

Given a  a validation rule $V$, a genesis list $G$ of $V$ and a subset $K$ of $\LOG(G,V)$, there is a natural way to  visualize  $K$ as a graph $\G(K)$. The set of nodes of $\G(K)$ is the set of blocks occurring in the lists in $K$, and there is an edge from a block $b_1$ to a block $b_2$ if there exists a list $L \in K$ such that $b_1 = L[i]$ and $b_2 = L[i+1]$, where $i \in \{1, \ldots, \length(L) -1\}$.
\begin{mylem}
Assume that $\LOG(G,V)$ is safe. The for every subset $K$ of $\LOG(G,V)$, it holds that $\G(K)$ is a tree rooted at $G[1]$.
\end{mylem}

\juan{Related to the next comment, why do we need $\K(G,V)$? this is just all non-empty finite subset $K$ of $\LOG(G,V)$}\\ 

Thus, assuming that $\LOG(G,V)$ is safe, from now on we refer to every non-empty finite subset $K$ of $\LOG(G,V)$ as a {\em knowledge tree} of $(G,V)$. Moreover, we define $\K(G,V)$ as the set of all knowledge trees of $(G,V)$.

%A \emph{knowledge tree} $K$ is a tree $K = (N,E)$ with $N \subseteq \B$ and such that 
%every path in $K$ from its root to a leaf belongs to $LOG_{G,V}$.
%Let $\mathcal{K}$ be the set of knowledge tree with respect to $(G,V)$.

%Intuitively, the knowledge tree represents all the blockchain information we know. 
%Abusing notation, we say that a block $B$ is in a knowledge tree $K = (N,E)$ if $B \in N$. 
%(this is informal) We use $\paths(K)$ to denote the set of all lists of blocks made out of a path 
%in $K$ from its root to a leaf. 

\subsection{Block chain and protocols}

\begin{mydef}
A relation $\preceq$ on $\LOG(G,V)$ is said to be a knowledge order over $(G,V)$ if $\preceq$ is a total preorder on $\LOG(G,V)$, that is, $\preceq$ is reflexive, transitive and total.

Moreover, $P$ is said to be a blockchain protocol over $(G,V)$ if $P$ is a sequence $\{ \preceq_i\}_{i \in \mathbb{N}}$ such that each $\preceq_i$ $(i \in \mathbb{N})$ is a knowledge order over $(G,V)$.
\end{mydef}

%\begin{mydef}
%	Let $\preceq_{G,V,t}$ be  a total preorder over $LOG_{G,V}$:
%\begin{eqnarray*}		
%	&\forall L_1, L_2, L_3 \in LOG_{G,V}, L_1 \preceq_{G,V,t} L_2 \land L_2 \preceq_{G,V,t} L_3 \implies L_1 \preceq_{G,V,t} L_3  \\
%	&\forall L_1, L_2 \in LOG_{G,V}, L_1 \preceq_{G,V,t} L_2 \lor L_2 \preceq_{G,V,t} L_1 
%\end{eqnarray*}
%	A block chain protocol over $LOG_{G,V}$ is a function noted $\preceq_{G,V}$ such that: $$ \forall t \in \mathbb{N}, \preceq_{G,V}(t) =  \preceq_{G,V,t}$$ where $\preceq_{G,V,t}$ is a total preorder over $LOG_{G,V}$
%\end{mydef}


\marcelo{Do we really need to use the notion of knowledge tree in the following definitions? If we say that $K$ is a knowledge tree then we need to use $\paths(K)$ to refer to the paths in $K$. On the other hand, if we directly say that $K \in \LOG(G,V)$, we can just refer to the elements of $K$ (we do not a special notation for paths). Knowledge trees are a nice way to visualize the blocks of set $K \in \LOG(G,V)$, but I think we should just mention them because of this (and we should not use them in the definitions). What do you think?}

\juan{You mean $K \subset \LOG(G,V)$ right? I agree that here we do not need safeness nor finitiness to make this work, so we could just stick with $\LOG(G,V)$ instead of all knowledge trees}

\begin{mydef}
Let $P = \{ \preceq_i\}_{i \in \mathbb{N}}$ be a blockchain protocol over $(G,V)$, $K \subseteq \LOG(G,V)$ and $t \in \mathbb{N}$. 
Then a maximal element of $K$ with respect to $\preceq_t$ is said to be a blockchain of $K$ at time $t$ with respect to the protocol $P$.
\end{mydef}


\marcelo{I stopped here. I am not totally convinced that we need to introduce the following notion of equivalence, this is something that we need to discuss.}

\begin{mydef}
	Let $K = (N,E)$ and $K'=(N',E')$ two knowledge trees we say that $K$ and $K'$ are $\equiv$ equivalent if :
	\begin{eqnarray*}
		& \equiv \textit{ is an equivalent function} \\
		& \forall L \in \paths(K), \exists L' \in \paths(K'), \forall i \in \llbracket 1,|L| \rrbracket, L[i] \equiv L'[i] \\
		& \forall L' \in \paths(K'), \exists L \in \paths(K), \forall i \in \llbracket 1,|L'| \rrbracket, L[i] \equiv L'[i] \\
	\end{eqnarray*}	
	We denote $K^{\equiv}$ the set of knowledge equivalent to $K$. 
\end{mydef}


\medskip
\noindent
\textbf{Action, states, reward and game}
\begin{mydef}
	Considering a set of player $P$ we denote $\mathcal{K}_P$ the set of function $K_P : P \rightarrow \mathcal{K}$ mapping a knowledge tree to each player. 
	We denote $\mathcal{K}^{\equiv}_P$ the set of mapping where:
	$$\forall K_P, K'_P \in \mathcal{K}^{\equiv}_P, \forall p \in P, K_P(p) \textit{ and } K'_P(p) \textit{ are } \equiv \textit{ equivalent } $$ 
\end{mydef}

Intuitively $\mathcal{K}_P$ represents the true knowledge of each player.

\begin{mydef}
	We call action a for player $w\in P$ function $a^\equiv_w: \mathcal{K}_P \rightarrow \mathcal{K}_P^{\equiv}$ such that: 
	\begin{eqnarray*}	
		&\forall K_P \in \mathcal{K}_P, \forall K'_P \in a_w(K_P),  \forall u \in P, K_P(u) \subseteq K'_P(u)  \\
		&\forall K_P \in \mathcal{K}_P, \forall K'_P \in a_w(K_P), \forall u \in P, K'_P(u) \subseteq K'_P(w) \cup K_P(u)  \\
	\end{eqnarray*}
	Let $A^\equiv_w$ be the set of action for player $w$.
\end{mydef}
An action of player $w$ is represented by a modification of $w$ knowledge and a round of communication. We are dealing with equivalence knowledge in order to reduce the number of action possible.

\begin{mydef}
	Let $P$ be a set of player, $\mathcal{A}^\equiv = A_{p_1}^\equiv \times A_{p_1}^\equiv \ldots \times A_{p_{|P|}}^\equiv$
\end{mydef}

\begin{mydef}
	We call reward for player $w$ a function $r_w : \mathcal{K}_P \times \mathcal{A}^\equiv \rightarrow \mathbb{R}^+$ 
	such that: $$ \forall A \in \mathcal{A}^\equiv, K_P \textit{ equivalent } K'_P \Rightarrow r_w(K_P,A) = r_w(K'_P,A) $$
\end{mydef}
Intuitively $\mathcal{K}_P$ represent the knowledge of each player assumed by $w$

\begin{mydef}
	Let $P$ be a set of player, $\mathcal{R} = r_{p_1} \times r_{p_1} \ldots \times r_{p_{|P|}}$
\end{mydef}
