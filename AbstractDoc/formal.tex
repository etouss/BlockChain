%!TEX root = main.tex

An attempt to give the most abstract possible definition of a blockchain.

\section{BlockChain}
\medskip
\noindent
\textbf{Chains as lists and their validation}

Given a set $S$, let $\flist(S)$ and $\fset(S)$ be the sets of all finite lists and all finite sets of elements of $S$, respectively. Given $L \in \flist(S)$, we use notation $|L|$ to refer to the number of elements in $L$, and notation $L[i]$ to refer to the $i$-th element in $L$, where $i \in \{1, \ldots, |L|\}$. From now on, assume that $\Sigma$ is a finite alphabet, and that $\B \subseteq \Sigma^*$ is the set of all possible blocks.  Moreover we extend the definition of $\subseteq$ such that :
\begin{eqnarray*}
	\forall S \in \fset(B), \forall L \in \flist(B), L \subseteq S \Leftrightarrow \forall i \in \{1, \ldots, |L|\}, L[i] \in S
\end{eqnarray*}

\begin{mydef}
A validation rule is a function $V : \flist(\B) \to \fset(\B)$
\end{mydef}
Intuitively $V$ is a function taking a list $L$ of block as input, and returning the set of blocks that could be added to $L$ to produce a valid blockchain.

\begin{mydef}
Let $G \in \flist(\B)$ be non-empty, and $V$ be a validation rule. Then a list $L \in \flist(\B)$ is a validated chain with respect to $(G,V)$ if:
\begin{enumerate}
\item $|G| \leq |L|$ and $L[i] = G[i]$, for every $i \in \{1, \ldots, |G|\}$.

\item $L[1] \in V([\ ])$ and $L[i+1] \in V([L[1], \ldots, L[i]])$, for every $i \in \{1, \ldots, |L|-1\}$.
\end{enumerate}
\end{mydef}
List $L$ in this definition is a valid chain according to the validation rule $V$ and the lists $G$ of genesis blocks (whose role is to provide the blocks to startup the system). Let $\LOG(G,V)$ be the set of validated chains with respect to $(G,V)$.

%\begin{myrem}
%	We introduce $LOG_{G,V}$ which really complicated but is actually necessary to deal with fork and consensus later.
%\end{myrem}
%
%\begin{mydef}
%	A set of validated chain $LOG_{G,V}$ is said to be infinite if:
%	$$\forall log_{G,V} \in LOG_{G,V}, \forall i \in \mathbb{N} , \forall b  \in V(log_{G,V}(i)-), V(log_{G,V}(i)-,b)\neq \emptyset $$
%\end{mydef}
%
%\begin{myrem}
%	Infinite here is used in a sense that whatever instance of a $G,V$ validated-chain we are dealing with we will always be able to complete it.
%\end{myrem}

\begin{mydef}
Let $G \in \flist(\B)$ be non-empty, and $V$ be a validation rule. Then $\LOG(G,V)$ is safe if for every $L \in \LOG(G,V)$ such that every $b_1, b_2 \in \B$ such that $b_1 \neq b_2$:
\begin{eqnarray*}
V([L[1], \ldots, L[|L|], b_1]) \cap V([L[1], \ldots, L[|L|], b_2]) & = & \emptyset
\end{eqnarray*}
\end{mydef}
Intuitively, in order to be secured $V$ should depend on the last block $b$ that is included in the blockchain.

\medskip
\noindent
\textbf{Knowledge}

\begin{mydef}
A \emph{knowledge tree} $K$ is a tree $K = (N,E)$ with $N \subseteq \B$ and such that 
every path in $K$ from its root to a leaf belongs to $LOG_{G,V}$.
Let $\mathcal{K}$ be the set of knowledge tree with respect to $(G,V)$.
\end{mydef}

Intuitively, the knowledge tree represents all the blockchain information we know. 
Abusing notation, we say that a block $B$ is in a knowledge tree $K = (N,E)$ if $B \in N$. 
(this is informal) We use $\paths(K)$ to denote the set of all lists of blocks made out of a path 
in $K$ from its root to a leaf. 


\medskip
\noindent
\textbf{Block chain, protocols}

\begin{mydef}
	Let $\preceq_{G,V,t}$ be  a total preorder over $LOG_{G,V}$:
\begin{eqnarray*}		
	&\forall L_1, L_2, L_3 \in LOG_{G,V}, L_1 \preceq_{G,V,t} L_2 \land L_2 \preceq_{G,V,t} L_3 \implies L_1 \preceq_{G,V,t} L_3  \\
	&\forall L_1, L_2 \in LOG_{G,V}, L_1 \preceq_{G,V,t} L_2 \lor L_2 \preceq_{G,V,t} L_1 
\end{eqnarray*}
	A block chain protocol over $LOG_{G,V}$ is a function noted $\preceq_{G,V}$ such that: $$ \forall t \in \mathbb{N}, \preceq_{G,V}(t) =  \preceq_{G,V,t}$$ where $\preceq_{G,V,t}$ is a total preorder over $LOG_{G,V}$
\end{mydef}

\begin{mydef}
Let $t \in \mathbb N$, $\preceq_{G,V}$ a block chain protocol and $K$ a knowledge tree. 
A block chain of $K$ with respect to $\preceq_{G,V}$ in $t$ is any minimal element in $\paths(K)$ with respect to $\preceq_{G,V}(t)$.
\end{mydef}

\begin{mydef}
	Let $K = (N,E)$ and $K'=(N',E')$ two knowledge trees we say that $K$ and $K'$ are $\equiv$ equivalent if :
	\begin{eqnarray*}
		& \equiv \textit{ is an equivalent function} \\
		& \forall L \in \paths(K), \exists L' \in \paths(K'), \forall i \in \llbracket 1,|L| \rrbracket, L[i] \equiv L'[i] \\
		& \forall L' \in \paths(K'), \exists L \in \paths(K), \forall i \in \llbracket 1,|L'| \rrbracket, L[i] \equiv L'[i] \\
	\end{eqnarray*}	
	We denote $K^{\equiv}$ the set of knowledge equivalent to $K$. 
\end{mydef}


\medskip
\noindent
\textbf{Action, states, reward and game}
\begin{mydef}
	Considering a set of player $P$ we denote $\mathcal{K}_P$ the set of function $K_P : P \rightarrow \mathcal{K}$ mapping a knowledge tree to each player. 
	We denote $\mathcal{K}^{\equiv}_P$ the set of mapping where:
	$$\forall K_P, K'_P \in \mathcal{K}^{\equiv}_P, \forall p \in P, K_P(p) \textit{ and } K'_P(p) \textit{ are } \equiv \textit{ equivalent } $$ 
\end{mydef}

Intuitively $\mathcal{K}_P$ represents the true knowledge of each player.

\begin{mydef}
	We call action a for player $w\in P$ function $a^\equiv_w: \mathcal{K}_P \rightarrow \mathcal{K}_P^{\equiv}$ such that: 
	\begin{eqnarray*}	
		&\forall K_P \in \mathcal{K}_P, \forall K'_P \in a_w(K_P),  \forall u \in P, K_P(u) \subseteq K'_P(u)  \\
		&\forall K_P \in \mathcal{K}_P, \forall K'_P \in a_w(K_P), \forall u \in P, K'_P(u) \subseteq K'_P(w) \cup K_P(u)  \\
	\end{eqnarray*}
	Let $A^\equiv_w$ be the set of action for player $w$.
\end{mydef}
An action of player $w$ is represented by a modification of $w$ knowledge and a round of communication. We are dealing with equivalence knowledge in order to reduce the number of action possible.

\begin{mydef}
	Let $P$ be a set of player, $\mathcal{A}^\equiv = A_{p_1}^\equiv \times A_{p_1}^\equiv \ldots \times A_{p_{|P|}}^\equiv$
\end{mydef}

\begin{mydef}
	We call reward for player $w$ a function $r_w : \mathcal{K}_P \times \mathcal{A}^\equiv \rightarrow \mathbb{R}^+$ 
	such that: $$ \forall A \in \mathcal{A}^\equiv, K_P \textit{ equivalent } K'_P \Rightarrow r_w(K_P,A) = r_w(K'_P,A) $$
\end{mydef}
Intuitively $\mathcal{K}_P$ represent the knowledge of each player assumed by $w$

\begin{mydef}
	Let $P$ be a set of player, $\mathcal{R} = r_{p_1} \times r_{p_1} \ldots \times r_{p_{|P|}}$
\end{mydef}



\begin{mydef}
	Considering a set of player $P$, a set of function $\mathcal{K}_P$, the set of action $\mathcal{A}$, the set of reward $\mathcal{R}$, and a function $\mathcal{P} : \mathcal{K}_P \times \mathcal{A} \times \mathcal{K}_P^\equiv \rightarrow [0;1]$ a transition probability ($\mathcal{P}(K_P,A,K_P^{1\equiv})$ is the probability of transitioning from $K_P$ to an element of $K_P^{1\equiv}$ after joint action $A$).
	We define a infinite stochastic game  such that:
	\begin{itemize}
		\item $P$ is the set of player.
		\item $\mathcal{A}$ is the set of available action.
		\item $\mathcal{K}_P$ is the set of states.
		\item $\mathcal{R}$ the set of pay-off function.
		\item $\mathcal{P}$ is the transition probability function.
	\end{itemize}
\end{mydef}

\begin{myrem}
	I don't like what i am doing with equivalence as it complicate stuff a lot. Moreover if i have to prove that my game is well defined as it is not immediate due to that. Finally we have an infinite number of states, equivalence states and finitely repeated game might be a solution.
\end{myrem}


\begin{mydef}
	We say that $K'_P$ is reasonable regarding $I$ and $K_P$ if exists an action $a \in A$ associate to a player $w \in P$ in an equilibrium profile such that: 	\begin{eqnarray*}	
		& K'_P(w) = a(K_P(w)) \\
	\end{eqnarray*}	
	$K'_P$ represent all the possible knowledge after one reasonable action (on belonging to a nash equilibrium) has happened with an optional comunication round from the winner ($K'_P(p) \subseteq K_P(p) \cup K'_P(w)$).  (We may want to include comunication is A instead of here as it's part of the strategie ... so will influence I).
\end{mydef}

\begin{mydef}
	We say that $K^n_p$ is $n$ reasonable regarding $K_p$ and $I$ if exists $(K^i_p), \forall i \leq n, K^i_p$ reasonable regarding $K^{i-1}_p$ and $I$ and $K^0_P = K_P$.   
\end{mydef}

Good to go we finally have a defintion of reasonable $K$ and can define blockchain property which should be verified over all reasonable $K$.
