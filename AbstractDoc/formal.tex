%!TEX root = main.tex

An attempt to give the most abstract possible definition of a blockchain.

\section{BlockChain}
\medskip
\noindent
\textbf{Chains as lists and their validation}

Given a set $S$, let $\flist(S)$ and $\fset(S)$ be the sets of all finite lists and all finite sets of elements of $S$, respectively. Given $L \in \flist(S)$, we use notation $|L|$ to refer to the number of elements in $L$, and notation $L[i]$ to refer to the $i$-th element in $L$, where $i \in \{1, \ldots, |L|\}$. From now on, assume that $\Sigma$ is a finite alphabet, and that $\B \subseteq \Sigma^*$ is the set of all possible blocks.  Moreover we extend the definition of $\subseteq$ such that :
\begin{eqnarray*}
	\forall S \in \fset(B), \forall L \in \flist(B), L \subseteq S \Leftrightarrow \forall i \in \{1, \ldots, |L|\}, L[i] \in S
\end{eqnarray*}

\begin{mydef}
A validation rule is a function $V : \flist(\B) \to \fset(\B)$
\end{mydef}
Intuitively $V$ is a function taking a list $L$ of block as input, and returning the set of blocks that could be added to $L$ to produce a valid blockchain.

\begin{mydef}
Let $G \in \flist(\B)$ be non-empty, and $V$ be a validation rule. Then a list $L \in \flist(\B)$ is a validated chain with respect to $(G,V)$ if:
\begin{enumerate}
\item $|G| \leq |L|$ and $L[i] = G[i]$, for every $i \in \{1, \ldots, |G|\}$.

\item $L[1] \in V([\ ])$ and $L[i+1] \in V([L[1], \ldots, L[i]])$, for every $i \in \{1, \ldots, |L|-1\}$.
\end{enumerate}
\end{mydef}
List $L$ in this definition is a valid chain according to the validation rule $V$ and the lists $G$ of genesis blocks (whose role is to provide the blocks to startup the system). Let $\LOG(G,V)$ be the set of validated chains with respect to $(G,V)$.

%\begin{myrem}
%	We introduce $LOG_{G,V}$ which really complicated but is actually necessary to deal with fork and consensus later.
%\end{myrem}
%
%\begin{mydef}
%	A set of validated chain $LOG_{G,V}$ is said to be infinite if:
%	$$\forall log_{G,V} \in LOG_{G,V}, \forall i \in \mathbb{N} , \forall b  \in V(log_{G,V}(i)-), V(log_{G,V}(i)-,b)\neq \emptyset $$
%\end{mydef}
%
%\begin{myrem}
%	Infinite here is used in a sense that whatever instance of a $G,V$ validated-chain we are dealing with we will always be able to complete it.
%\end{myrem}

\begin{mydef}
Let $G \in \flist(\B)$ be non-empty, and $V$ be a validation rule. Then $\LOG(G,V)$ is safe if for every $L \in \LOG(G,V)$ such that every $b_1, b_2 \in \B$ such that $b_1 \neq b_2$:
\begin{eqnarray*}
V([L[1], \ldots, L[|L|], b_1]) \cap V([L[1], \ldots, L[|L|], b_2]) & = & \emptyset
\end{eqnarray*}
\end{mydef}
Intuitively, in order to be secured $V$ should depend on the last block $b$ that is included in the blockchain.

\medskip
\noindent
\textbf{Knowledge}

\begin{mydef}
A \emph{knowledge tree} $K$ is a tree $K = (N,E)$ with $N \subseteq \B$ and such that 
every path in $K$ from its root to a leaf belongs to $LOG_{G,V}$.
Let $\mathcal{K}$ be the set of knowledge tree with respect to $(G,V)$.
\end{mydef}

Intuitively, the knowledge tree represents all the blockchain information we know. 
Abusing notation, we say that a block $B$ is in a knowledge tree $K = (N,E)$ if $B \in N$. 
(this is informal) We use $\paths(K)$ to denote the set of all lists of blocks made out of a path 
in $K$ from its root to a leaf. 


\medskip
\noindent
\textbf{Block chain, protocols}

\begin{mydef}
	Let $\preceq_{G,V,t}$ be  a total preorder over $LOG_{G,V}$:
\begin{eqnarray*}		
	&\forall L_1, L_2, L_3 \in LOG_{G,V}, L_1 \preceq_{G,V,t} L_2 \land L_2 \preceq_{G,V,t} L_3 \implies L_1 \preceq_{G,V,t} L_3  \\
	&\forall L_1, L_2 \in LOG_{G,V}, L_1 \preceq_{G,V,t} L_2 \lor L_2 \preceq_{G,V,t} L_1 
\end{eqnarray*}
	A block chain protocol over $LOG_{G,V}$ is a function noted $\preceq_{G,V}$ such that: $$ \forall t \in \mathbb{N}, \preceq_{G,V}(t) =  \preceq_{G,V,t}$$ where $\preceq_{G,V,t}$ is a total preorder over $LOG_{G,V}$
\end{mydef}

\begin{mydef}
Let $t \in \mathbb N$, $\preceq_{G,V}$ a block chain protocol and $K$ a knowledge tree. 
A block chain of $K$ with respect to $\preceq_{G,V}$ in $t$ is any minimal element in $\paths(K)$ with respect to $\preceq_{G,V}(t)$.
\end{mydef}

\medskip
\noindent
\textbf{Action, incentive and game}
\begin{mydef}
	We call action a function $a: \mathcal{K} \rightarrow \mathcal{K}$ such that: 
	\begin{eqnarray*}	
		&\forall K \in \mathcal{K}, K \subseteq a(K) \\
		&\forall K \in \mathcal{K} , |a(K) \setminus K| \leq 1
	\end{eqnarray*}
	Let $A$ be the set of action.
\end{mydef}

\begin{mydef}
	We call incentive a function $I : A \times \mathcal{K} \times P \rightarrow \mathbb{R}^+$ 
\end{mydef}

\begin{mydef}
	Considering a set of player $P$, a function $K_P : P \rightarrow \mathcal{K}$, the set of action $A$, and a incentive $I$. We define a strategic game such that:
	\begin{itemize}
		\item $P$ is the set of player.
		\item $\forall p \in P$, $A$ is the set of available action.
		\item $\forall p \in P, \forall a_1, a_2 \in A$ we say that $a_1$ is preferred to $a_2$ if $I(a_1,K_P(p),p) \geq I(a_2,K_P(p),p)$.
	\end{itemize}
\end{mydef}

Tweak definition a bit to reach finite game (doable if i touch to $A$) and proove equilibrium existence..

\begin{mydef}
	We say that $K'_P$ is reasonable if exists an equilibrium profile $E$ 
	such that: $$\forall p \in P, K'_P(p) = E(p)(K_P(p))$$
\end{mydef}

Good to go we finally have a defintion of reasonable $K$ and can define blockchain property which should be verified over all reasonable $K$.
