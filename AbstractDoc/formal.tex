%!TEX root = main.tex

An attempt to give the most abstract possible definition of a blockchain.

\section{General Comments}
I have assisted to my first class with Aggelos Kiayias (backbone paper author) and a talk this morning about how bitcoin "revolutionised" the historical framework. From my understanding, cryptographic community manage to prove that in the bitcoin open setup (Unknown identity and number of player involved), that a consensus (through a robust transaction ledger) is reachable under the assumption of at least 50\% of honest players. The consensus problem (without prior setup information) upper bound was 33.33..\% of Malicious players.
The new upper-bound, mostly come from the proof of work which avoid malicious player to synchronize (and does not involve the money/reward at all). 

However non of those studies tackle the problem the way we tried, we are in an open setting with unknown player so 50\% of honest player is a strong assumption (knowing that being honest mean following the exact protocol). 

We are proving stuff on a higher abstraction by showing that player have an incentives to behave according to the protocol. 

This only to say that i may have to talk with Aggelos in order not to offense any one regarding the previous state of the art. As if i am following carefully, even terminology as blockchain protocol for our total preorder might be a declaration of war for the crypto community.



\section{BlockChain}

\subsection{Lists and their validation}

Given a set $S$, let $\set(S)$ be the set of sets of elements of $S$ and $\flist(S)$ be the set of all finite lists of elements of $S$. Given $L \in \flist(S)$, we use notation $\length(L)$ to refer to the number of elements in $L$, notation $L[i]$ to refer to the $i$-th element in $L$, where $i \in \{1, \ldots, \length(L)\}$, and notation $L[i,j]$ to refer to the sublist $[L[i], \ldots, L[j]]$ of $L$, where $i,j \in \{1, \ldots, \length(L)\}$ and $i \leq j$. Notice that $\length(L) = 0$ if and only if $L$ is the empty list $[\ ]$. Finally, we say that a list $L_1$ is a prefix of a list $L_2$ if $L_1$ is the empty list, or $1 \leq \length(L_1) \leq \length(L_2)$ and $L_1 = L_2[1, \length(L_1)]$. 

From now on, assume that $\Sigma$ is a finite alphabet, and that $\B \subseteq \Sigma^*$ is the set of all possible blocks.  
%MARCELO: Do we really need the following?
%Moreover we extend the definition of $\subseteq$ such that :
%\begin{eqnarray*}
%	\forall S \in \fset(B), \forall L \in \flist(B), L \subseteq S \Leftrightarrow \forall i \in \{1, \ldots, |L|\}, L[i] \in S
%\end{eqnarray*}

\begin{mydef}
	A validation rule is a function $V : \flist(\B) \to \set(\B)$
\end{mydef}
Intuitively, $V$ is a function taking a finite list $L$ of blocks as input, and returning the set of blocks that could be added to $L$ to produce a valid blockchain.

A list $G \in \flist(\B)$  is said to be a genesis list of $V$ if $\length(G) \geq 1$, $G[1] \in V([\ ])$ and $G[i+1] \in V(G[1,i])$, for every $i \in \{1, \ldots, \length(G)-1\}$. That is, $G$ is a genesis list if $G$ is a non-empty valid blockchain. 
\begin{mydef}
	Let $V$ be a validation rule and $G$ be a genesis list of $V$. Then a list $L \in \flist(\B)$ is valid with respect to $(G,V)$ if:
	\begin{enumerate}
		\item $\length(G) \leq \length(L)$ and $G = L[1,\length(G)]$.
		
		\item $L[i+1] \in V(L[1,i])$, for every $i \in \{\length(G), \ldots, \length(L)-1\}$.
	\end{enumerate}
\end{mydef}
The role of $G$ in this definition is to provide the blocks to startup the system. Let $\LOG(G,V)$ be the set of valid lists with respect to $(G,V)$.

%\begin{myrem}
%	We introduce $LOG_{G,V}$ which really complicated but is actually necessary to deal with fork and consensus later.
%\end{myrem}
%
%\begin{mydef}
%	A set of validated chain $LOG_{G,V}$ is said to be infinite if:
%	$$\forall log_{G,V} \in LOG_{G,V}, \forall i \in \mathbb{N} , \forall b  \in V(log_{G,V}(i)-), V(log_{G,V}(i)-,b)\neq \emptyset $$
%\end{mydef}
%
%\begin{myrem}
%	Infinite here is used in a sense that whatever instance of a $G,V$ validated-chain we are dealing with we will always be able to complete it.
%\end{myrem}

Two lists $L_1, L_2 \in \flist(\B)$ are said to disagree in the last element if one of the following conditions holds: (1) $\length(L_1) = 0$ and $\length(L_2) > 0$, (2) $\length(L_1) > 0$ and $\length(L_2) = 0$, or (3) $\length(L_1) > 0$, $\length(L_2) > 0$ and $L_1[\length(L_1)] \neq L_2[\length(L_2)]$.
\begin{mydef}
	Let $V$ be a validation rule and $G$ be a genesis list of $V$. Then $\LOG(G,V)$ is safe if for every pair $L_1, L_2 \in \flist(\B)$ that disagree in the last element, it holds that $V(L_1) \cap V(L_2) = \emptyset$.
\end{mydef}


\subsection{Body of knowledge}

In this document,  we will give a game-theoretic characterization of the notion of blockchain where it plays a key role the knowledge of each participant. More precisely, given a validation rule $V$ and a genesis list $G$ of $V$, a body of knowledge of $(G,V)$ is a non-empty and finite subset $K$ of $\LOG(G,V)$ satisfying the following closure property:
\begin{itemize}
\item if $L_1 \in \LOG(G,V)$, $L_1$ is a prefix of $L_2$ and $L_2 \in K$, then $L_1 \in K$.
\end{itemize}
Intuitively, if at some iteration a participant considers a list $L \in \LOG(G,V)$ as valid, then she should also consider as valid every prefix of $L$ including the genesis list, that is, every prefix of $L$ belonging to $\LOG(G,V)$. The set of bodies of knowledge of $(G,V)$ is denoted by $\BK(G,V)$. 

There is a natural way to  visualize a body of knowledge $K$ as a graph $\G(K)$. The set of nodes of $\G(K)$ is the set of blocks occurring in the lists in $K$, and there is an edge from a block $b_1$ to a block $b_2$ if there exists a list $L \in K$ such that $b_1 = L[i]$ and $b_2 = L[i+1]$, where $i \in \{1, \ldots, \length(L) -1\}$.
\begin{mylem}
	Assume that $\LOG(G,V)$ is safe. Then for every $K \in \BK(G,V)$, it holds that $\G(K)$ is a tree rooted at $G[1]$.
\end{mylem}

%\juan{Related to the next comment, why do we need $\K(G,V)$? this is just all non-empty finite subset $K$ of $\LOG(G,V)$}\\ 
%
%Thus, assuming that $\LOG(G,V)$ is safe, from now on we refer to every non-empty finite subset $K$ of $\LOG(G,V)$ as a {\em knowledge tree} of $(G,V)$. Moreover, we define $\K(G,V)$ as the set of all knowledge trees of $(G,V)$.

%A \emph{knowledge tree} $K$ is a tree $K = (N,E)$ with $N \subseteq \B$ and such that 
%every path in $K$ from its root to a leaf belongs to $LOG_{G,V}$.
%Let $\mathcal{K}$ be the set of knowledge tree with respect to $(G,V)$.

%Intuitively, the knowledge tree represents all the blockchain information we know. 
%Abusing notation, we say that a block $B$ is in a knowledge tree $K = (N,E)$ if $B \in N$. 
%(this is informal) We use $\paths(K)$ to denote the set of all lists of blocks made out of a path 
%in $K$ from its root to a leaf. 

\subsection{Protocols and blockchain}

\begin{mydef}
	A relation $\preceq$ on $\LOG(G,V)$ is said to be a knowledge order over $(G,V)$ if $\preceq$ is a total preorder on $\LOG(G,V)$, that is, $\preceq$ is reflexive, transitive and total.
	
	Moreover, a sequence $\{ \preceq_i\}_{i \in \mathbb{N}}$ is said to be a blockchain protocol over $(G,V)$ if every $\preceq_i$ $(i \in \mathbb{N})$ is a knowledge order over $(G,V)$.
\end{mydef}

%\begin{mydef}
%	Let $\preceq_{G,V,t}$ be  a total preorder over $LOG_{G,V}$:
%\begin{eqnarray*}		
%	&\forall L_1, L_2, L_3 \in LOG_{G,V}, L_1 \preceq_{G,V,t} L_2 \land L_2 \preceq_{G,V,t} L_3 \implies L_1 \preceq_{G,V,t} L_3  \\
%	&\forall L_1, L_2 \in LOG_{G,V}, L_1 \preceq_{G,V,t} L_2 \lor L_2 \preceq_{G,V,t} L_1 
%\end{eqnarray*}
%	A block chain protocol over $LOG_{G,V}$ is a function noted $\preceq_{G,V}$ such that: $$ \forall t \in \mathbb{N}, \preceq_{G,V}(t) =  \preceq_{G,V,t}$$ where $\preceq_{G,V,t}$ is a total preorder over $LOG_{G,V}$
%\end{mydef}


%\marcelo{Do we really need to use the notion of knowledge tree in the following definitions? If we say that $K$ is a knowledge tree then we need to use $\paths(K)$ to refer to the paths in $K$. On the other hand, if we directly say that $K \in \LOG(G,V)$, we can just refer to the elements of $K$ (we do not a special notation for paths). Knowledge trees are a nice way to visualize the blocks of set $K \in \LOG(G,V)$, but I think we should just mention them because of this (and we should not use them in the definitions). What do you think?}
%
%\juan{You mean $K \subset \LOG(G,V)$ right? I agree that here we do not need safeness nor finitiness to make this work, so we could just stick with $\LOG(G,V)$ instead of all knowledge trees}


\begin{mydef}
	Let $\{ \preceq_i\}_{i \in \mathbb{N}}$ be a blockchain protocol over $(G,V)$, $K \in \BK(G,V)$ and $t \in \mathbb{N}$. 
	Then a maximal element of $K$ with respect to $\preceq_t$ is said to be a blockchain of $K$ at iteration $t$ with respect to the protocol $\{ \preceq_i\}_{i \in \mathbb{N}}$.
\end{mydef}


%\marcelo{I stopped here. I am not totally convinced that we need to introduce the following notion of equivalence, this is something that we need to discuss.}


\subsection{Definition of the game}

Fix a validation rule $V$ and a genesis list $G$ of $V$. From now on, we assume that $\cP = \{1, \ldots, m\}$ is a finite set of players, and we say that a state $\bq$ is a tuple $(q_1, \ldots, q_m) \in \BK(G,V)^m$. Intuitively, each component $q_i$ of $\bq$ represents the knowledge of player $i$, so $\bq$ contains the knowledge of all the players. Moreover, we denote by $\cQ$ the set of all possible states, that is, $\cQ = \BK(G,V)^m$.

%\etienne{There is an issue with : $\cQ = \BK(G,V)^m$ should be closer to :$\cQ = \BK(G,V)^{n \times m}$}

%\begin{mydef}
%	Considering a set of player $P$ we denote $\mathcal{K}_P$ the set of function $K_P : P \rightarrow \mathcal{K}$ mapping a knowledge tree to each player. 
%	%We denote $\mathcal{K}^{\equiv}_P$ the set of mapping where:
%	%$$\forall K_P, K'_P \in \mathcal{K}^{\equiv}_P, \forall p \in P, K_P(p) \textit{ and } K'_P(p) \textit{ are } \equiv \textit{ equivalent } $$ 
%\end{mydef}

%Intuitively $\mathcal{K}_P$ represents the true knowledge of each player.

\begin{mydef}\label{def-action}
Given a player $p \in \cP$, a function $a : \cQ \to \cQ$ is an action for $p$ if
\begin{itemize}
\item for every $\bq \in \cQ$ and $p' \in \cP$, if $\bq = (x_1, \ldots, x_m)$ and $a(\bq) = (y_1, \ldots, y_m)$, then it holds that:
\begin{eqnarray*}
x_{p'} \ \subseteq \ y_{p'} \ \subseteq \ x_{p'} \cup y_p.
\end{eqnarray*}

\end{itemize}
Moreover, $\cA_p$ is the set of all actions for player $p$.
\end{mydef}
An action of a player $p$ is represented by a modification of the knowledge of $p$ and a round of communication between players. 

If we need to restrict the number of blocks that can be added when an action is executed (like in the case of Bitcoin), then we need to include in Definition \ref{def-action} a condition like the following:
\begin{itemize}
\item for every $\bq \in \cQ$, if $\bq = (x_1, \ldots, x_m)$ and $a(\bq) = (y_1, \ldots, y_m)$, then it holds that $|y_p| \leq |x_p| + 1$.
\end{itemize}
In this case, at most one block can be added as the result of executing action $a$ by player $p$ (notice that this holds because every body of knowledge is closed under prefix).

From now on, assume that $\cA = \cA_{1} \times \cA_{2} \times \cdots \times \cA_{m}$. Thus, every element of $\ba \in \cA$ is a tuple containing exactly one action for each player. Moreover, given a player $p \in \cP$, a function $r_p : \cQ \times \cA \to \mathbb{R}$ is called a pay-off function for $p$. Intuitively, given $(\bq, \ba) \in \cQ \times \cA$, we have that $r_p(\bq, \ba)$ is the pay-off of player $p$ when the set of actions to be executed is $\ba$ and the knowledge of each player is encoded in $\bq$. Finally, assuming that there is a  function $r_p$ for each player $p \in \cP$, define $\cR = (r_1, \ldots, r_m)$ as the pay-off function of the game.

As a last component of the game, we assume that $\pr : \cQ \times \cA \times \cQ \to [0,1]$ is a transition probability function satisfying the following conditions:
\begin{enumerate}
\item For every $\bq \in \cQ$ and $\ba \in \cA$:
\begin{eqnarray*}
\sum_{\bq' \in \cQ} \pr(\bq, \ba, \bq' ) & = & 1.
\end{eqnarray*}

\item \label{c-z} For every $\bq \in \cQ$, $\ba \in \cA$ and $\bq' \in \cQ$, it holds that $\pr(\bq, \ba, \bq') = 0$ if $\ba = (a_1, \ldots, a_m)$ and $a_i(\bq) \neq \bq'$ for every $i \in \{1, \ldots, m\}$.
\end{enumerate}
Intuitively, $\pr(\bq, \ba, \bq')$ tell us what the probability of generating $\bq'$ from $\bq$ is 0 when one of the actions in the tuple $\ba$ is executed.

Note that condition \ref{c-z} above forces that the communication happens only by the player that is mining (player $i$ such that 
action $a_i$ satisfies $a_i(\bq) = \bq'$). If we want to relax this we should require instead that there is some $i$ such that $a_i(\bq)$ can be extended to $\bq'$, or that $\bq'$ is built by somehow merging some $a_j(\bq)$'s (but might be tricky to define that).


If the knowledge of all the players is given by $\bq \in \cQ$, and a player $p$ decides to execute an action $a_p$, its rewards not only depends on $\bq$ and $a_p$, but also on the actions to be executed by the other players. If the tuple of actions to be executed by all the players is $\ba = (a_1, \ldots, a_m)$, then the computation of the value $r_p(\bq, \ba)$ should take into consideration the knowledge in $\bq$ and the probability that the action executed is $a_p$. Thus, intuitively, if player $p$ foresees to receive $C(\bq, a_p)$ as reward, then we should have that $r_p(\bq, \ba) = C(\bq, a_p) \cdot \pr(\bq, \ba, a_p(\bq))$.

\etienne{I completely agree with what you say about reward function, but i am not sure people will get this is only a fact in our game and not a generality of stochastic game}

Summing up, from now on we consider an infinite stochastic game $\Gamma = (\cP,\cA,\cQ,\cR,\pr)$ where:
\begin{itemize}
	\item $\cP$ is the set of player.
	\item $\cA$ is the set of available action.
	\item $\cQ$ is the set of states.
	\item $\cR$ is the pay-off function.
	\item $\pr$ is the transition probability function.
\end{itemize} 


\subsection{Stationary equilibrium}

A stationary strategy for a player $p$ is a function $s : \cQ \rightarrow \cA_p$. 
We define $\cS_p$ as the set of all strategies for player $p$, and $\cS = \cS_{1} \times \cS_{2} \times \cdots \times \cS_{m}$ as the set of combined strategies for the game (recall that we are assuming that $\cP = \{1, \ldots, m\}$ is the set of players). Thus, every element $\bs$ of $\cS$ is a tuple containing exactly one strategy for each player. 
For a combined strategy $\bs = (s_1, \ldots, s_m)$, 
we also write $\bs(\bq)$ to refer to the tuple of actions $(s_1(\bq), \ldots, s_m(\bq))$. 

%\etienne{We may want to avoid changing between n and m for the player and the state}

Next we define a notion of how likely is reaching a set of states using a particular strategy, when starting at some specific state.
Formally, given an initial state $\bq_0 \in \cQ$ and a strategy $\bs \in \cS$, 
the probability of reaching state $\bq \in \cQ$ in $n$ iterations is recursively defined as follows:
\begin{eqnarray*}
\pr_0^{\bs}(\bq \mid \bq_0) & = &...
\begin{cases}
1 & \text{if } \bq = \bq_0\\
0 & \text{otherwise}
\end{cases}\\
\pr_{n+1}^{\bs}(\bq \mid \bq_0) & = & \sum_{\bq' \in \cQ} \pr_n^{\bs}(\bq'\mid \bq_0) \cdot \pr(\bq', \bs(\bq'), \bq) \quad \quad \quad  \quad \quad \text{ for every } n \in \mathbb{N}
\end{eqnarray*}


\begin{mydef}
Let $p \in \cP$, $\bq_0 \in \cQ$, $\bs \in \cS$ and $\beta \in [0,1]$. Then the $\beta$ discounted pay-off of the player $p$ for the strategy $\bs$ from the state $\bq_0$, denoted by $u_p(\bs \mid \bq_0)$, is defined as:
\begin{eqnarray*}
u_p(\bs \mid \bq_0) & = & (1 - \beta) \cdot \sum_{i=0}^{\infty}\beta^{i} \cdot  \bigg(\sum_{\bq \in \cQ} r_p(\bq,\bs(\bq)) \cdot 
\pr_i^{\bs}(\bq \mid \bq_0)\bigg)
\end{eqnarray*}
\end{mydef}

\etienne{We define utility only over infinity ?}

Given $p \in \cP$, $\bs \in \cS$, with $\bs = (s_1, \ldots, s_m)$, and $s \in \cS_p$, we denote by $(\bs_{-p}, s)$ strategy of the game $(s_1, \ldots s_{p-1},s,s_{p+1}, \ldots, s_{m})$.
\begin{mydef}
Let $\bq_0 \in \cQ$, $\bs \in \cS$ and $\beta \in [0,1]$. Then $\bs$ is a $\beta$ discounted stationary equilibrium for $(\Gamma, \bq_0)$ if for every player $p \in \cP$ and every strategy $s$ for player $p$ $(s \in\cS_p)$, it holds that:
\begin{eqnarray*}
u_p(\bs \mid \bq_0)  & \geq  & u_p ((\bs_{-p},s) \mid \bq_0).
\end{eqnarray*}
\end{mydef}

\subsection{Properties of a blockchain}

We are interested in proving that certain conditions are verified for stationary equilibria of the blockchain game. 
However, we cannot just ask that ``every state reachable with a high probability satisfies these conditions'', or 
``every state not satisfying these conditions is reached with a low probability''. 
The first statement is too weak, as in a game with several rounds 
most likely all states will be reached with low probability, rendering the statement useless. Moreover, the second statement is not strong enough; even if every \emph{bad} state is reached with 
low probability, the set of \emph{bad} states could be large enough so that the probability of reaching it is high, so our game does not satisfy the  required conditions in its most likely outcome. 
\etienne{We have an issue here, we use weak and not strong enough as antonym}

We overcome the issues mentioned in the previous paragraph by considering a statement that talks about sets of states. Recall that we  fixed a validation rule $V$, a genesis list $G$ for $V$ and an infinite stochastic game $\Gamma = (\cP,\cA,\cQ,\cR,\pr)$. Moreover,  define a condition $\cC$ on the set of states $\cQ$ simply as a subset of $\cQ$ ($\cC \subseteq \cQ$). 

\begin{mydef}
Let $\bq_0 \in \cQ$ be an initial state. Then a condition $\cC$ on the set of states $\cQ$ is satisfied by $(\Gamma, \bq_0)$ if:
%	We say that $P$ is verified by $\Gamma$ with a probability $\alpha$ if and only if:
	\begin{itemize}
		\item There exists a $\beta$-discounted stationary equilibrium for $(\Gamma, \bq_0)$
		
		\item For every $\beta$-discounted stationary equilibrium $\bs$ for $(\Gamma, \bq_0)$, it holds that:
\begin{eqnarray*}
\lim_{n \to \infty} \pr_n^{\bs}(\cC \mid \bq_0) & = & 1, 
\end{eqnarray*}
where ${\displaystyle \pr_n^{\bs}(\cC \mid \bq_0) = \sum_{\bq \in \cC} \pr_n^{\bs}(\bq \mid \bq_0)}$. 
		%verifies the following: 
		%for every set $\bQ \in \set{\cQ}$, if $\pr^{\bs}(\bQ \mid \bq_0)\geq(\alpha)$ then $\bQ \cap P \neq \emptyset$.	

		\end{itemize}
\end{mydef}

%\etienne{the paper probably miss some input (definition of probability over Condition. But to be honest i am sceptical on how to reach such properties at the end. Moreover as we define it as a limit over n through infinity it forces us to consider an infinite number of step so states. I would gladly take an update on that}

%The previous definition is a bit strong, we can reduce it by considering only the majority of knowledge in $\bq$

%\juan{still need to iterate over the definition above. I put everything about computing the probability in this separate piece below. }
%\domagoj{Should be $1-\alpha$ above; we want to say that even when the probability of reaching something is small (we want $\alpha$ to be big) we can find an element with the desired property.}
%
%\medskip
%\noindent
%\textbf{Computing $\pr^{\bs}(\bQ \mid \bq_0)$}. Given an initial state $\bq_0 \in \cQ$ and a strategy $\bs \in \cS$, the probability of reaching a state of $\bQ \in \set(\cQ)$  without walking by a state of $\bQ' \in \set(\cQ)$ in $k$ iterations is recursively defined as follows:
%\begin{eqnarray*}
%	\pr_0^{\bs}(\bQ \setminus \bQ' \mid \bq_0) & = &
%	\begin{cases}
%		1 & \text{if } \bq_0 \in \bQ\\
%		0 & \text{otherwise}
%	\end{cases}\\
%	\pr_{k+1}^{\bs}(\bQ,\bQ' \mid \bq_0) & = & \sum_{\bq' \in \cQ \setminus \bQ'} \pr_k^{\bs}(\{\bq'\},\bQ'\mid \bq_0) \cdot \sum_{\bq \in \bQ} \pr(\bq', \bs(\bq'), \bq) \quad \quad \quad  \quad \quad \text{ for every } k \in \mathbb{N}
%\end{eqnarray*}
%
%\begin{mylem}
%	Given an initial state $\bq_0 \in \cQ$ and a strategy $\bs \in \cS$, the probability to reach a state of $\bQ \in \set(\cQ)$ for the first time in $k$-step is equal to $\pr^{\bs}_k(\bQ,\bQ \mid \bq_0)$
%\end{mylem}
%
%\begin{proof}
%	immediate.
%\end{proof}
%
%\begin{myprop}
%	Given an initial state $\bq_0 \in \cQ$ and a strategy $\bs \in \cS$, the probability to reach a state of $\bQ \in \set(\cQ)$  is equal to $$\pr^{\bs}(\bQ \mid \bq_0) = \sum_{i=0}^{+\infty}\pr^{\bs}_i(\bQ,\bQ \mid \bq_0)$$
%\end{myprop}
%
%\begin{proof}
%	to do.
%\end{proof}
%
%
%\section{Block Equivalence}

\subsection{Equivalence between blocks, and games over equivalence classes}

Fix an equivalence relation $\equiv$ on $\B$. Then we say that two lists $L_1, L_2 \in \LOG(G,V)$ are equivalent, denoted by $L_1 \equiv L_2$, if $\length(L_1) = \length(L_2)$ and $L_1[i] \equiv L_2[i]$ for every $i \in \{1, \ldots, \length(L_1)\}$. Moreover, we say that two bodies of knowledge $K_1, K_2 \in \BK(G,V)$ are equivalent, denoted by $K_1 \equiv K_2$, if (i) for every $L_1 \in K_1$, there exists $L_2 \in K_2$ such that $L_1 \equiv L_2$, and (ii) for every $L_2 \in K_2$, there exists $L_1 \in K_1$ such that $L_1 \equiv L_2$. 

\begin{mylem}
$\equiv$ is an equivalence relation on $\BK(G,V)$.
\end{mylem}

%\begin{mydef}
%	Given a validation rule $V$, a genesis list $G$ of $V$ and an equivalence relationship $\equiv$ over $\B$. We say that $K_1 \in \BK(G,V)$ and $K_2 \in \BK(G,V)$ are $\equiv$ equivalent if and only if:
%	\begin{eqnarray*}
%		& \forall L_1 \in K_1, \exists L_2 \in K_2, \forall i \in \llbracket 1,|L_1| \rrbracket, L_1[i] \equiv L_2[i] \\
%		& \forall L_2 \in K_2, \exists L_1 \in K, \forall i \in \llbracket 1,|L_2| \rrbracket, L_1[i] \equiv L_2[i] \\
%	\end{eqnarray*}	
%	By extension we denote $K_1 \equiv K_2$ resp. $L_1 \equiv L_2$ when two body knowledge resp. list are $\equiv$ equivalent.
%\end{mydef}
%We denote $\BK^\equiv(G,V)$ the set of equivalence classes of $\BK(G,V)$
%
%
%\begin{mydef}
%	Given a validation rule $V$, a genesis list $G$ of $V$ and $\{ \preceq_i\}_{i \in \mathbb{N}}$ a blockchain protocol over $(G,V)$ we say that an equivalence relationship $\equiv$ over $\B$ is $\{ \preceq_i\}_{i \in \mathbb{N}}$ compatible if and only if:
%	$$\forall K_1 , K_2 \in \BK(G,V)$$
%	$$K_1 \equiv K_2 \implies \forall i \in \mathbb{N}, \forall L_1 \in \{L | L \in K_1, \forall L' \in K_1, L' \preceq_i L \}, \exists L_2 \in \{L | L \in K_2, \forall L' \in K_2, L' \preceq_i L \}, L_1 \equiv L_2$$
%\end{mydef}


%\subsection{Game with equivalence}

%For now on we consider a game $\Gamma = (\cP,\cA,\cV,\cR,\pr)$ associated to  a validation rule $V$ a genesis list $G$ and a blockchain protocol $\{ \preceq_i\}_{i \in \mathbb{N}}$. 

The previous definitions of equivalence can be extended to states . Given two states $\bq_1, \bq_2 \in \cQ$ such that $\bq_1 = (q_1, \ldots, q_m)$ and $\bq_2 = (q'_1, \ldots, q'_m)$, we say $\bq_1$ and $\bq_2$ are equivalent, denoted by $\bq_1 \equiv \bq_2$, if $q_i \equiv q'_i$ for every $i \in \{1, \ldots, m\}$. 

\begin{mylem}
$\equiv$ is an equivalence relation on $\cQ$.
\end{mylem}


Finally, the previous definitions of equivalence can be extended to actions.
Given a player $p$ and two actions $a_1, a_2 \in \cA_p$, we say that $a_1$ and $a_2$ are equivalent, denoted by $a_1 \equiv a_2$, if for every $\bq \in \cQ$, it holds that $a_1(\bq) \equiv a_2(\bq)$. Moreover, given two actions $\ba_1, \ba_2 \in \cA$ such that $\ba_1 = (a_1, \ldots, a_m)$ and $\ba_2 = (a'_1, \ldots, a'_m)$, we say that $\ba_1$ and $\ba_2$ are equivalent, denoted by $\ba_1 \equiv \ba_2$, if $a_i \equiv a'_i$ for every $i \in \{1, \ldots, m\}$.
\begin{mylem}
For every $p \in \cP$, it holds that $\equiv$ is an equivalence relation on $\cA_p$. Moreover, $\equiv$ is an equivalence relation on $\cA$. 
\end{mylem}


%We say that two view $\bq_1, \bq_2 \in \cQ$ are equivalent regarding $\equiv$ a equivalent relationship over $\B$ noted $\bq_1\equiv \bq_2$ if $$\forall p \in \cP, q_{1p} \equiv q_{2p}$$

%We denote $\cQ^\equiv$ the set of equivalence classes of $\cQ$

Recall that given an equivalence relation $\sim$ on a set $X$, the equivalent class of $a \in X$ is denoted by $[a]_\sim$. Moreover, the set of equivalence classes of $\sim$, or quotient space, is denoted by $X/\!\!\sim$. 

\begin{mydef}
$\equiv$ is consistent with $\Gamma = (\cP,\cA,\cQ,\cR,\pr)$ if the following conditions are satisfied:
\begin{enumerate}
\item For every $p \in \cP$, $\ba \in \cA_p$ and $\bq_1, \bq_2 \in \cQ$ such that $\bq_1 \equiv \bq_2$, it holds that $\ba(\bq_1) \equiv \ba(\bq_2)$.

\item For every $p \in \cP$, $\bq_1, \bq_2 \in \cQ$ and $\ba_1,\ba_2 \in \cA$ such that $\bq_1 \equiv \bq_2$ and  $\ba_1 \equiv \ba_2$, it holds that $r_p(\bq_1, \ba_1) = r_p(\bq_2, \ba_2)$.

\item For every $\ba_1,\ba_2 \in \cA$, $\bq_1, \bq_2 \in \cQ$ and $E \in \quot{\cQ}$ such that $\ba_1 \equiv \ba_2$ and $\bq_1 \equiv \bq_2$, it holds that 
\begin{eqnarray*}
\sum_{\bq \in E} \pr(\bq_1, \ba_1, \bq) & = & \sum_{\bq \in E} \pr(\bq_2, \ba_2, \bq)
\end{eqnarray*}

\end{enumerate}
\end{mydef}
If $\equiv$ is consistent with $\Gamma = (\cP,\cA,\cQ,\cR,\pr)$, then a new game on equivalence relations can be defined as $\Gamma' = (\cP,\cA',\quot{\cQ},\cR',\pr')$, where:
\begin{enumerate}
\item $\cA' = \quot{\cA_1} \times \cdots \times \quot{\cA_m}$, and for every $p \in \cP$, $a \in \cA_p$ and $\bq \in \cQ$, we have that:
\begin{eqnarray*}
[a]_\equiv([\bq]_\equiv) & = & [a(\bq)]_\equiv
\end{eqnarray*}

\item $\cR' = (r'_1, \ldots, r'_m)$, and for every $p \in \cP$, $\bq \in \cQ$ and $\ba \in \cA$ such that $\ba = (a_1, \ldots, a_m)$, we have that:
\begin{eqnarray*}
r_p([\bq]_\equiv, ([a_1]_\equiv, \ldots, [a_m]_\equiv)) & = & r_p(\bq, \ba)
\end{eqnarray*}

\item For every $\bq_1, \bq_2 \in \cQ$ and $\ba \in  \cA$ such that $\ba = (a_1, \ldots, a_m)$, we have that:
\begin{eqnarray*}
\pr'([\bq_1]_\equiv, ([a_1]_\equiv, \ldots, [a_m]_\equiv), [\bq_2]_\equiv) & = & \sum_{\bq \in [\bq_2]_\equiv} \pr(\bq_1, \ba, \bq).
\end{eqnarray*}
\end{enumerate}
It is not difficult to see that this game is well-defined. In particular, for every $\bq \in \cQ$ and $\ba \in  \cA$ such that $\ba = (a_1, \ldots, a_m)$, we have by definition of $\Gamma'$ that:
\begin{eqnarray*}
\sum_{E \in \quot{\cQ}} \pr'([\bq]_\equiv, ([a_1]_\equiv, \ldots, [a_m]_\equiv), E) & = & \sum_{E \in \quot{\cQ}} \sum_{\bq' \in E} \pr(\bq, \ba, \bq')\\
& = & \sum_{\bq' \in \cQ} \pr(\bq, \ba, \bq')\\
& = & 1
\end{eqnarray*}
Moreover, given $\bq_1, \bq_2 \in \cQ$ and $\ba \in \cA$ such that $\ba = (a_1, \ldots, a_m)$, if we have that $[a_i]_\equiv([\bq_1]_\equiv) \neq [\bq_2]_\equiv$ for every $i \in \{1, \ldots, m\}$, then it holds that $\pr'([\bq_1]_\equiv, ([a_1]_\equiv, \ldots, [a_m]_\equiv), [\bq_2]_\equiv) = 0$. For the sake contradiction, assume that $\pr'([\bq_1]_\equiv, ([a_1]_\equiv, \ldots, [a_m]_\equiv), [\bq_2]_\equiv) > 0$. Then by definition of $\Gamma'$, there exists $\bq \in [\bq_2]_\equiv$ such that $\pr'(\bq_1, \ba, \bq) > 0$. Hence, given that 
$\Gamma$ is an infinite stochastic game, there exists $j \in \{1, \ldots, m\}$ such that $a_j(\bq_1) = \bq$. Thus, we conclude that $[a_j]_\equiv([\bq_1]_\equiv) = [a_j(\bq_1)]_\equiv = [\bq]_\equiv = [\bq_2]_\equiv$ (recall that $\bq \equiv \bq_2$ since $\bq \in [\bq_2]_\equiv$), which contradicts our initial assumption. 


%
%
%\begin{mydef}
%	Let $\equiv$ a equivalence relationship over $\B$ we say that $\equiv$ is $\cA$ compatible if $\forall p \in \cP$ and $\forall a \in \cA_p$ we have $$\forall \bq_1,\bq_2 \in \cQ, \bq_1 \equiv \bq_2 \implies a(\bq_1) \equiv a(\bq_2 )$$
%\end{mydef}
%
%\begin{mydef}
%	Let $p \in \cP$,considering $a_1, a_2 \in \cA_p$ and $\equiv$ a equivalence relationship $\cA$ compatible we say that $a_1$ and $a_2$ are equivalent noted $a_1 \equiv a_2$ if and if:
%	$$\forall \bq_1,\bq_2 \in \cQ, \bq_1 \equiv \bq_2 \implies a_1(\bq_1) \equiv a_2(\bq_2)$$
%\end{mydef}
%We denote $\cA^\equiv_p$ the set of equivalence classes of $\cA$ then a element of $\cA^\equiv_p$ is a function $$a^\equiv : \cV^\equiv \rightarrow \cV^\equiv$$.
%
%\begin{myprop}
%	Let $p \in \cP$, $\equiv$ a equivalence relationship $\cA$ compatible and $a^\equiv \in \cA^\equiv_p $ then
%	\begin{itemize}
%		\item for every $\bq^\equiv \in \cQ^\equiv$ and $p' \in \cP$, if $\bq^\equiv = (q_1^\equiv, \ldots, q_n^\equiv)$ and $a^\equiv(\bq^\equiv) = (q_1^{'\equiv}, \ldots, q_n^{'\equiv})$, then it holds that:
%		\begin{eqnarray*}
%			\forall q_{p'} \in q_{p'}^\equiv,\forall q'_{p'} \in q_{p'} ^{'\equiv},\forall q'_p \in q_p^{'\equiv},  q_{p'} \ \subseteq \ q'_{p'} \ \subseteq \ q_{p'} \cup q_p.
%		\end{eqnarray*}
%	\end{itemize}
%\end{myprop}
%
%\begin{proof}
%	to do.
%\end{proof}
%
%
%\begin{myprop}
%	Let $p \in \cP$, and $\equiv$ a $\cA$ compatible equivalence relationship over $\B$
%	then the function $\pr^\equiv : \cV^\equiv \times \cA^\equiv \times \cV^\equiv \rightarrow [0,1]$ such that: 
%	$$\pr^\equiv(\bq^\equiv,\ba^\equiv,\bq^{'\equiv}) = \pr(\bq,\ba,\bq') \mbox{ where : } \bq \in \bq^\equiv \mbox{ and } \bq' \in \bq^{'\equiv} \mbox{ and } \ba \in \ba^\equiv$$ 
%	is well defined and 
%	$$\forall \bq^\equiv \in \cQ^\equiv,\forall \ba \in \cA^\equiv, \sum_{\bq ^{'\equiv} \in \cQ^\equiv} \pr^\equiv(\bq^\equiv, \ba^\equiv, \bq^{'\equiv})  =  1 $$
%\end{myprop}
%\begin{proof}
%	to do.
%\end{proof}
%
%\begin{mydef}
%	Let $\equiv$ a equivalence relationship over $\B$ we say that $\equiv$ is $\cR$ compatible if its $\cA$ compatible and $\forall p \in \cP$ and $\forall \bq \in \cQ$ we have 
%	$$\forall \ba_1,\ba_2 \in \cA, \ba_1 \equiv \ba_2 \implies r_p(\bq,\ba_1) = r_p(\bq,\ba_2)$$
%\end{mydef}
%
%\begin{myprop}
%	Let $p \in \cP$, and $\equiv$ a $\cR$ compatible equivalence relationship over $\B$ then the function $r_p^\equiv : \cQ^\equiv \times \cA^\equiv \rightarrow \mathbb{R}$ such that :
%	$$r_p^\equiv(\bq^\equiv,\ba^\equiv,) = r_p(\bq,\ba) \mbox{ where : } \bq\in \bq^\equiv \mbox{ and } \ba \in \ba^\equiv $$
%	is well defined.
%\end{myprop}
%\begin{proof}
%	to do.
%\end{proof}
%
%\begin{myprop}
%	Let $\equiv$ a $\cR$ compatible equivalence relationship over $\B$ then $\Gamma^\equiv = (\cP,\cA^\equiv,\cV^\equiv,\cR^\equiv,\pr^\equiv)$ is a well defined infinite stochastic game.
%\end{myprop}
%
%\begin{proof}
%	immediate.
%\end{proof}

