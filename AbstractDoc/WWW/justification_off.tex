%!TEX root = main.tex

\subsection{On the pay-off and utility of a miner}\label{sec-wtf}

The next and last component of our game is the reward for miners: how much do they earn for each mined block. Our definitions are based in the following 
two definitions that are common to most cryptocurrencies based on blockchain, and Bitcoin in particular. 

\noindent
(1) Miners receive a one-time reward per each block they mine. For example, in Bitcoin, each miner will receive a pre-set reward (12.5 bitcoins at the time of writing) when she mines a block.

\noindent
(2) The only blocks that are valid are those in the blockchain; if a block is not part of the blockchain then the reward given for mining this block cannot be spent. To prevent drastic changes on what are the valid blocks, the Bitcoin protocol enforces that block rewards can only be spent when there are one hundred or more blocks on top of it. Since a fork of a hundred blocks is practically impossible, this strongly enforces that the miner's reward can only be spent when the block is part of the blockchain.  

So how should the reward function look like? The first naive idea is to give miners the complete payoff whenever they mine a block, thus defining the next state of the game. The problem with this solution is that it provides no incentive to extend the current blockchain. This can be easily fixed by assigning award only to blocks that are at the top of the current blockchain. However, if there are two branches competing to be the blockchain, it would be intuitive that each player would mine on top of the branch that contains more of her blocks in order to keep them in the blockchain. For instance, in state $q'$ of Example \ref{ex-mining}, 0 should prefer to mine on top of the block 110, to keep her last block in the blockchain, but if is the reward for block $110$ has already been cashed in, then player $0$ can equally mine on top of 111 and receive the same reward should she be successful. The way that the Bitcoin protocol goes around this is by paying for blocks which are buried deep into the blockchain (i.e. they have been confirmed one hundred times).

%
%So how should the reward function look like? The first naive idea is to give miners the complete payoff whenever they mine a block on top of  the current blockchain. However, if we do this, 
%
%
%But if we do this we remove both the incentive to maintain this block in the blockchain, and the incentive to fork at some other block. 
%
%
%all incentives to maintain this blockchain, because blocks out of the blockchain are worthless, and also all incentives to maintain a newly mined block in the blockchain. To see this, consider state $q'$ in Example \label{ex-mining}. Intuitively, one would expect player $0$ to try to mine over block $110$ instead of block $111$, in order to try to keep this block in the blockchain. However, is the reward for block $110$ has already been cashed in, then player $0$ does not care about which block to mine next. Another possible solution would be to pay for blocks which are buried deep into the blockchain, say with a number $\ell$ of blocks ahead. 

Therefore, miners should have incentives to mine blocks, but they should also have incentives to maintain their blocks in the blockchain. In our game we 
ensure this by using a reward $r_p(q)$ of a player $p$ in a state $q$ that depends on all the blocks that player $p$ owns in $q$. More formally, $r_p(q)= \sum_{b \in \bchain(q)} r_p(b,q)$, where $r_p(b,q) > 0$ is the reward that player $p$ receives for a block $b$ she owns in the blockchain of a state $q$, and equals zero if $\bchain(q)$ is not defined. In other words, if a miner owns a block, then she will be rewarded for this block in every state where this block is part of the blockchain. This type of reward functions is general enough to express other game-theoretical formalizations of Bitcoin mining \cite{mininggames:2016}.

But since we pay for a block across multiple states, does this mean players receive multiple pay-offs for their blocks? Not at all. 
Our definition of utility ensures that we pay a portion of block $b$'s reward each time it is included in the current blockchain. In other words, 
when a player mines a new block, she will receive the full amount for this block only if she manages to maintain the block in the blockchain up to infinity. Otherwise, if this block 
ceases to be in the blockchain, we only pay a fraction of the full amount. Formally, given a combined strategy $\bs$, we can define the utility of a block $b$ for a player $p$, denoted by $u_p^b(\bs)$,  as follows:
\begin{eqnarray*}
u_p^b(\bs) & =  & (1 - \beta) \cdot  \sum_{q \in \bQ \,:\, b \in \bchain(q)} \beta^{|q|-1} \cdot  r_p(q,b) \cdot \pr^{\bs}(q).
\end{eqnarray*}
For simplicity, here we assume that the game starts in the genesis block $\varepsilon$, and not in an arbitrary state $q_0$. The discount factor in this case is $\beta^{|q|-1}$, since $|\{\varepsilon\}|= 1$.  


To see that we pay each block precisely once, consider a game where all the player mine on the existing blockhain, thus making each state a single branch starting in the genesis block $\varepsilon$. Denote by $\bs_0$ this combined strategy. If a player $p$ won the first block, call it $b_0$, then $b_0$ will be included in every possible state of the game, and will receive some reward in each of these states. Intuitively, since $b_0$ was won in the first step of the game, its value for the player $p$ will be $r_p(q_0,b_0) \cdot \beta$, where $q_0$ is the state $\{\varepsilon,p\}$\footnote{The discount factor means that owning an asset one step in the future implies a deduction of its value by a factor $\beta$, two steps in the future by a factor of $\beta^2$, etc.}. By the definition above we have that $u_p^{b_0}(\bs_0) = (1-\beta) \cdot \sum_{q\in \bQ} \beta^{|q|-1}\cdot r_p(q,b_0)\cdot \pr^{\bs_0}(q)$. Considering how each state looks under $\bs_0$, we can write this as $u_p^{b_0}(\bs_0) = (1-\beta)\cdot\sum_{i\geq 1} \sum_{q : |q| = i }\beta^i\cdot r_p(q,b_0)\cdot \pr^{\bs_0}(q)$. Assuming that $r_p(q,b)=r_p(q',b)$, for all $q,q'$, which is a natural assumption holding true for all the strategies we consider, and in real life protocols, such as the one deployed by e.g. Bitcoin, we get $u_p^{b_0}(\bs_0) = (1-\beta)\cdot r_p(q,b_0)\cdot \sum_{i\geq 1}\beta^i\cdot \sum_{q : |q| = i } \pr^{\bs_0}(q)$. Since the inner sum always add us to 1 (as it counts the probability of reaching a particular state on a single branch under $\bs_0$), and $\sum_{i\geq 1}\beta^i = \frac{\beta}{1-\beta}$, we get that $u_p^{b_0}(\bs_0) = r_p(q_0,b_0) \cdot \beta$, as expected.

More generally, assume that there is a maximum value for the reward of a block $b$ for player $p$, which is denoted by $M_p(b)$. Thus, we have that there exists $q_1 \in \bQ$ such that $b \in q_1$ and $M_p(b) = r_p(b,q_1)$, and for every $q_2 \in \bQ$ such that $b \in q_2$, it holds that $r_p(b,q_2) \leq M_p(b)$. Again, such an assumption is satisfied by the pay-off functions considered in this paper and in other game-theoretical formalizations of Bitcoin mining \cite{mininggames:2016}. Then we have that:
\begin{myprop}\label{prop-ub-block}
For every player $p \in \bP$, block $b \in \bB$ and combined strategy $\bs \in \bS$, it holds that:
\begin{eqnarray*}
u^b_p(\bs) & \leq &  \beta^{|b|} \cdot M_p(b).
\end{eqnarray*}
\end{myprop}
%Thus, the utility obtained by player $p$ for a block $b$ is at most $\beta^{|b|} \cdot M_p(b)$, that is, the maximum reward that she can obtained for the block $b$ in a state multiplied by the discount factor $\beta^{|b|}$, where $|b|$ is the minimum number of steps that has to be performed to reach a state containing $b$ from the initial state $\{\varepsilon\}$. 
%Moreover, a miner can only aspire to get the maximum utility for a block $b$ if once $b$ is included in the blockchain, it stays in the blockchain in every future state. This again tell us that our framework puts a strong incentive for each player in maintaining her blocks in the blockchain.



