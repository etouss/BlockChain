%!TEX root = main.tex

\subsection{On the pay-off and utility of a miner}
\label{sec-pay-uti}
When we consider that long forks are impossible, a reward function which give miners their complete payoff as soon as a block is burried deep enough in the blockchain can be build \cite{mininggames:2016}. However the goal of a game-theoretical formalisation is to prove assumption such as: \textit{long forks are impossible}. Therefore it is a behaviour that our model should allow, and reward according to the protocol. If we follow the rules, when a long fork happens each block burried deep enough in the new blockchain should reward their owner, while the block from the previous chain, even burried deeply, become invalid. A reward model strictly representing this system can not be build. Indeed consider a situation where two long chains were tying and finally resolves, then one of them is becoming the blockchain. In this state, blockchain's new blocks buried deep enough should give reward to their owner but, due to the memoryless nature of reward functions in stochastic games, there is no possibility to distinguish new blocks from already paid ones.

Hence we design our payoff model to mimic the incencitives of the blochain's payement system. More prisely, given a player $p$ and a state $q$, for every block $b \in q$ assume that the reward obtained by $p$ for the block $b$ in $q$ is given by $r_p(b,q)$, so that $r_p(q) = \sum_{b \in q} r_p(b,q)$. This decomposition can be done in a natural and straightforward way for the pay-off functions considered in this paper and in other game-theoretical formalizations of Bitcoin mining \cite{mininggames:2016}. 
%the reward of a player $p$ in a state $q$, denoted $r_p(q)$ is defined as $r_p(q)= \sum_{b \in \bchain(q)} r_p(b,q)$, where $r_p(b,q) > 0$ is the reward that player $p$ receives for a block $b$ she owns in the blockchain of the state $q$, and equals zero if $\bchain(q)$ is not defined. 
Then to enforce the fact that the block reward for the block $b$ is not granted immediately, we pay in Definition \ref{def-utility} a 
%($\beta$-discounted) 
portion of $r_p(b,q)$, for each state $q$ where $b$ is in. In other words, if a miner owns a block, then she will be rewarded for this block in every state where this block is part of the blockchain, in which case $r_p(b,q) > 0$. %This type of reward functions is general enough to express other game-theoretical formalizations of Bitcoin mining \cite{mininggames:2016}.

This means that we might pay the miner infinitely many times for a single block. A natural question is then do we overpay for the blocks? This is where the discount factors in our definition of utility come into play.  More precisely, we pay a portion of block $b$'s reward each time it is included in the current blockchain. In other words, 
when a player mines a new block, she will receive the full amount for this block only if she manages to maintain the block in the blockchain up to infinity. Otherwise, if this block 
ceases to be in the blockchain, we only pay a fraction of the full amount. Formally, given a combined strategy $\bs$, we can define the utility of a block $b$ for a player $p$, denoted by $u_p^b(\bs)$,  as follows:
\begin{eqnarray*}
u_p^b(\bs) & =  & (1 - \beta) \cdot  \sum_{q \in \bQ \,:\, b \in \bchain(q)} \beta^{|q|-1} \cdot  r_p(q,b) \cdot \pr^{\bs}(q).
\end{eqnarray*}
For simplicity, here we assume that the game starts in the genesis block $\varepsilon$, and not in an arbitrary state $q_0$. The discount factor in this case is $\beta^{|q|-1}$, since $|\{\varepsilon\}|= 1$.  


To see that we pay the correct amount for each block, assume that there is a maximum value for the reward of a block $b$ for player $p$, which is denoted by $M_p(b)$. Thus, we have that there exists $q_1 \in \bQ$ such that $b \in q_1$ and $M_p(b) = r_p(b,q_1)$, and for every $q_2 \in \bQ$ such that $b \in q_2$, it holds that $r_p(b,q_2) \leq M_p(b)$. Again, such an assumption is satisfied by most currently circulating cryptocurrencies, by the pay-off functions considered in this paper, and by other game-theoretical formalizations of Bitcoin mining \cite{mininggames:2016}. Then we have that:
\begin{myprop}\label{prop-ub-block}
For every player $p \in \bP$, block $b \in \bB$ and combined strategy $\bs \in \bS$, it holds that:
\begin{eqnarray*}
u^b_p(\bs) & \leq &  \beta^{|b|} \cdot M_p(b).
\end{eqnarray*}
\end{myprop}
Thus, the utility obtained by player $p$ for a block $b$ is at most $\beta^{|b|} \cdot M_p(b)$, that is, the maximum reward that she can obtained for the block $b$ in a state multiplied by the discount factor $\beta^{|b|}$, where $|b|$ is the minimum number of steps that has to be performed to reach a state containing $b$ from the initial state $\{\varepsilon\}$. 
Moreover, a miner can only aspire to get the maximum utility for a block $b$ if once $b$ is included in the blockchain, it stays in the blockchain in every future state. This tell us that our framework puts a strong incentive for each player in maintaining her blocks in the blockchain.

%Of course, one could ask why not simply reward a portion of the block's reward until it is buried hundred blocks deep into the blockchain? One can think of the Bitcoin's protocol working in this way, and it is also studied in other game theoretic formalizations of the protocol \cite{??}. Apart from being able to reason about games in a more elegant way, the main motivation for this approach is that it allows us to prove assumptions such as that arbitrarily long forks are virtually impossible.  


