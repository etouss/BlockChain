
\section{Proofs and Intermediate Results}
\subsection{Convergence of the utility function}
\label{sec-conver}

To ensure that the utility function $u_p(\bs \mid q_0)$ is well defined, we impose the restriction that for every payoff function $\bR = (r_0, \ldots, r_{m-1})$, there exists a polynomial $P$ such that $|r_p(q)| \leq P(|q|)$ for every player $p \in \bP$ and state $q \in \bQ$. In this section, we prove that this is indeed a sufficient condition for $u_p(\bs \mid q_0)$ to be a real number, for which we first need a technical lemma. 

\begin{mylem}\label{lem-prop-k}
Let $q_0 \in \bQ$ and $\bs$ be a combined strategy. Then for every $k \geq 0$, it holds that
\begin{eqnarray*}
\sum_{\substack{q \in \bQ \,: \\ q_0 \subseteq q \text{ {\rm and} } |q| - |q_0| = k}} \pr^{\bs}(q \mid q_0) & = & 1.
\end{eqnarray*}
\end{mylem}

\begin{proof}
We prove the lemma by induction on $k$. For $k=0$ the property trivially holds since $\pr^{\bs}(q_0 \mid q_0) = 1$. Thus, assuming that the property holds for $k$, we need to prove that it holds for $k+1$. We have that:
\begin{align*}
&\sum_{\substack{q \in \bQ \,: \\ q_0 \subseteq q \text{ {\rm and} } |q| - |q_0| = k+1}} \pr^{\bs}(q \mid q_0) \ =\\
&\hspace{30pt}\sum_{\substack{q \in \bQ \,: \\ q_0 \subseteq q \text{ {\rm and} } |q| - |q_0| = k+1}} 
\bigg(\sum_{\substack{q' \in \bQ \,: \\ q_0 \subseteq q' \text{ {\rm and} } |q'| - |q_0| = k}} \pr^{\bs}(q' \mid q_0) \cdot \pr(q',\bs(q'),q)\bigg) \ = \\
&\hspace{30pt}\sum_{\substack{q' \in \bQ \,: \\ q_0 \subseteq q' \text{ {\rm and} } |q'| - |q_0| = k}}
\bigg(\sum_{\substack{q \in \bQ \,: \\ q_0 \subseteq q \text{ {\rm and} } |q| - |q_0| = k+1}} 
 \pr^{\bs}(q' \mid q_0) \cdot \pr(q',\bs(q'),q)\bigg) \ =\\
 &\hspace{30pt}\sum_{\substack{q' \in \bQ \,: \\ q_0 \subseteq q' \text{ {\rm and} } |q'| - |q_0| = k}}
\pr^{\bs}(q' \mid q_0) \cdot \bigg(\sum_{\substack{q \in \bQ \,: \\ q_0 \subseteq q \text{ {\rm and} } |q| - |q_0| = k+1}} 
  \pr(q',\bs(q'),q)\bigg) \ =\\
&\hspace{30pt}\sum_{\substack{q' \in \bQ \,: \\ q_0 \subseteq q' \text{ {\rm and} } |q'| - |q_0| = k}}
\pr^{\bs}(q' \mid q_0) \cdot \bigg(\sum_{\substack{q \in \bQ \,: \\ q_0 \subseteq q,\ |q| - |q_0| = k+1,\ \bs(q') = (a_0, \ldots, a_{m-1}) \text{ {\rm and}}\\
\text{{\rm there exists }} p \in \{0, \ldots, m-1\} \text{ {\rm such that} } q = a_p(q')}}
  \pr(q',\bs(q'),q)\bigg) \ =\\
  &\hspace{30pt}\sum_{\substack{q' \in \bQ \,: \\ q_0 \subseteq q' \text{ {\rm and} } |q'| - |q_0| = k}}
\pr^{\bs}(q' \mid q_0) \cdot \bigg(\sum_{\substack{p \in \{0, \ldots, m-1\} \, : \\ \bs(q) = (a_0, \ldots, a_{m-1})}} \pr(q',\bs(q'),a_p(q'))\bigg) \ =\\
&\hspace{30pt}\sum_{\substack{q' \in \bQ \,: \\ q_0 \subseteq q' \text{ {\rm and} } |q'| - |q_0| = k}}
\pr^{\bs}(q' \mid q_0).
\end{align*}
Hence, given that
\begin{eqnarray*}
\sum_{\substack{q' \in \bQ \,: \\ q_0 \subseteq q' \text{ {\rm and} } |q'| - |q_0| = k}}
\pr^{\bs}(q' \mid q_0) & = & 1
\end{eqnarray*}
by induction hypothesis, we conclude that
\begin{eqnarray*}
\sum_{\substack{q \in \bQ \,: \\ q_0 \subseteq q \text{ {\rm and} } |q| - |q_0| = k+1}}
\pr^{\bs}(q \mid q_0) & = & 1.
\end{eqnarray*}
\end{proof}

\begin{myprop}\label{prop-conv}
Let $p \in \{0, \ldots, m-1\}$, $q_0 \in \bQ$ and $\bs$ be a combined strategy. If there exist a polynomial $P$ such that $|r_p(q)| \leq P(|q|)$ for every $q \in \bQ$, then $u_p(\bs \mid q_0)$ is a real number.
\end{myprop}

\begin{proof}
Notice that if $P$ is a zero polynomial, then the property trivially holds. Thus, we assume that $P$ is a nonzero polynomial. 
Then we have that:
\begin{eqnarray}
\notag
u_p(\bs \mid q_0) & =  & \sum_{q \in \bQ \,:\, q_0 \subseteq q} \beta^{|q|-|q_0|} \cdot  r_p(q) \cdot \pr^{\bs}(q \mid q_0)\\
\notag
& = & \sum_{n=0}^\infty \bigg(\sum_{\substack{q \in \bQ \,: \\ q_0 \subseteq q \text{ {\rm and} } |q| - |q_0| = n}} \beta^{|q|-|q_0|} \cdot  r_p(q) \cdot \pr^{\bs}(q \mid q_0)\bigg)\\
\label{eq-gen-form}
& = & \sum_{n=0}^\infty \beta^n \cdot \bigg(\sum_{\substack{q \in \bQ \,: \\ q_0 \subseteq q \text{ {\rm and} } |q| - |q_0| = n}} r_p(q) \cdot \pr^{\bs}(q \mid q_0)\bigg).
\end{eqnarray}
Let $f : \mathbb{N} \to \mathbb{R}$ be a function defined as:
\begin{eqnarray*}
f(n) & = & \sum_{\substack{q \in \bQ \,: \\ q_0 \subseteq q \text{ {\rm and} } |q| - |q_0| = n}} r_p(q) \cdot \pr^{\bs}(q \mid q_0).
\end{eqnarray*}
Notice that this function is well-defined as there exists a finite number of states $q \in \bQ$ such that $|q| - |q_0| = n$. Then by equation \eqref{eq-gen-form}, we have that:
\begin{eqnarray*}
u_p(\bs \mid q_0) & = & \sum_{n=0}^\infty \beta^n \cdot f(n).
\end{eqnarray*}
Therefore, to show that $u_p(\bs \mid q_0)$ is a real number, we need to show that the series $ \sum_{n=0}^\infty \beta^n \cdot f(n)$ converges, for which we prove that the series $ \sum_{n=0}^\infty |\beta^n \cdot f(n)|$ converges (that is, we show that $ \sum_{n=0}^\infty \beta^n \cdot f(n)$ converges absolutely, which is known to imply that this series is convergent). By definition of function $f$, we have that:
\begin{eqnarray}
\notag
|f(n)| & = & \bigg| \sum_{\substack{q \in \bQ \,: \\ q_0 \subseteq q \text{ {\rm and} } |q| - |q_0| = n}} r_p(q) \cdot \pr^{\bs}(q \mid q_0) \bigg|\\
\notag
& \leq & \sum_{\substack{q \in \bQ \,: \\ q_0 \subseteq q \text{ {\rm and} } |q| - |q_0| = n}} |r_p(q)| \cdot \pr^{\bs}(q \mid q_0)\\
\notag
& \leq & \sum_{\substack{q \in \bQ \,: \\ q_0 \subseteq q \text{ {\rm and} } |q| - |q_0| = n}} P(n) \cdot \pr^{\bs}(q \mid q_0)\\
\label{eq-f-abs}
& = & P(n) \cdot \bigg(\sum_{\substack{q \in \bQ \,: \\ q_0 \subseteq q \text{ {\rm and} } |q| - |q_0| = n}} \pr^{\bs}(q \mid q_0)\bigg).
\end{eqnarray}
We have by Lemma \ref{lem-prop-k} that
\begin{eqnarray*}
\sum_{\substack{q \in \bQ \,: \\ q_0 \subseteq q \text{ {\rm and} } |q| - |q_0| = n}} \pr^{\bs}(q \mid q_0) & = & 1.
\end{eqnarray*}
Hence, we conclude by equation \eqref{eq-f-abs} that:
\begin{eqnarray*}
|f(n)| & \leq & P(n).
\end{eqnarray*}
Thus, we have that:
\begin{eqnarray}\label{eq-bound-p}
\sum_{n=0}^\infty |\beta^n \cdot f(n)| \ = \ \sum_{n=0}^\infty \beta^n \cdot |f(n)|
\ \leq \ \sum_{n=0}^\infty \beta^n \cdot P(n).
\end{eqnarray}
Given that every term in the series $\sum_{n=0}^\infty |\beta^n \cdot f(n)|$ is non-negative, to show that this series converges it is enough to prove that it is bound by a (non-negative) real number. Thus, by equation \eqref{eq-bound-p}, to finish the proof we need to show that the series $\sum_{n=0}^\infty \beta^n \cdot P(n)$ converges. By this can be easily established by using the Ratio Test, as we have that $\beta \in [0,1)$ and
\begin{eqnarray*}
\lim_{n \to \infty} \frac{\beta^{n+1} \cdot P(n+1)}{\beta^{n} \cdot P(n)} \ = \ \beta \cdot \lim_{n \to \infty} \frac{P(n+1)}{P(n)}
\ = \ \beta,
\end{eqnarray*}
since $\lim_{n \to \infty} \frac{P(n+1)}{P(n)} = 1$ as $P$ is a nonzero polynomial.
This concludes the proof of the proposition.
\end{proof}
