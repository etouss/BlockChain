%!TEX root = main.tex
%Etienne 's modification space



\section{Bitcoin}
In this part we apply our framework to bitcoin, we will try to justify every choice we made. 
We will assume that hash function are perfect, meaning ... (no 2 element with the same hash to simplify)

\subsection{Bitcoin 's Verification Rules}
Genesis block ?

Assume that exists a decision procedure $DP_{bit}$ which takes in input a list of blocks $L \in \flist(\B)$ and a block $b \in \B$, and return true if and only if $L+b$ is a valid list of blocks. (A bitcoin full node implement such a decision procedure).

Let the function $V_{bit} : \flist(\B) \rightarrow \set(\B)$ such that:
$$\forall L \in \flist(\B), V_{bit}(L) = \{ b | b \in \B, DP_{bit}(L,b) \}$$

\begin{myrem}
	$V_{bit}$ is also computable as we only have to test a finite number of block, indeed there is a maximal size for blocks in bitcoin.
\end{myrem}

We define the set of well-formed block regarding bitcoin rules as : 
$$\B_{bit} = \{b| \exists L \in \flist(\B), L[0] = G, b \in V_{bit}(L) \}$$

\begin{mylem*}
	From the specification (quote) we get that every block $b \in \B_{bit}$ can be visualize as a mapping from keywords to values: 
	\begin{itemize}
		\item $b("hash") \rightarrow$ hash of the block
		\item $b("previous hash") \rightarrow$ hash of the previous block
		\item ...
	\end{itemize}
\end{mylem*}

We will focus on a few important properties of bitcoin validation rules.
\begin{mylem*}
	The cryptographic rules in bitcoin impose that :
	$$ \forall b_1,b_2 \in \B_{bit}, b_1("hash")=b_2("hash") \implies b_1 = b_2 $$
\end{mylem*}

\begin{mylem*}
	Every bitcoin's block contains the hash of the previous block:
	$$\forall L \in \LOG(G,V_{bit}), \forall b \in V_{bit}(L),b("previous hash") = L[\length(L)]("hash") $$
\end{mylem*}

\begin{myprop}
	$\LOG(G,V_{bit})$ is safe.
\end{myprop}

\begin{proof}
	To do.
\end{proof}

\subsection{Bitcoin's Protocol}

We define the relation $\preceq^{bit}$ over $\LOG(G,V_{bit})$ as : 

$$ L_1 \preceq^{bit} L_2  \Leftrightarrow \length(L_1) \leq \length(L_2)$$

\begin{myprop}
	$\{\preceq^{bit}_i\}_{i\in\mathbb{N}}$ is a blockchain protocol.
\end{myprop}

\begin{proof}
	to do.
	$\forall i\in \mathbb{N}, \preceq^{bit}_i$ is a knowledge order.
\end{proof}

\subsection{Bitcoin's Game}

\etienne{I am thinking of a way to define action, i want the whole set of action like we define it, and i also want the non-mining, non-communication action. I had a miss-conception about the actions, it is just used in order to define the payoff function, this mean that the transition function is not representing the probability that an action finally end, meaning that the transition function associated to non-mining behaviour can be 0.}
The issue is that action presented here are not a good representation of what one would expect. Indeed one would expect to say that a player is mining over a block, without equivalence, we express the fact that one is mining a specific block over a block.


\subsubsection{PayOff Function}
Intuitively we want the payoff function to immediately represent the earning of a miner. However this can not be achieve easily due to possible instability of the main block chain support. 

The first definition of payoff we offer is node orientated meaning that if at least half of the player agree on a block belonging to the blockchain then the block reward is "accessible". An other one is miner orientated meaning that if at least half of the hash power agree on a block belonging to the blockchain then the block reward is "accessible".

%\begin{mydef}
%	We define the set of consensus blocks for $\bq$ and $p$ as :
%	$$\{b | \forall L \in min(q_p), b \in L  \} $$  
%	We say that a block is accepted at state $\bq$ by a set of player $P$ if he belongs to the consensus blocks of a majority of player. We denote $C_{\bq}$ this set of blocks.
%\end{mydef}

%We then define the payoff function for a player $p$, a state $\bq$ and an action vector $\ba$ as : 
%$$ r_p(\bq,\ba) = \sum_{b \in C_{a_p(\bq)}} b("reward") * 1_{b("owner")=p}$$

%\etienne{I dont like it at all but that the idea}


We define the $\alpha \in [0,1]$ consensus body of knowledge for a state $\bq$ as: 
$$q_{\alpha} = \{L | \exists P \in \set(\cP); |P|> \alpha |\cP|;\forall p \in P; L\in q_p \}$$
Moreover we define the set of list :
$$bc_{\alpha}^\bq = \{L |L\in q_{\alpha} ;\forall L' \in q_{\alpha}; L'\preceq^{bit} L \} $$
Finally we define the intersection of this set as the list:
$$L_\alpha^\bq = \bigcap_{L \in bc_{\alpha}} L$$

\begin{mylem*}
	$L_\alpha^\bq \in \LOG(G_{bit},V_{bit})$
\end{mylem*}
\begin{proof}
	To do.
\end{proof}

\begin{mydef}
We define the payoff function for a player $p$, a state $\bq$ and an action vector $\ba$ as : 
$$ r_p(\bq,\ba) = \left ( \sum_{b \in L_{1/2}^{a_p(\bq)}} b("reward") * \mathbb{1}_{b("owner")=p} \right ) * \pr(\bq, \ba, a_p(\bq))$$
\end{mydef}



\etienne{Maybe we should add a cost of mining for electricity}

\subsubsection{Transition probability function}
Here we want the probability to be inferred by a player hash power. Intuitively if a player $p$ own a third of the hash power which is involved in the action vector then he should have $1/3$ to transition. 

\etienne{it doesn't seem valid at first due to the definition of action which are define as looking for a specific block, but as every action in the vector is defined the same way, the transition probability does not change}



\begin{eqnarray*}
	\pr(\bq, \ba, \bq') & = &
	\begin{cases}
		\frac{\sum_{a_p \in \ba} \mathbb{1}_{a_p(\bq) = \bq'}*p("hashpower")}{\sum_{a_p \in \ba} \mathbb{1}_{a_p(\bq) \neq \bq}*p("hashpower")} \textit{ if }  \bq \neq \bq' \\
		0 \textit{ otherwise } 
	\end{cases}\\
\end{eqnarray*}

\subsection{Equivalence}

\begin{mydef}
	Let $\equiv_{bit}$ be the relation over block of $\B_{bit}$ defined as : 
	\begin{eqnarray*}
		b_1 \equiv_{bit} b_2 & \Leftrightarrow &
		\begin{cases}
			b_1("prevhash") = b_2("prevhash") \\
			b_1("owner") = b_2("owner") \\
			b_1("reward") = b_2("reward")
		\end{cases}\\
	\end{eqnarray*}
\end{mydef}

\begin{myprop}
	$\equiv_{bit}$ is an equivalence relationship which is $R$ compatible. 
\end{myprop}

\begin{proof}
	To do.
\end{proof}
