\documentclass{article}


% \let\bfseriesbis=\bfseries \def\bfseries{\sffamily\bfseriesbis}
%
%
% \newenvironment{point}[1]%
% {\subsection*{#1}}%
% {}
%
% \setlength{\parskip}{0.3\baselineskip}


%% USEFUL packages
\usepackage{mypackages}

%% USEFUL macros
%% Macros Juan

\newcommand{\paths}{\text{PATHS}}

%% Macros comments
\newcommand{\tover}[1]{\textcolor{red}{#1}}
\newcommand{\td}[1]{\textcolor{blue}{[TODO: #1]}}

%% Macros logics
\newcommand{\NN}{\mathbb{N}}
\newcommand{\ZZ}{\mathbb{Z}}
\newcommand{\MM}{\mathbb{M}}
\newcommand{\SE}{\mathbb{S}}
\newcommand{\BB}{\mathbb{B}}
\newcommand{\RR}{\mathbb{R}}
\newcommand{\cF}{\mathcal{F}}
\newcommand{\cI}{\mathcal{I}}
\newcommand{\QFBILIA}{\textsf{QFBILIA}}
\newcommand{\nnf}{\textsf{f}}

\newcommand{\ite}{\textsf{ite}}
\newcommand{\limp}{\Rightarrow}
\newcommand{\flc}{\rightarrow}

\newcommand{\bagone}[1]{\llbracket #1 \rrbracket}
\newcommand{\bsingle}{\textsf{bag}}
\newcommand{\bplus}{\oplus}
\newcommand{\bminus}{\ominus}

%% Macros tools
\newcommand{\spen}{\textsc{spen}}
\newcommand{\zzz}{\textsc{Z3}}

%% Environments
\newtheorem{mythm}{Theorem}[section]
\newtheorem{mydef}[mythm]{Definition}
\newtheorem{myprop}[mythm]{Proposition}
\newtheorem{mylem}[mythm]{Lemma}
\newtheorem{myex}[mythm]{Example}
\newtheorem{mycor}[mythm]{Corollary}

\newtheorem*{myrem}{Remark} %% based on amsthm
\newtheorem*{mynota}{Notation}
\newtheorem*{mylem*}{Lemma}
\newtheorem{myclaim}{Claim}
\newtheorem*{myclaim*}{Claim}
\newtheorem*{myprop*}{Proposition}
\newtheorem*{comp}{Efficiency Study}


\newenvironment{point}[1]
{\subsection*{#1}}%
{}

\newcommand{\flist}{\text{\sc FList}}
\newcommand{\set}{\text{\sc Set}}
\newcommand{\fset}{\text{\sc FSet}}
\newcommand{\B}{\text{\bf B}}
%\newcommand{\G}{\mathcal{G}}
\newcommand{\K}{\mathcal{K}}
\newcommand{\LOG}{\text{\sc Log}}
\newcommand{\length}{\text{\rm length}}
\newcommand{\BK}{\text{\sc BK}}
%\newcommand{\cP}{\mathcal{P}}
%\newcommand{\cV}{\mathcal{V}}
%\newcommand{\cC}{\mathcal{C}}
%\newcommand{\cS}{\mathcal{S}}
%\newcommand{\cA}{\mathcal{A}}
%\newcommand{\cR}{\mathcal{R}}
%\newcommand{\cQ}{\mathcal{Q}}
\newcommand{\cG}{\mathcal{G}}
\newcommand{\cT}{\mathcal{T}}
\newcommand{\pr}{\mathbf{Pr}}
\newcommand{\Dyck}{\mathcal{D}}
\newcommand{\expected}{\mathbf{E}}
\newcommand{\bv}{\mathbf{v}}
\newcommand{\bV}{\mathbf{V}}
\newcommand{\bs}{\mathbf{s}}
\newcommand{\bsigma}{\mathbf{\sigma}}
\newcommand{\bw}{\mathbf{w}}
\newcommand{\ba}{\mathbf{a}}
\newcommand{\bq}{\mathbf{q}}
\newcommand{\bx}{\mathbf{x}}
\newcommand{\by}{\mathbf{y}}

\newcommand{\bstring}{\{0,1\}^\ast}

\newcommand{\marcelo}[1]{{\color{red} {\bf Marcelo: #1}}}
\newcommand{\etienne}[1]{{\color{blue} {\bf Etienne: #1}}}
\newcommand{\juan}[1]{{\textcolor{violet} {\bf Juan: #1}}}
\newcommand{\domagoj}[1]{{\color{green} {\bf Domagoj: #1}}}
\newcommand{\francisco}[1]{{\color{magenta} {\bf Francisco: #1}}}
\newcommand{\martin}[1]{{\color{orange} {\bf Martin: #1}}}

\newcommand{\quot}[1]{#1/\!\equiv}

\newcommand{\then}{\,|\,}
\newcommand{\body}{q}
\newcommand{\bchain}{\text{bc}}
\newcommand{\epath}{\text{path}}

\newcommand{\owner}{\text{\rm owner}}
\newcommand{\pred}{\text{\rm pred}}
\newcommand{\mine}{\text{\rm mine}}
\newcommand{\suc}{\text{\rm succ}}


\newcommand{\bP}{\mathbf{P}}
\newcommand{\bB}{\mathbf{B}}
\newcommand{\bA}{\mathbf{A}}
\newcommand{\bR}{\mathbf{R}}
\newcommand{\bS}{\mathbf{S}}
\newcommand{\bH}{\mathbf{H}}
\newcommand{\bQ}{\mathbf{Q}}

\newcommand{\forkm}[1]{F^{#1}}
\newcommand{\mfork}{F_m}
\newcommand{\mgfork}{F_{m,g}}
\newcommand{\last}{\text{\rm last}}
\newcommand{\best}{\text{\rm best}}
\newcommand{\cho}{\text{\rm choose}}

\newcommand{\ie}{i.e.$\!$ }

\newcommand{\longest}{{\text{\rm longest}}}


\newcommand{\subbody}{{\text{\rm sub-state}}}

\newcommand{\cdf}{\text{\rm {\bf DF}}}
\newcommand{\df}{\text{\rm DF}}
\newcommand{\fg}{\text{\rm FG}}
\newcommand{\fr}{\text{\rm FR}}
\newcommand{\bdf}{\text{\rm {\bf DF}}}
\newcommand{\bfg}{\text{\rm {\bf FG}}}
\newcommand{\bfr}{\text{\rm {\bf FR}}}


%\newcommand{\af}{{\rm {F_\infty}}}
%\newcommand{\baf}{{\rm {\bf F_\infty}}}

\newcommand{\af}{\text{\rm AF}}
\newcommand{\baf}{\text{\rm {\bf AF}}}

\newcommand{\pf}[1]{\text{\rm F[{#1}]}}
\newcommand{\bpf}[1]{\text{\rm {\bf F}[{$#1$}]}}

\newcommand{\cat}{\text{\rm Catalan}}


\newcommand{\gup}{{\rm {G}}}
\newcommand{\bgup}{{\rm {\bf G}}}


\newcommand{\catalan}{{\rm {\it C}}}
%\newcommand{\dyck}{{\rm {\it D}}}


\newcommand{\meet}{\text{\rm meet}}

\DeclareMathOperator*{\argmax}{argmax}


%\newcommand{\rpa}{r_p^\alpha}
\newcommand{\rpa}{r_p}

\newcommand{\ameet}{\text{\rm all-meet}}
\newcommand{\base}{\text{\rm base}}

\newcommand{\ucl}{\text{\rm up-cl}}
\newcommand{\dcl}{\text{\rm down-cl}}

\newcommand{\pol}{\text{\it P}}







%% Title
% \title{}
% \author{}
% \date{}

\begin{document}

% \sloppy
% \maketitle


\section{A Game-theoretic Characterization of Bitcoin Mining}

The mining game is played by a set $\bP = \{1, \ldots, m\}$ of players, with $m \geq 2$. In this game, each player has some reward depending on the number of blocks she owns.\francisco{"in this game, P has some reward" can be more precise, for instance "at any given moment of this game", or "at the end of the game"} We denote by $\bB$ the set of all possible blocks, and we assume that there is a special block $\varepsilon \in \bB$ that is called the genesis block. Moreover, we associate the following functions to these blocks:
\begin{itemize}
\item $\owner : (\bB \setminus \{\varepsilon\}) \to \bP$: This function assigns an owner to each block, except for the genesis block that is assumed not to have an owner.

%\item $\pred : (\B \setminus \{\varepsilon\}) \to \B$: This functions assigns a predecessor to each block, except for the genesis block that is the first block in every blockchain.
\item $\suc: \bB \times \bP \to \bB$: This function tells us which block will a player $p$ use to extend the current blockchain when mining on top of a block $b$. We require this function to be injective and if $\suc(b_1,p)=b_2$, then $\owner(b_2)=p$.
\end{itemize}
%The injectivity and the other condition on $\suc$ tell us that the hash of each block is unique. This function tells us which block can a player $p$ put when mining on top of a block $b_1$.

Note that in Bitcoin there are several different blocks that a player $p$ can use to extend the blockchain when mining upon a block $b$ (depending e.g on the ordering of transactions, or the nonce being used to announce the block). Since we are interested primarily in miners' behaviour, we just focus on the owner of the block following $b$, and do not consider the possibility of two different blocks belonging to $p$ being added on top of $b$. Alternatively, if we consider the Bitcoin protocol, we could say that all the different blocks that $p$ can put on top of $b$ are considered equivalent, since they give $p$ the same reward. %(or a similar one in the actual Bitcoin protocol).

To give a game-theoretic characterization of bitcoin mining, we need to formalize the knowledge each player has. More precisely, given a subset $q$ of $\bB$, define $\cG(q) = (N,E)$ as a graph satisfying the following:
\begin{eqnarray*}
N &=& q\\
E &=& \{ (b_1,b_2) \in q^2 \mid \text{there exists } p \in \bP \text{ such that }\suc(b_1,p) = b_2\}
\end{eqnarray*}
Then $q$ is said to be a body of knowledge (of a player) if $\cG(q)$ is a tree rooted at $\varepsilon$ (in particular, $\varepsilon \in q$). If $q$ is a body of knowledge, then we use notation $\cT(q)$ instead of $\cG(q)$ to make explicit the fact that $\cG(q)$ is a rooted tree.

Given a body of knowledge $q$, we say that the blockchain of $q$ is the longest path in $\cT(q)$, if such a path is unique, in which case we denote it by $\bchain(q)$. If two or more different paths are tied for the longest, then we say that the blockchain in $q$ does not exists, and we assume that $\bchain(q)$ is not defined (so that $\bchain(\cdot)$ is a partial function).

On each step, miners looking to maximise their rewards choose a block in the current body of knowledge, and attempt to mine from this block. Thus, in each turn, each of the players race to place the next block in the body of knowledge, and only one of them succeeds. The probability of succeeding is directly related to the comparative amount of hash power available to this player, the more hash power the likely it is that she will mine the next block before the rest of the players. Once a player places a block, this block is added to the current state, obtaining a different body of knowledge, and the game continues from this new state. 

Let $Q$ be the set of all possible bodies of knowledge, and let $\bQ = Q^m$. Each tuple $\bq = (q_1, \ldots, q_m)$ in $\bQ$ is a state of the mining game, where each component $q_p$ of $\bq$ is a body of knowledge that represents the knowledge of player $p$.

Given a player $p \in \bP$, a block $b \in \bB$ and bodies of knowledge $q, q' \in Q$, we denote by $\mine(p,b,q,q')$ the action played in the mining game, in which player $p$ mines block $b$, places the block $\suc(b,p)$ in her current body of knowledge $q$ and decides to disclosure a portion $q'$ of $q \cup \{\suc(b,p)\}$. Thus, action $\mine(p,b,q,q')$ is valid if:
\begin{itemize}
%\item $\owner(b) = p$,

\item $b \in q$,

%\item $\pred(b) \in q$, and

\item $\suc(b,p) \not\in q$

\item $q' \subseteq q \cup \{\suc(b,p)\}$.
\end{itemize}
Notice that in the previous definition $q'$ is assumed to be a body of knowledge, so that the condition $q' \subseteq q \cup \{\suc(b,p)\}$ is equivalent to the condition that $\cT(q')$ is a subtree of $\cT(q \cup \{\suc(b,p)\})$ rooted at $\varepsilon$. Moreover, action $\mine(p,b,q,q')$ is said to be valid in a state $\bq = (q_1, \ldots, q_m)$ in $\bQ$ if $\mine(p,b,q,q')$ is a valid action and $q = q_p$.

Let $p \in \bP$ be a player, $\bx =  (x_1, \ldots, x_m)$ a state in $\bQ$ and $\mine(p,b,x_p,q)$ a valid action of $p$ in state $\bx$. Then the result of applying $\mine(p,b,x_p,q)$ to $\bx$ is a state $\by = (y_1, \ldots, y_m)$ in $\bQ$ such that:
\begin{itemize}
\item $y_p = x_p \cup \{\suc(b,p)\}$, and

\item for every $i \in \{1, \ldots, m\}$ such that $i \neq p$, it holds that $y_i = x_i \cup q$.
\end{itemize}
In a full disclosure scenario all allowed actions are of the form $\mine(p,b,q,q \cup \{\suc(b,p)\})$, that is, all the information of player $p$ after placing $\suc(b,p)$ in its body of knowledge is sent to the other players. In this case, we simplify the notation and use $\mine(p,b,q)$ instead of $\mine(p,b,q,q\cup \{\suc(b,p)\})$.

Given a player $p \in \bP$, the set of actions for player $p$ is defined as:
\begin{eqnarray*}
\bA_p & = & \{ \mine(p,b,q,q') \mid b \in \bB, q, q' \text{ are bodies of knowledge and }\mine(p,b,q,q') \text{ is a valid action}\}.
\end{eqnarray*}
\francisco{In the latter definition, I think we have to include some conditions on $p,b,q,q'$... if for no better reason, they are not defined in the scope of the set definition.} \marcelo{Done}
Given an action $a \in \bA_p$ and a state $\bq \in \bQ$ such that $a$ is a valid action in $\bq$, we use $a(\bq)$ to denote the state resulting of applying $a$ to $\bq$. Moreover, we denote by $\bA$ the set of all possible actions, that is, $\bA = \bA_1 \cup \cdots \cup \bA_m$.

Given a player $p \in \bP$ and a state $\bq \in \bQ$, the pay-off of player $p$ in $\bq$ is denoted by $r_p(\bq)$. Moreover, assuming that there is a  function $r_p$ for each player $p \in \bP$, define $\bR = (r_1, \ldots, r_m)$ as the pay-off function of the game.

Finally, we assume that the hash power of each player $p$ is given by a number $h_p \in (0,1)$ that represents the probability that player $p$ succeeds in placing the next block. Thus, we assume that:
\begin{eqnarray*}
\sum_{p=1}^m h_p & = &  1, 
\end{eqnarray*}and we define $\bH = (h_1, \ldots, h_m)$ as the hash power distribution.


Summing up, from now on we consider an infinite stochastic game $\Gamma = (\bP,\bA,\bQ,\bR,\bH)$ where:
\begin{itemize}
	\item $\bP$ is the set of players.
	\item $\bA$ is the set of possible actions.
	\item $\bQ$ is the set of states.
	\item $\bR$ is the pay-off function.
	\item $\bH$ is the hash power distribution.
\end{itemize} 


\subsection{Stationary equilibrium}
A strategy for a player $p \in \bP$ is a function $s : \bQ \rightarrow \bA_p$. 
We define $\bS_p$ as the set of all strategies for player $p$, and $\bS = \bS_{1} \times \bS_{2} \times \cdots \times \bS_{m}$ as the set of combined strategies for the game (recall that we are assuming that $\bP = \{1, \ldots, m\}$ is the set of players). 

Next we define a notion of how likely is reaching a state using a particular strategy, when starting at some specific state. Formally, given an initial state $\bq_0 \in \bQ$ and a strategy $\bs = (s_1, \ldots, s_m)$ in $\bS$, 
the probability of reaching state $\bq \in \bQ$ such that $\bq_0 \subseteq \bq$ is recursively defined as follows:
\begin{eqnarray*}
\pr^{\bs}(\bq \mid \bq_0) & = &
\begin{cases}
1 & \text{if } \bq =  \bq_0\\
& \\
{\displaystyle \sum_{\substack{\bq' \in \bQ \,:\\ \bq_0 \subseteq \bq' \text{ and } |\bq'| - |\bq_0| = k-1}} \pr^{\bs}(\bq' \mid \bq_0) \cdot \bigg(\sum_{\substack{p \in \{1, \ldots, m\} \,: \\ s_p(\bq') = a \text{ and } a(\bq') = \bq}} h_p\bigg)}
 & \text{if } |\bq| - |\bq_0| = k \text{ and } k \geq 1
\end{cases}
\end{eqnarray*}
We finally have all the necessary ingredients to define the pay-off of a player in a mining game given a particular strategy.

\begin{mydef}
Let $p \in \bP$, $\bq_0 \in \bQ$, $\bs \in \bS$, $\beta \in [0,1]$ and $n \geq 0$. Then the $\beta$--discounted utility of player $p$ for the strategy $\bs$ from the state $\bq_0$ in a mining game with $n$ steps, denoted by $u_p^n(\bs \mid \bq_0)$, is defined as:
\begin{eqnarray*}
u_p^n(\bs \mid \bq_0) & = & \sum_{i=0}^{n}\beta^{i} \cdot  \bigg(\sum_{\substack{\bq \in \bQ \,: \\ \bq_0 \subseteq \bq \text{ {\rm and} } |\bq| - |\bq_0| = i}} r_p(\bq) \cdot 
\pr^{\bs}(\bq \mid \bq_0)\bigg)
\end{eqnarray*}
Moreover, the $\beta$--discounted utility of player $p$ for the strategy $\bs$ from the state $\bq_0$ in an infinite mining game, denoted by $u_p(\bs \mid \bq_0)$, is defined as:
\begin{eqnarray*}
u_p(\bs \mid \bq_0) & = & \sum_{i=0}^{\infty}\beta^{i} \cdot  \bigg(\sum_{\bq \in \bQ \,:\, |\bq| = i} r_p(\bq) \cdot 
\pr^{\bs}(\bq \mid \bq_0)\bigg)
\end{eqnarray*}
\end{mydef}
\francisco{Should we call this the {\it cumulative} $\beta$--discounted utility? In light of the fact that it counts dependent paths, representing accumulated reward?}
Given a player $p \in \bP$, a combined strategy $\bs \in \bS$, with $\bs = (s_1, \ldots, s_m)$, and a strategy $s$ for player $p$ ($s \in \bS_p$), we denote by $(\bs_{-p}, s)$ the strategy $(s_1, \ldots s_{p-1},s,s_{p+1}, \ldots, s_{m})$.
\begin{mydef}
Let $\bq_0 \in \bQ$, $\bs \in \bS$, $\beta \in [0,1]$ and $n \geq 0$. Then $\bs$ is a $\beta$ discounted stationary equilibrium in a mining game with $n$ steps if for every player $p \in \bP$ and every strategy $s$ for player $p$ $(s \in\bS_p)$, it holds that:
\begin{eqnarray*}
u_p^n(\bs \mid \bq_0)  & \geq  & u_p^n((\bs_{-p},s) \mid \bq_0).
\end{eqnarray*}
Moreover, $\bs$ is a $\beta$ discounted stationary equilibrium  in  the infinite mining game if for every player $p \in \bP$ and every strategy $s$ for player $p$ $(s \in\bS_p)$, it holds that:
\begin{eqnarray*}u_p(\bs \mid \bq_0)  & \geq  & u_p ((\bs_{-p},s) \mid \bq_0).
\end{eqnarray*}
\end{mydef}



\section{Full Disclosure Scenario and Two Players}
In this section, we consider $\bP = \{1,2\}$, and we assume full disclosure of information between players.
For every state $\bq \in \bQ$ with $\bq = (q_1, \ldots, q_m)$, in this case it holds that $q_i = q_j$ for every $i,j \in \{1, \ldots, m\}$. Thus, in this section we simplify the notation and use a body of knowledge $q$ to represent a state of the game, instead of a tuple $\bq$ of the form $(q, \ldots, q)$.

For $p \in \{1,2\}$, we define the indicator  function $\chi_p : \bB \to \{0,1\}$ as follows:
\begin{eqnarray*}
\chi_p(b) & = & 
\begin{cases}
1 & \owner(b) = p\\
0 & \text{otherwise}
\end{cases}
\end{eqnarray*}
\francisco{I actually like this notation, just pointing out this other one: $\mathbb 1_p$}

Given a body of knowledge $q$ and a path $\pi$ in $\cT(q)$ from the root $\varepsilon$, the length of $\pi$ is defined as the number of edges in $\pi$, and it is denoted by $|\pi|$. Moreover, given $i \in \{0, \ldots, |\pi|\}$, we denote the $i$-th block of
 $\pi$ by $\pi[i]$ (assuming that the first block of the path is in position 0). 
 
In this section, we consider the following pay-off functions:
\begin{itemize}
\item {\bf Constant $d$-delayed reward.} The pay-off of a player $p \in \bP$ is defined as:
\begin{eqnarray*}
r_p(q) & = & 
\begin{cases}
0 & \text{if } \bchain(q) \text{ is not defined}\\
{\displaystyle c \cdot \sum_{i=0}^{|\bchain(q)|-d} \chi_p(\bchain(q)[i])} & \text{otherwise}
\end{cases}
\end{eqnarray*}
where $c$ is a positive real number and $d \in \mathbb{N}$. Here the number $d$ is the amount of confirmations needed to spend the block (6 in the case of Bitcoin).

\item {\bf $\alpha$-discounted  $d$-delayed reward.} The pay-off of a player $p \in \bP$ is defined as:
\begin{eqnarray*}
r_p(q) & = & 
\begin{cases}
0 & \text{if } \bchain(q) \text{ is not defined}\\
{\displaystyle c \cdot \sum_{i=0}^{|\bchain(q)|-d} \alpha^i \cdot \chi_p(\bchain(q)[i])} & \text{otherwise}
\end{cases}
\end{eqnarray*}
where $c$ is a positive real number, $\alpha \in (0,1]$ and $d \in \mathbb{N}$. 

\item {\bf $\alpha$-discounted $d$-delayed $k$-block reward.} The pay-off of a player $p \in \bP$ is defined as:
\begin{eqnarray*}
r_p(q) & = & 
\begin{cases}
0 & \text{if } \bchain(q) \text{ is not defined}\\
\\
{\displaystyle c \cdot \bigg(
\sum_{i=0}^{\lfloor\frac{|\bchain(q)|-d}{k}\rfloor-1}
\sum_{j=i \cdot k}^{(i+1) \cdot k -1} \alpha^i \cdot \chi_p(\bchain(q)[j])
\ +}\\
{\displaystyle  \hspace{50pt} \sum_{j= \lfloor\frac{|\bchain(q)|-d}{k}\rfloor \cdot k + 1}^{|\bchain(q)|-d} \alpha^{\lfloor\frac{|\bchain(q)|-d}{k}\rfloor} \cdot \chi_p(\bchain(q)[j])\bigg)}
& \text{otherwise}
\end{cases}
\end{eqnarray*}
where $c$ is a positive real number, $\alpha \in (0,1]$ and $k$ is a natural number greater than 0.
\end{itemize}

The next step step is to describe different strategies that the players can use in the game. For this we need some notation. 

Assume that $q$ is a body of knowledge. Then for every path $\pi$ in $q$ from the root $\varepsilon$, we associate 


First, for a state $q$, we will denote by $\last(\bchain(q))$ the final block in the path $\bchain(q)$, when $\bchain(q)$ is defined, and we leave it undefined otherwise. Similarly, if $q$ is any path (i.e. a body of knowledge with no branches), we define $\last(q)$ as the final block in this path. If $p$ is a player, and $q$ is a state such that $\bchain(q)$ is not defined, we will define the function $\cho_p(q)$ as the block that is most convenient for the player $p$ to mine upon. Formally, if $S=\{ q'\mid q' \text{ is a path in } \cT(q) \text{ from the root to a leaf}\}$, we define $\cho_p(q)$ as the lexicographically smallest $q'\in S$ such that $r_p(q')=max_{q''\in S} \{r_p(q'')\}$.

%To this end, given a player $p \in \bP$ and a body of knowledge $q$,
%
%
% Let $q$ be a body of knowledge. 
%
%
%terminology. 
%
%
%given a body of knowledge, we assume that 


For the mining game we will consider the following strategies:
\begin{itemize}
\item {\bf The default strategy.}  The first strategy we describe will be called $\df$, and it will reflect the desired behaviour of the miners participating in the Bitcoin network. Intuitively, in a state $q$, a player following this strategy will try to mine upon the final block that appears in the blockchain of $q$. If the blockchain in state $q$ does not exist, meaning that there are two longest paths from the genesis block, the player will mine on the final block of the path that contains the highest number of her blocks. We call this strategy \df, and we define it formally as follows:

\begin{eqnarray*}
\df_p(q) & = &
\begin{cases}
\mine(p,\last(\bchain(q)),q) & \text{if } \bchain(q) \text{ exists }\\
\mine(p,\cho(q),q) & \text{if } \bchain(q) \text{ does not exist }
\end{cases}
\end{eqnarray*}

%Here $last(\bchain(q))$ returns the last block in $\bchain(q)$, and $best(q)$ returns the last block of the path that is of maximal length in $q$, and on which the player $p$ has the highest number of blocks compared to all maximal paths in $q$. If there is more than one such path, $best(q)$ is the one that is smallest lexicographically. Intuitively, $best(q)$ is the block on which a benevolent player will mine upon when it is not clear what the blockchain is.

\item {\bf Fork on the $k$th from the end of the blockchain.} If we assume two players, one of them playing the default strategy, and the other will fork only once, this means that the fork will happen in some state where the blockchain is defined. This strategy says that the player that will fork, does this by mining on a block that is $k$ blocks away from $\last(\bchain(q))$. Following this, the player always mines on the last block of this chain. $k=\infty$ means fork on genesis. 

\item {\bf Fork on the $k$th block belonging to me counting from the end of the blockchain.} Similar to the previous strategy, but this time the player will mine on the $k$th block belonging to her, counting from $\last(\bchain(q))$. Following this, the player always mines on the last block of this chain. With $k=1$ the player are forking on her ultimate block in the blockchain, and with $k=\infty$ in the genesis.

\item {\bf Give up time $g$.} This can be a parameter in any of the above strategies. Once forked, if the branch belonging to the non forking player is $g$ block ahead of the forking branch, the game continues on this branch with no more forks.
\end{itemize}




\end{document}
