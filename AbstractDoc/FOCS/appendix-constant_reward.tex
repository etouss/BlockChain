%!TEX root = focs.tex

\section{Proofs and definitions for Section \ref{sec-const_rew}}
\label{app-const}

DEFINE STRATEGIES HERE??

OR IN AN APPENDIX BEFORE THIS ONE?

\paragraph{Longest paths in greedy strategies.}

\begin{mylem}\label{lem-length-greedy}
Let $\bs$ be a greedy strategy. Then for every $q \in \bQ$ such that $\pr^{\bs}(q \mid \varepsilon) > 0$, the following conditions hold:
\begin{enumerate}
\item For every $p \in \{0,1\}$: $|\longest(q,p)| = 1$ 

\item $1 \leq |\longest(q)| \leq 2$

\item If $|\longest(q)| = 2$, then $\longest(q) = \longest(q,0) \cup \longest(q,1)$
\end{enumerate}
\end{mylem}

\begin{proof}
Let $S = \{ q \in \bQ \mid \pr^{\bs}(q \mid \varepsilon) > 0 \}$. Then we have that $S$ is the smallest subset of $\bQ$ satisfying the following conditions:
\begin{itemize}
\item $\{\varepsilon\} \in S$.

\item If $p \in \{0,1\}$, $b \in \bB$, $q \in S$ and $\mine(p,b,q)$ is a greedy action, then $q \cup \{ b \cdot p\} \in S$.
\end{itemize}
Hence, we have an inductive definition of $S$, and we can prove the proposition by induction on the structure of this set of states. If $q = \{ \varepsilon \}$, then we have that $\longest(q) = \longest(q,1) = \longest(q,2) = \{ \varepsilon \}$ and, thus, we have that  the three conditions in the proposition hold since $|\longest(q)| = |\longest(q,0)| = |\longest(q,1)| = 1$. Assume that the property holds for $q \in S$, and assume that $p \in \{0,1\}$, $b \in \bB$ and $\mine(p,b,q)$ is a greedy action. Then we need to prove that the conditions in the proposition hold for $q ' = q \cup \{b \cdot p \}$, for which we consider the following cases.
\begin{itemize}
\item Assume that $\longest(q,0) = \{b_0\}$,  $\longest(q, 1) = \{b_1\}$ and $\longest(q) = \{b_0,b_1\}$, and without loss of generality assume that $p = 0$. Given that $|b_0| = |b_1|$ and $\mine(p,b,q)$ is a greedy action, we have that either $b = b_0$ or $b = b_1$. If $b = b_0$, then it holds $b_0 \cdot 0 \in q'$, from which we conclude that the three conditions of the proposition hold since $\longest(q',0) = \{b_0 \cdot 0\}$, $\longest(q',1) = \{b_1\}$ and $\longest(q') = \{b_0 \cdot 0\}$.  If $b = b_1$, then it holds $b_1 \cdot 0 \in q'$, from which we conclude that the three conditions of the proposition hold since $\longest(q',0) = \{b_1 \cdot 0\}$, $\longest(q',1) = \{b_1\}$ and $\longest(q') = \{b_1 \cdot 0\}$.

\item Assume that $\longest(q,0) = \{b_0\}$,  $\longest(q, 1) = \{b_1\}$ and $\longest(q) = \{b_0\}$, so that $|b_1| < |b_0|$. Notice that if $p =0$, then we have that $b=b_0$ since $\mine(p,b,q)$ is a greedy action. Hence, it holds $b_0 \cdot 0 \in q'$, from which we conclude that the three conditions of the proposition hold since $\longest(q',0) = \{b_0 \cdot 0\}$, $\longest(q',1) = \{b_1\}$ and $\longest(q') = \{b_0 \cdot 0\}$. Therefore, assume that $p = 1$, from which we have that  $|b_1| \leq |b|$ and $|b| \leq |b_0|$, since $\mine(p,b,q)$ is a greedy action and $\longest(q) = \{b_0\}$. Thus, given that $b \cdot 1 \in q'$, we conclude that $\longest(q',0) = \{b_0\}$ and $\longest(q',1) = \{b \cdot 1\}$. 
Moreover, if $|b \cdot 1| < |b_0|$, then it holds that $\longest(q') = \{b_0\}$ and the three conditions of the proposition are satisfied. If $|b \cdot 1| = |b_0|$, then we have that $\longest(q') = \{b_0, b \cdot 1\}$, from which we conclude that the three conditions of the proposition are satisfied since $|\longest(q')| = 2$ and $\longest(q') = \longest(q',0) \cup \longest(q',1)$. Finally, if $|b \cdot 1| > |b_0|$ (that is, if $b = b_0$), then $\longest(q') = \{b \cdot 1\}$ and again the three conditions of the proposition are satisfied.

\item Assume that $\longest(q,0) = \{b_0\}$,  $\longest(q, 1) = \{b_1\}$ and $\longest(q) = \{b_1\}$. This case is analogous to the previous case, which concludes the proof of the proposition.
\end{itemize}
\end{proof}


\paragraph{Longest blocks and optimal strategies.}
For a state $q$ and a block $b \in q$, let us denote by $\subbody(q,b)$ the state
given by $\{u \mid b\cdot u$ is a block in $q\}$, that is, the subtree of $q$ rooted at $b$, but in which $b$ is renamed 
$\epsilon$ and all its descendants are renamed accordingly. 

%Furthermore, let us denote by $\meet(q,p)$ the greatest block in the set $\{b \in q \mid b$ is a prefix all nodes in $\longest(q)\}$, the greatest common block owned by $p$ that is a prefix of all 
%blocks in $\longest(q)$. 

The following Lemma tells us that an optimal strategy for player $p$ can only differentiate the portion of 
a state that goes after $\meet(q,p)$: 

\begin{mylem}\label{lem-optimal}
Let $s = (s_1,s_2)$ be a $\beta$ discounted stationary equilibrium in an infinite mining game with two players. 
Then there is a $\beta$ discounted stationary equilibrium such that $u_p(s \mid q_0) = u_p(s' \mid q_0)$ for 
any player $p$ and for every pair $q$ and $q'$ of 
bodies of knowledge in which $\subbody(q,\meet(q,p)) = \subbody(q',\meet(q',p))$ we have that 
$s_p(q) = s_p(q')$. 
\end{mylem}

\begin{proof}
\end{proof}