%!TEX root = focs.tex


\section{Equilibria in games with constant payoff}
\label{sec-const_rew}

The first version of the game we analyse is when the payoff function $r_p(q)$ pays each block in the blockchain the same amount $c$. While this does not reflect the current reality of the Bitcoin protocol, this simplification serves as a good baseline for future results, and it establishes the main techniques we will have to utilize. Furthermore, the results we obtain here could serve as a good recommendation of how the Bitcoin protocol can be modified in order to enforce fair behaviour of the miners when the block reward becomes insignificant.

\subsection{Defining constant pay-off}
When considering the constant reward $c$ for each block, $r_p(q)$ will equal $c$ times the number of blocks owned by $p$ in $\bchain(q)$, when the latter is defined. On the other hand, when $\bchain(q)$ is not defined it might seem tempting to simply define $r_p(q) = 0$. However, even if there is more than one longest branch from the root of $q$ to its leaves, it might be the case that all such branches share a common subpath. In fact, this often happens in the Bitcoin's network, when two blocks were mined on top of the same block (with a small time delay). While in this situation the blockchain is not defined, the miners know that they will at least be able to collect their reward on the portion of the state these two branches agree on. Figure \ref{fig-simple-fork} illustrates this situation. In order to model this scenario we will have to introduce some more notation. 

\begin{figure}
\begin{center}
\begin{tikzpicture}[->,>=stealth',auto,thick, scale = 1.0,state/.style={circle,inner sep=2pt}]

    % The graph
	\node [state] at (0,0) (R) {$\varepsilon$};
	\node [state] at (1.5,0) (1) {$1$};
	\node [state] at (3,0) (10) {$10$};
	\node [state] at (4.5,0) (100) {$100$};
	\node [state] at (6,0) (1001) {$1001$};
	\node [state] at (8,0.75) (10011) {$10011$};
	\node [state] at (8,-0.75) (10010) {$10010$};	
	
	% Graph edges
	\path[->]
	(R) edge (1)
	(1) edge (10)
	(10) edge (100)
	(100) edge (1001)
	(1001) edge (10011)
	(1001) edge (10010)
	;  	
	


\end{tikzpicture} 
\end{center}
\label{fig-simple-fork}
\caption{Although two branches are contesting to become a blockchain, the blocks up to 1001 will contribute to the reward in each case.}
\end{figure}


Recall that a block $b$ is a string over the alphabet $\bP$, and we use notation $|b|$ for the length of $b$ as a string. Moreover, given blocks $b_1, b_2$, we use notation $b_1 \preceq b_2$ to indicate that $b_1$ is a prefix of $b_2$  when considered as strings. Then we define: 
\begin{eqnarray*}
\longest(q) & = & \{ b \in q \mid \text{for every } b' \in q: |b'| \leq |b|\}\\
\meet(q) & = & {\displaystyle \max_{\preceq} \ \{b \in q \mid \text{for every } b' \in \longest(q): b \preceq b'\}}
\end{eqnarray*}
Intuitively, $\longest(q)$ contains the leaves of all branches in the state $q$ that are currently competing for the blockchain, and $\meet(q)$ corresponds to the last block for which all these branches agree on. For instance, if $q$ is the state from Figure \ref{fig-simple-fork}, then we have that $\longest(q)=\{10011,10010\}$, and $\meet(q)=1001$. Notice that $\meet(q)$ is well defined as $\preceq$ is a linear order on the finite and non-empty set $\{b \in q \mid \text{for every } b' \in \longest(q): b \preceq b'\}$. Also notice that $\meet(q)=\bchain(q)$, whenever $\bchain(q)$ is defined.

As mentioned above, the pay-off function will reward a player for the blocks she owns in the path from the genesis block to $\meet(q)$. Thus, to define the pay-off, we need to identify who is the owner of each one of these blocks, which is done by considering the function $\chi_p$, for each $p \in \bP$. More precisely, given $b \in \bB$ and $i \in \{1, \ldots, |b|\}$, we have that:
%First, for a player $p$, define the indicator function $\chi_p : \bB \to \{0,1\}$ as follows:
\begin{eqnarray*}
\chi_p(b,i) & = & 
\begin{cases}
1 & \text{if the } i\text{-th symbol in } b \text{ is } p\\
0 & \text{otherwise}
\end{cases}
\end{eqnarray*}

We can finally define the payoff function we consider in this section, which we call the \textbf{constant reward}. For a player $p$, we define it as 
\begin{eqnarray*}
r_p(q) & = & 
{\displaystyle c \cdot \sum_{i=1}^{|\meet(q)|} \chi_p(\meet(q),i)} 
\end{eqnarray*}
where $c$ is a positive real number. As mentioned above, this function is well defined, since $\meet(q)$ is always defined, and equals $\bchain(q)$ when the latter is defined for the state $q$.
%Here we use notation $b[i]$ for the $i$-th symbol in $b$, where $i \in \{1, \ldots, |b|\}$. 
% and $d \in \mathbb{N}$. Here the number $d$ is the amount of confirmations needed to spend the block (6 in the case of Bitcoin).
 
 \subsection{The default strategy is an equilibrium}

Let us start with analysing the most obvious strategies for all players: regardless of what everyone else does, keep mining on the blockchain. We call this 
the \emph{default} strategy, as is it reflects the desired behaviour of the miners participating in the Bitcoin network. 
For a player $p$, let us denote this strategy 
by $\df_p$, and consider the combined strategy $\df = (\df_0,\df_1,\dots,\df_{m-1})$. Notice that under this strategy, each state $q$ consists of a single path from the genesis block $\varepsilon$ to the final block in $\bchain(q)$.

We can now easily calculate the utility of player $p$ under $\df$. Intuitively, a player $p$ will receive a fraction $h_p$ of the next block that is being placed in the blockchain, corresponding to her hash power. Therefore, at stage $i$ of the mining game, $i$ blocks will be placed in the blockchain defined by the game, and the expected amount of blocks owned by the player $p$ will be $h_p\cdot i$. This means that the total utility for player $p$ amounts to 
$$u_p(\bs \mid \varepsilon) = h_p \cdot c\cdot \sum_{i=0}^{\infty}i \cdot \beta^{i}.$$% = h_p\cdot c \cdot \frac{\beta}{(1-\beta)^2}.$$

%% Juan: I think it is way too nahive to put the analytical form of this
%$$u_p(\bs \mid q_0) = c\cdot h_p \cdot \sum_{i=0}^{\infty}i \cdot \beta^{i} \text{,}\ \ \ \ \text{ which evaluates to }\frac{c\cdot h_p}{(1-\beta)^2}.$$ %(recall that $q_0$ is the genesis block). 

The question the is: can any player do better? As we show, the answer is no if we assume that the rest of the players behave according to $\df$. More precisely, we have the following theorem.

\begin{mythm}\label{thm-conts_equlibria}
For any $0 \leq \beta \leq 1$, the strategy $\df$ is a $\beta$-discounted stationary equilibrium.
\end{mythm} 

\begin{proof}
Let $s_p$ be an arbitrary strategy for player $p$. We need to show that the utility of $(\df_{-p},s_p)$ is not higher than the utility of $\df$. 

First, observe that we just need to consider a two-player game, by merging all the players except $p$ into a single player whose hash power is the sum of the hash power 
of each of the aggregated players. Thus, we just consider $\bP = \{0,1\}$, and for readability we assume $p = 0$ (the other case being symmetric). 

For a strategy $s$, let $Q_s = \{q^* \in \bQ \mid \pr^s(q^* \mid \epsilon) > 0\}$ be the set of states that can be reached from the genesis block using $s$. 
We define a mapping $\sigma: Q_{(s_0),\df_1)} \rightarrow 2^{Q_\df}$ as follows. 
For each state $q \in Q_{(s_0),\df_1)}$, enumerate all distinct paths $\pi = q_0,\dots,q_n$ such that $q_{i+1}$ can be reached from 
$q_i$ in one step, $q_0 = \varepsilon$ and $q_n = q$. To each such sequence $\pi$ we associate a word $b_1,\dots,b_n$ in $\{0,1\}^n$ as follows: 
$b_i = 0$, if $q_{i} = a_0(q_{i-1})$, where $a_0 = s_0(q_{i-1})$, and $b_i = 1$ otherwise, that is, when  $q_{i} = a_1(q_{i-1})$, where $a_1 = \df(q_{i-1})$. 

Then, $\sigma(q)$ contains all states $q^* \in Q_\df$ such that there is a path $\pi$ and a corresponding word $w$ and where 
$q^*$ is the smallest prefix closed set of strings containing $w$. We need the following claim. 

\begin{myclaim}
\label{claim-nonempty-inter}
For every different pair of states $q,q'$ in $Q_{(s_0),\df_1)}$, the sets $\sigma(q)$ and $\sigma(q')$ are disjoint. 
\end{myclaim}

\begin{proof}
Assume for contradiction two different  states $q,q'$ in $Q_{(s_0),\df_1)}$ such that both $\sigma(q)$ and $\sigma(q')$ contain a state $q^* \in Q_\df$. By definition, there is a word 
$w$ such that $q^*$ is the closure (over prefixes) of $w$. That word originated from sequences $\pi = q_0,\dots,q_n$ (for $q$) and $\pi' = q_0',\dots,q_n'$ (for $q'$). If 
$\pi = \pi'$ then $q = q'$, so $\pi$ must be different from $\pi'$. Let $i$ be the first position where $\pi$ and $\pi'$ are different, so that 
sequences $q_0,\dots,q_i$ and $q_0,\dots,q_i'$ are the same except for the last state. Then both $q_i$ and $q_i'$ are reachable from $q_{i-1}$ using exactly one step; it follows 
by the construction of our game that one of $q_i$ and $q_i'$ is the result of applying action $\df_1(q_{i-1})$ and the other is the result of applying $s_0(q_{i-1})$, which implies that 
the word generated from $\pi$ and $\pi'$ is not the same. 
\end{proof}

Recall the utility of player $0$ using $\df$ at the genesis tree is defined as 
\begin{eqnarray*}
u_0(\df \mid \varepsilon) & = & \sum_{q \in \bQ} \beta^{|q|} \cdot  r_0(q) \cdot \pr^{\df}(q \mid \varepsilon)
\end{eqnarray*}

If we choose to sum only over states in the image of some $q^* \in Q_{(s_0),\df_1)}$ under $\sigma$, we now have: 
\begin{eqnarray*}
u_0(\df \mid \varepsilon) & \geq & \sum_{q \in \sigma(q^*) \,\mid\, q^* \in Q_{(s_0,\df_1)}} \beta^{|q|} \cdot  r_0(q) \cdot \pr^{\df}(q \mid \varepsilon),
\end{eqnarray*}

because Claim \ref{claim-nonempty-inter} guarantees that we are not summing each state in $Q_\df$ more than once. We rearrange the term in the right: 

\begin{eqnarray*}
u_0(\df \mid \varepsilon) & \geq &\sum_{q^* \in Q_{(s_0,\df_1)}}   \sum_{q \in \sigma(q^*)} \beta^{|q|} \cdot  r_0(q) \cdot \pr^{\df}(q \mid \varepsilon)
\end{eqnarray*}

For each state $q^* \in Q_{(s_0,\df_1)}$, note that 

$$\sum_{q \in \sigma(q^*)} \beta^{|q|} \cdot  r_0(q) \cdot \pr^{\df}(q \mid \varepsilon) \geq \sum_{q \in \sigma(q^*)} \beta^{|q^*|} \cdot  r_0(q^*) \cdot \pr^{\df}(q \mid \varepsilon) = \beta^{|q^*|} \cdot  r_0(q^*) \sum_{q \in \sigma(q^*)}  \pr^{\df}(q \mid \varepsilon), $$ because $|q| = |q^*|$ and 
$q$ and $q^*$ have the same number of blocks owned by $0$, and by definition,
$$\sum_{q \in \sigma(q^*)}  \pr^{\df}(q \mid \varepsilon) = \pr^{(s_0,\df_1)}(q^* \mid \varepsilon).$$ 
Summing up and rearranging, we have: 

\begin{eqnarray*}
u_0(\df \mid \varepsilon) & \geq &\sum_{q^* \in Q_{(s_0,\df_1)}}  \beta^{|q^*|} \cdot  r_0(q^*) \cdot \pr^{(s_0,\df_1)}(q^* \mid \varepsilon) = u_0((s_0,\df_1) \mid \varepsilon), 
\end{eqnarray*}

which was to be shown.
\end{proof}


While constant block rewards do not faithfully model reality, since in the Bitcoin protocol the reward decreases every 200.000 blocks or so, we would like to argue why Theorem \ref{thm-conts_equlibria} could serve as a good recommendation on how to enforce good behaviour on miners at the moment block rewards become insignificant. More precisely, if block rewards are negligible, the transaction fees will dictate the miners' pay-off, so the protocol could place a (constant) total fee limit on newly created blocks. Assuming that the volume of transactions is high, the blocks would regularly achieve the maximal reward, thus making the block reward constant. Theorem \ref{thm-conts_equlibria} then tells the miners that their best strategy is to mine on top of the existing blockchain, as this will maximize their utility in the long run.


The remainder of this section is devoted to explaining the proof of  Theorem \ref{thm-conts_equlibria}.


\subsection{Greedy strategies and proof of Theorem \ref{thm-conts_equlibria}} 

We begin by showing that $\df$ is an equilibrium when we slightly restrict the space of strategies that the player use, and concentrate on the so called {\em greedy} strategies. Intuitively, under greedy strategies, the players refrain from forking on top of blocks that appear before their latest block when there is no blockchain, or their latest block in the blockchain, when the latter is defined. Greedy strategies can be formally defined as follows. 
Given a player $p \in \bP$ and a state $q \in \bQ$, let:
\begin{multline*}
\longest(q,p) \ = \ \{ b \in q \mid (b = \varepsilon \text{ or } \owner(b) = p),\\
\text{ and for every } b' \in q \text{ such that } \owner(b') = p : |b'| \leq |b|\}
\end{multline*}
Notice that $\varepsilon \in \longest(p,q)$ if and only if there is no $b \in q$ such that $\owner(b) = p$. Moreover, define $\length(q,p)$ as the length of an arbitrary string in $\longest(q,p)$ (all of them have the same length).
\begin{mydef}\label{def-greedy}
Given $p \in \bP$, $b \in \bB$ and $q \in \bQ$,  an action $\mine(p,b,q)$ is {\em greedy} if $\mine(p,b,q)$ is a valid action and $\length(q,p) \leq |b|$.

Moreover, a combined strategy $\bs = (s_0, s_1, \ldots, s_{m-1})$ is {\em greedy} if for every $p \in \bP$ and  $q \in \bQ$ such that $\pr^{\bs}(q \mid \varepsilon) > 0$, it holds that $s_p(q)$ is a greedy action.
\end{mydef}

For now, we only consider greedy strategies. 
%
%Under greedy strategies, players refrain to fork on top of blocks that appear before their latest block, or their latest block in the blockchain. 
The consequence of this is that every state in an $m$-player game under greedy strategies cannot have more than $m$ paths contesting for the blockchain. More precisely:% (see Lemma \ref{lem-length-greedy} in Appendix \ref{sec-char-states-greedy}). %This is captured b the following technical lemma: 
\begin{mylem}\label{lem-length-greedy}
Let $\bs$ be a greedy strategy. Then for every $q \in \bQ$ such that $\pr^{\bs}(q \mid \varepsilon) > 0$, the following conditions hold:
\begin{enumerate}
\item For every $p \in \bP$ $:$ $|\longest(q,p)| = 1$ 

\item There exists $I \subseteq \bP$ such that$:$
\begin{eqnarray}\label{eq-max-set}
\longest(q) & = & \bigcup_{p \in I} \longest(q,p).
\end{eqnarray}
Moreover, if $q \neq \{\varepsilon\}$, then there exists a unique $I \subseteq \longest(q,p)$ such that \eqref{eq-max-set} holds.
\end{enumerate}
\end{mylem}

The key property of greedy strategies needed to show that $\df$ is an equilibrium, is the fact that if two strategies are optimal for a player $p$, then they can not differentiate two states $q$ and $q'$ in which the subtree rooted at $\longest(q,p)$ and $\longest(q',p)$, respectively, are isomorphic. A strategy $s$ for a player $p$ is called a {\em basic strategy}, if $s(q)=s(q')$, whenever the subtree of $q$ rooted at $\longest(q,p)$ is isomorphic to the subtree of $q'$ rooted at $\longest(q',p)$. We can show that for greedy strategies the following holds:

\begin{mylem}
\label{lem-meet}
Consider a game with $m$ players and let $s_p$ be a greedy strategy for player $p$. Then there is a basic strategy $s'_p$ such that $u_p((s_{-p},s'_p) \mid \varepsilon) \geq u_p((s_{-p},s_p) \mid \varepsilon)$ for any set $s_{-p}$ of basic greedy strategies.  
\end{mylem}


With this lemma at hand, we can now show that $\df$ is indeed a stationary equilibrium when we are considering only greedy strategies.

\begin{mythm}%\label{thm-conts_equlibria}
For any $0 \leq \beta \leq 1$, the strategy $\df$ is a $\beta$-discounted stationary equilibrium under greedy strategies. 
\end{mythm} 

\etienne{In order for the theorem to be true we have to considere stable DF strategy and not whatever df strategy. I think the easiest way to add this constraint without too much work is directly in the definition of greedy strategy ! A greedy action is ok, but to be a greedy strategy you also have to be stable.}
EXPLAIN SOME BASIC IDEAS BEHIND THE PROOF.

Having established that $\df$ is an equilibrium under greedy strategies, we will now show that this restriction is not necessary, as any non greedy strategy can be replaced by a greedy one in an equilibrium. That is, we can show the following:

LEMMA REUTTER-TOUSSAINT

EXPLAIN WHY THE LEMMA SHOW THAT DF IS GREAT.


%We conclude this section by some remarks on the potential significance of Theorem \ref{thm-conts_equlibria}. As we have already mentioned, constant block rewards do not faithfully model reality, since in the Bitcoin protocol the reward decreases every 200.000 blocks or so. However, we would like to argue that Theorem \ref{thm-conts_equlibria} can serve as a good recommendation on how to enforce good behaviour on miners (assuming they will use the utility function as an indicator of their monetary gain), at the moment block rewards become insignificant. More precisely, if block rewards are insignificant, and the transaction fees dictate the miners' pay-off, the protocol could place a (constant) fee limit on newly created blocks. Assuming that the volume of transactions is high, the blocks would regularly achieve the maximal reward, thus making the block reward constant. Theorem \ref{thm-conts_equlibria} then tells the miners that their best strategy is to mine on top of the existing blockchain, as this will maximize their utility in the long run.


%We already mentioned that constant block rewards do not faithfully model reality, since in the Bitcoin protocol the reward decreases every 200.000 blocks or so. However, we would like to argue that Theorem \ref{thm-conts_equlibria} can serve as a good recommendation on how to enforce good behaviour on miners (assuming they will use the utility function as an indicator of their monetary gain), at the moment block rewards become insignificant. More precisely, if block rewards are insignificant, and the transaction fees dictate the miners' pay-off, the protocol could place a (constant) fee limit on newly created blocks. Assuming that the volume of transactions is high, the blocks would regularly achieve the maximal reward, thus making the block reward constant. Theorem \ref{thm-conts_equlibria} then tells the miners that their best strategy is to mine on top of the existing blockchain, as this will maximize their utility in the long run.