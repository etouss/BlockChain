%!TEX root = focs.tex

\section{Decreasing Payoff}
\label{sec-dec}

While interesting, readers could argue that the payoff function considered before is not completely modelling bitcoin, because in bitcoin 
the reward given for mining blocks is reduced by half every 200.000 blocks or so. Thus the natural question, do the results above continue to hold under these 
circumstances? Our answer is a resounding no, and now forking can be a valid strategy depending on the hash power of a player (and the other parameters of the game). 

For simplicity, we model the diminishing rewards in bitcoin as a constant factor that is lowered after every new block in the blockchain. That is, in this section 
we use the following pay-off function $r_p$ for all players $p \in \bP$, denoted as the \textbf{$\alpha$-discounted reward}: 
\begin{eqnarray*}
r_p(q) & = & 
{\displaystyle c \cdot \sum_{i=1}^{|\meet(q)|} \alpha^i \cdot \chi_p(\meet(q),i)} \end{eqnarray*}
where $c$ is a positive real number and $\alpha \in (0,1]$.

\medskip

(say that the approach we use is the same as above: we consider a two player game and group a bunch of well behaved players 
into a single player. Also state that the subtree lemma holds also for this payoff)

\subsection{When is forking a good strategy}
\label{sec-forkingstrategies}
(probably say this is a very interesting open question in the literature). 

In order to understand when forking could be a good strategy, let us consider again the two player game in which player $p$ has just mined a block in the blockchain. 
Now, if player $(1-p)$ wins the next block, player $p$ faces two different options. She could continue mining on the blockchain, playing according to $\df$, or she 
could fork, mining instead upon her previously won block. Which strategy is better? To answer this question, we need to calculate the utility in both cases. For 
simplicity, and without loss of generality (due to Lemma [?]), we will focus on this question when the game is just starting, in the genesis block, and we will assume that 
player $p$ resumes the default strategy (mining upon the blockchain) as soon as she wins the fork. 

More precisely, we will calculate the utility of the combined strategy $(\df_{1-p},\fg_p)$, where $\fg_p$ is the \textbf{Fork on the genesis} 
strategy, defined as follows for a state $q \in \bQ$:
%\begin{eqnarray*}
%\fg_p(q) & = &
%\begin{cases}
%\mine(p,\varepsilon,q) & \text{if } p \not\in q\\
%\mine(p,\bchain(q),q) &  \text{if } \bchain(q) \text{ is defined and } p \preceq \bchain(q)\\
%\mine(p,b,q) &  \text{if } \bchain(q) \text{ is defined},\  p \in q,\ p \not\preceq \bchain(q)
%\text{ and}\\
%&  \hspace{50pt} {\displaystyle b = \max_{\preceq} \, \{ b' \in q \mid \longest(q,p) \preceq b'\}}\\
%\mine(p,\longest(q,p),q) &  \text{if } \bchain(q) \text{ is not defined},\ p \in q \text{ and}\\
%& \hspace{50pt} p \not\preceq \longest(q,1-p)\\
%\mine(p,\longest(q,p),q) &  \text{if } \bchain(q) \text{ is not defined},\ p \preceq \longest(q,1-p) \text{ and}\\
%& \hspace{50pt} r_p(\longest(q,p)) \geq r_p(\longest(q,1-p))\\
%\mine(p,\longest(q,1-p),q) &  \text{if } \bchain(q) \text{ is not defined},\  p \preceq \longest(q,1-p) \text{ and}\\
%& \hspace{50pt} r_p(\longest(q,p)) < r_p(\longest(q,1-p))
%\end{cases}
%\end{eqnarray*}
\begin{eqnarray*}
\fg_p(q) & = &
\begin{cases}
\mine(p,\varepsilon,q) & \text{if } p \not\in q\\
\df_p(q) & \text{if } \bchain(q) \text{ is defined and } p \preceq \bchain(q) \text{ or}\\
& \hspace{70pt} \bchain(q) \text{ is not defined and } p \preceq \longest(q,1-p)\\
\mine(p,b,q) &  \text{otherwise, where } {\displaystyle b = \max_{\preceq} \, \{ b' \in q \mid \longest(q,p) \preceq b'\}}
\end{cases}
\end{eqnarray*}
Notice that in this definition, the set $\{ b' \in q \mid \longest(q,p) \preceq b'\}$ has a maximum element under the partial order $\preceq$ as all the elements in this set are of the form $\longest(q,p) \cdot (1-p)^\ell$ with $\ell \geq 0$.

(show how to compute the utility, put graphics, compare. 

\subsection{More complicated cases}

Speak of $m$-fork as a refinement of the previous strateegy, put give-up time if we have it as well. Graphics, compare, etc. 

\subsection{Stationary equilibrium}

Need to fill this up! 
