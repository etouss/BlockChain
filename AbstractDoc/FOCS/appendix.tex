%!TEX root = focs.tex

\section{Additional Material and Proofs}
\label{sec-appendix}

\subsection{Combined strategies in Section \ref{sec-dec}}
In section \ref{sec-dec}, we calculate the utility of the following combined strategies.
\begin{itemize}
\item {\bf Fork on the genesis.} In this case, we consider the combined strategy $\fg = (\fg_0, \df_1$, $\ldots$, $\df_{m-1})$, where $\fg_0$ is defined as follows for a state $q \in \bQ$:
\begin{eqnarray*}
\fg_0(q) & = &
\begin{cases}
\mine(0,\varepsilon,q) & \text{if } 0 \not\in q\\
\df_0(q) & \text{if } \bchain(q) \text{ is defined and } 0 \preceq \bchain(q)\\
\mine(0,\longest(q,0),q) &  \text{otherwise}
\end{cases}
\end{eqnarray*}

\item {\bf Fork on the genesis with give up time.} Given a natural number $k \geq 1$, in this case, we consider the combined strategy $\fg_k = (\fg_{0,k}, \df_1$, $\ldots$, $\df_{m-1})$, where $\fg_{0,k}$ is defined as follows for a state $q \in \bQ$:
\begin{eqnarray*}
\fg_{0,k}(q) & = &
\begin{cases}
\mine(0,\varepsilon,q) & \text{if } 0 \not\in q,\ \bchain(q) \text{ is defined and } |\bchain(q)| \leq k\\
\mine(0,b,q) & \text{if } 0 \not\in q,\ \bchain(q) \text{ is defined and } |\bchain(q)| > k\\
\df_0(q) & \text{if } \bchain(q) \text{ is defined and } 0 \preceq \bchain(q)\\
\mine(0,\longest(q,0),q) &  \text{otherwise}
\end{cases}
\end{eqnarray*}

\begin{eqnarray*}
\fg_{0,k}(q) & = &
\begin{cases}
\mine(p,\longest(q,1-p),q) & \text{if } \length(q,1-p) - \length(q,p) > k\\
\fg_p(q) &   \text{otherwise}
\end{cases}
\end{eqnarray*}
\end{itemize}


\subsection{Proof of Lemma \ref{lem-length-greedy}}
Let $S = \{ q \in \bQ \mid \pr^{\bs}(q \mid \varepsilon) > 0 \}$. Then we have that $S$ is the smallest subset of $\bQ$ satisfying the following conditions:
\begin{itemize}
\item $\{\varepsilon\} \in S$.

\item If $p \in \bP$, $b \in \bB$, $q \in S$ and $\mine(p,b,q)$ is a greedy action, then $q \cup \{ b \cdot p\} \in S$.
\end{itemize}
Hence, we have an inductive definition of $S$, and we can prove the lemma by induction on the structure of this set of states. If $q = \{ \varepsilon \}$, then we have that $\longest(q) = \{ \varepsilon \}$ and $\longest(q,p) = \{ \varepsilon \}$ for every $p \in \bP$. Thus, we have that  the two conditions in the lemma hold since $|\longest(q,p)| = 1$ for every $p \in \bP$, and $\longest(q) = \longest(q,0)$. Assume that the property holds for $q \in S$, and assume that $p \in \bP$, $b \in \bB$ and $\mine(p,b,q)$ is a greedy action. Then we need to prove that the conditions in the lemma hold for $q ' = q \cup \{b \cdot p \}$.

Given that $\mine(p,b,q)$ is a greedy action, we know that $\length(q,p) \leq |b| < |b \cdot p|$. Thus, we have that $\longest(q',p) = \{b \cdot p\}$ and, hence, $|\longest(q',p)| = 1$. Moreover, we have that for every $p' \in \bP$ such that $p' \neq p$, it holds that $\longest(q',p') = \longest(q,p')$ and, hence, $|\longest(q',p')| = 1$. Therefore, the first condition of the lemma is satisfied. To prove that there exists $I \subseteq \bP$ such that \eqref{eq-max-set} holds, we consider two cases.
\begin{itemize}
\item Assume that $b \in \longest(q)$. Then we have that $\longest(q') = \{ b \cdot p \}$ and, hence, there exists $I \subseteq \bP$ such that \eqref{eq-max-set} holds since $\longest(q') = \longest(q',p)$.

\item Assume that $b \not\in \longest(q)$. By induction hypothesis we know that:
\begin{eqnarray*}
\longest(q) & = & \bigcup_{p' \in I} \longest(q,p').
\end{eqnarray*}
Given that $b \not\in \longest(q)$, we have that $p \not\in I$. Therefore,  for every $p' \in I$ we have that $\longest(q,p') = \longest(q',p')$. Hence, we deduce that:
\begin{eqnarray}\label{eq-lem-length-greedy}
\longest(q) & = & \bigcup_{p' \in I} \longest(q',p').
\end{eqnarray}
Let $k$ be the length of an arbitrary element in $\longest(q)$ (all the elements of this set have the same length). If $|p \cdot b| < k$, then $\longest(q') = \longest(q)$. Thus, from \eqref{eq-lem-length-greedy} we conclude that there exists $I \subseteq \bP$ such that \eqref{eq-max-set} holds since:
\begin{eqnarray*}
\longest(q') & = & \bigcup_{p' \in I} \longest(q',p').
\end{eqnarray*}
Assume now that $|p \cdot b| = k$. In this case, we have that $\longest(q') = \longest(q) \cup \{p \cdot b\}$. Thus,  from \eqref{eq-lem-length-greedy} and the fact that $\longest(q',p) = \{b \cdot p\}$, we conclude that:
\begin{eqnarray*}
\longest(q') & = & \bigg(\bigcup_{p' \in I} \longest(q',p')\bigg) \cup \longest(q',p)\\
& = & \bigcup_{p' \in I \cup \{p\}} \longest(q',p').
\end{eqnarray*}
Hence, in the case that $|p \cdot b| = k$, there also exists $I \subseteq \bP$ such that \eqref{eq-max-set} holds.
\end{itemize}
To conclude the proof of the lemma, we assume that $q' \neq \{\varepsilon\}$ and prove that there exists a unique $I \subseteq \bP$ such that \eqref{eq-max-set} holds. For the sake of contradiction, assume that there exist distinct $I_1, I_2 \subseteq \bP$ such that 
$\longest(q') = \bigcup_{p \in I_1} \longest(q',p) = \bigcup_{p \in I_2} \longest(q',p)$. 
Without loss of generality, assume that there exists $p_1 \in I_1$ such that $p_1 \not\in I_2$. Given that $q' \neq \{\varepsilon\}$, we have that $\varepsilon \not\in \longest(q')$. 
Let $b \in \longest(q',p_1)$. Given that $p_1 \in I_1$, we have that $b \neq \varepsilon$, from which we conclude that $\owner(b) = p_1$. Given that 
$ \bigcup_{p \in I_1} \longest(q',p) = \bigcup_{p \in I_2} \longest(q',p)$ and $p_1 \not\in I_2$, we conclude that there exists $p_2 \neq p_1$ such that $p_2 \in I_2$ and $b \in \longest(q',p_2)$, which leads to a contradiction since $b \neq \varepsilon$ and $\owner(b) \neq p_2$. This concludes the proof of the lemma.




%\begin{mylem}\label{lem-dist-1}
%Given $p \in \bP$, a combined strategy $\bs$ and $\bq_0 \in \bQ$, it holds that:
%\begin{eqnarray*}
%u_p(\bs \mid \bq_0) & = & r_p(\bq_0) +  \beta \cdot 
%\bigg(\sum_{\substack{\bq \in \bQ \,: \\ \bq_0 \subseteq \bq \text{ {\rm and} } |\bq| - |\bq_0| = 1}}
%\pr(\bq \mid \bq_0) \cdot u_p(\bs \mid \bq)\bigg).
%\end{eqnarray*}
%\end{mylem}
%
%\begin{proof} We have that:
%\begin{align*}
%u_p(\bs \mid & \ \bq_0)  \ =  \\
%&\sum_{i=0}^{\infty}\beta^{i} \cdot  \bigg(\sum_{\substack{\bq \in \bQ \,: \\ \bq_0 \subseteq \bq \text{ {\rm and} } |\bq| - |\bq_0| = i}} r_p(\bq) \cdot 
%\pr^{\bs}(\bq \mid \bq_0)\bigg) \ =\\
%& r_p(\bq_0) + \sum_{i=1}^{\infty}\beta^{i} \cdot  \bigg(\sum_{\substack{\bq \in \bQ \,: \\ \bq_0 \subseteq \bq \text{ {\rm and} } |\bq| - |\bq_0| = i}} r_p(\bq) \cdot 
%\pr^{\bs}(\bq \mid \bq_0)\bigg)\ = \\
%& r_p(\bq_0) + \sum_{i=1}^{\infty}\beta^{i} \cdot  \bigg(
%\sum_{\substack{\bq \in \bQ \,: \\ \bq_0 \subseteq \bq \text{ {\rm and} } |\bq| - |\bq_0| = 1}} \pr^{\bs}(\bq \mid \bq_0) \cdot
%\bigg(\sum_{\substack{\bq' \in \bQ \,: \\ \bq \subseteq \bq' \text{ {\rm and} } |\bq'| - |\bq| = i-1}} r_p(\bq) \cdot 
%\pr^{\bs}(\bq' \mid \bq)\bigg)\bigg)\ = \\
%& r_p(\bq_0) + \beta \cdot \sum_{i=1}^{\infty}\beta^{i-1} \cdot  \bigg(
%\sum_{\substack{\bq \in \bQ \,: \\ \bq_0 \subseteq \bq \text{ {\rm and} } |\bq| - |\bq_0| = 1}} \pr^{\bs}(\bq \mid \bq_0) \cdot
%\bigg(\sum_{\substack{\bq' \in \bQ \,: \\ \bq \subseteq \bq' \text{ {\rm and} } |\bq'| - |\bq| = i-1}} r_p(\bq) \cdot 
%\pr^{\bs}(\bq' \mid \bq)\bigg)\bigg)\ = \\
%& r_p(\bq_0) + \beta \cdot \sum_{j=0}^{\infty}\beta^{j} \cdot  \bigg(
%\sum_{\substack{\bq \in \bQ \,: \\ \bq_0 \subseteq \bq \text{ {\rm and} } |\bq| - |\bq_0| = 1}} \pr^{\bs}(\bq \mid \bq_0) \cdot
%\bigg(\sum_{\substack{\bq' \in \bQ \,: \\ \bq \subseteq \bq' \text{ {\rm and} } |\bq'| - |\bq| = j}} r_p(\bq) \cdot 
%\pr^{\bs}(\bq' \mid \bq)\bigg)\bigg)\ = \\
%& r_p(\bq_0) + 
% \beta \cdot \bigg(\sum_{\substack{\bq \in \bQ \,: \\ \bq_0 \subseteq \bq \text{ {\rm and} } |\bq| - |\bq_0| = 1}} \pr^{\bs}(\bq \mid \bq_0) \cdot \bigg(\sum_{j=0}^{\infty}\beta^{j} \cdot  \bigg(\sum_{\substack{\bq' \in \bQ \,: \\ \bq \subseteq \bq' \text{ {\rm and} } |\bq'| - |\bq| = j}} r_p(\bq) \cdot 
%\pr^{\bs}(\bq' \mid \bq)\bigg)\bigg)\bigg)\ = \\
%& r_p(\bq_0) +  \beta \cdot 
%\bigg(\sum_{\substack{\bq \in \bQ \,: \\ \bq_0 \subseteq \bq \text{ {\rm and} } |\bq| - |\bq_0| = 1}}
%\pr(\bq \mid \bq_0) \cdot u_p(\bs \mid \bq)\bigg).
%\end{align*}
%\end{proof}


\begin{mylem}\label{lem-dist-k}
Given $p \in \bP$, a combined strategy $\bs$, $\bq_0 \in \bQ$ and $k \geq 1$, it holds that:
\begin{multline*}
u_p(\bs \mid \bq_0) \ = \  
\sum_{i=0}^{k-1} \beta^i \cdot \bigg(\sum_{\substack{\bq \in \bQ \,: \\ \bq_0 \subseteq \bq \text{ {\rm and} } |\bq| - |\bq_0| = i}}
r_p(\bq) \cdot \pr^{\bs}(\bq \mid \bq_0)\bigg) \ + \\
\beta^k \cdot 
\bigg(\sum_{\substack{\bq \in \bQ \,: \\ \bq_0 \subseteq \bq \text{ {\rm and} } |\bq| - |\bq_0| = k}}
\pr^{\bs}(\bq \mid \bq_0) \cdot u_p(\bs \mid \bq)\bigg).
\end{multline*}
\end{mylem}

\begin{proof} We have that:
\begin{align*}
u_p(\bs \mid & \ \bq_0)  \ =  \\
&\sum_{i=0}^{\infty}\beta^{i} \cdot  \bigg(\sum_{\substack{\bq \in \bQ \,: \\ \bq_0 \subseteq \bq \text{ {\rm and} } |\bq| - |\bq_0| = i}} r_p(\bq) \cdot 
\pr^{\bs}(\bq \mid \bq_0)\bigg) \ =\\
&\sum_{i=0}^{k-1} \beta^i \cdot \bigg(\sum_{\substack{\bq \in \bQ \,: \\ \bq_0 \subseteq \bq \text{ {\rm and} } |\bq| - |\bq_0| = i}}
r_p(\bq) \cdot \pr^{\bs}(\bq \mid \bq_0)\bigg) + 
\sum_{i=k}^{\infty}\beta^{i} \cdot  \bigg(\sum_{\substack{\bq \in \bQ \,: \\ \bq_0 \subseteq \bq \text{ {\rm and} } |\bq| - |\bq_0| = i}} r_p(\bq) \cdot 
\pr^{\bs}(\bq \mid \bq_0)\bigg)\ = \\
&  \sum_{i=0}^{k-1} \beta^i \cdot \bigg(\sum_{\substack{\bq \in \bQ \,: \\ \bq_0 \subseteq \bq \text{ {\rm and} } |\bq| - |\bq_0| = i}} r_p(\bq) \cdot \pr^{\bs}(\bq \mid \bq_0)\bigg) \ + \\
& \hspace{15pt} \sum_{i=k}^{\infty}\beta^{i} \cdot  \bigg(
\sum_{\substack{\bq \in \bQ \,: \\ \bq_0 \subseteq \bq \text{ {\rm and} } |\bq| - |\bq_0| = k}} \pr^{\bs}(\bq \mid \bq_0) \cdot
\bigg(\sum_{\substack{\bq' \in \bQ \,: \\ \bq \subseteq \bq' \text{ {\rm and} } |\bq'| - |\bq| = i-k}} r_p(\bq') \cdot 
\pr^{\bs}(\bq' \mid \bq)\bigg)\bigg)\ = \\
& \sum_{i=0}^{k-1} \beta^i \cdot \bigg(\sum_{\substack{\bq \in \bQ \,: \\ \bq_0 \subseteq \bq \text{ {\rm and} } |\bq| - |\bq_0| = i}} r_p(\bq) \cdot \pr^{\bs}(\bq \mid \bq_0)\bigg) \ + \\
& \hspace{15pt}
\beta^k \cdot \sum_{i=k}^{\infty}\beta^{i-k} \cdot  \bigg(
\sum_{\substack{\bq \in \bQ \,: \\ \bq_0 \subseteq \bq \text{ {\rm and} } |\bq| - |\bq_0| = k}} \pr^{\bs}(\bq \mid \bq_0) \cdot
\bigg(\sum_{\substack{\bq' \in \bQ \,: \\ \bq \subseteq \bq' \text{ {\rm and} } |\bq'| - |\bq| = i-k}} r_p(\bq') \cdot 
\pr^{\bs}(\bq' \mid \bq)\bigg)\bigg)\ = \\
& \sum_{i=0}^{k-1} \beta^i \cdot \bigg(\sum_{\substack{\bq \in \bQ \,: \\ \bq_0 \subseteq \bq \text{ {\rm and} } |\bq| - |\bq_0| = i}} r_p(\bq) \cdot \pr^{\bs}(\bq \mid \bq_0)\bigg) \ + \\
& \hspace{15pt} \beta^k \cdot \sum_{j=0}^{\infty}\beta^{j} \cdot  \bigg(
\sum_{\substack{\bq \in \bQ \,: \\ \bq_0 \subseteq \bq \text{ {\rm and} } |\bq| - |\bq_0| = k}} \pr^{\bs}(\bq \mid \bq_0) \cdot
\bigg(\sum_{\substack{\bq' \in \bQ \,: \\ \bq \subseteq \bq' \text{ {\rm and} } |\bq'| - |\bq| = j}} r_p(\bq') \cdot 
\pr^{\bs}(\bq' \mid \bq)\bigg)\bigg)\ = \\
& \sum_{i=0}^{k-1} \beta^i \cdot \bigg(\sum_{\substack{\bq \in \bQ \,: \\ \bq_0 \subseteq \bq \text{ {\rm and} } |\bq| - |\bq_0| = i}} r_p(\bq) \cdot \pr^{\bs}(\bq \mid \bq_0)\bigg) \ + \\
& \hspace{15pt}
 \beta^k \cdot \bigg(\sum_{\substack{\bq \in \bQ \,: \\ \bq_0 \subseteq \bq \text{ {\rm and} } |\bq| - |\bq_0| = k}} \pr^{\bs}(\bq \mid \bq_0) \cdot \bigg(\sum_{j=0}^{\infty}\beta^{j} \cdot  \bigg(\sum_{\substack{\bq' \in \bQ \,: \\ \bq \subseteq \bq' \text{ {\rm and} } |\bq'| - |\bq| = j}} r_p(\bq') \cdot 
\pr^{\bs}(\bq' \mid \bq)\bigg)\bigg)\bigg)\ = \\
& \sum_{i=0}^{k-1} \beta^i \cdot \bigg(\sum_{\substack{\bq \in \bQ \,: \\ \bq_0 \subseteq \bq \text{ {\rm and} } |\bq| - |\bq_0| = i}} r_p(\bq) \cdot \pr^{\bs}(\bq \mid \bq_0)\bigg) +  \beta^k \cdot 
\bigg(\sum_{\substack{\bq \in \bQ \,: \\ \bq_0 \subseteq \bq \text{ {\rm and} } |\bq| - |\bq_0| = k}}
\pr^{\bs}(\bq \mid \bq_0) \cdot u_p(\bs \mid \bq)\bigg).
\end{align*}
\end{proof}

\section{Trapezoidal Dyck paths}
\label{appendix-trapezoid}

For nonnegative integers $a,b$, let $\mathcal L_{a,b}$ be the trapezoid in $\ZZ^2$ whose vertices are
$\{(0,0),(a,0),(a+b,b),(0,b)\}.$ Also, let $\mathcal D_{a,b}$ the set of one step north-east paths from $(0,0)$ to $(a+b,b)$ that stay inside $\mathcal L_{a,b}$, and $\sigma_{a,b}=\# \mathcal D_{a,b}$ the amount of possible paths. Finally, let $C_n$ denote the $n$-th Catalan number. 
\begin{myprop*}
The sequence $\sigma:\NN^2\to \NN$ verifies the following
$$\begin{cases}
\sigma_{a,b}=\sum_{i=0}^b\sigma_{a-1,i}C_{b-i}& \mbox{ for }a\geq 1,b\in \NN,\\
\sigma_{x,0}=1 & \mbox{ for }x\in \NN,\\
\sigma_{0,y}=C_y & \mbox{ for }y\in \NN.
\end{cases}$$
\end{myprop*}
\begin{proof}
To prove this, we write $\mathcal D_{a,b}$ as a union of disjoint sets. As a first remark, note that every path in $\mathcal D_{a,b}$ touches the line $l=\overline{(a,0)(a+b,b)}$ at least once. For $i\in \{0,\dots,b\}$, let $P_i$ be the point $(a+i,i)\in l$ and $\mathcal D_{a,b}^{(i)}$ all paths in $\mathcal D_{a,b}$ that touch the line $l$ for the first time at $P_i$. We have the disjoint union
$$\mathcal D_{a,b}=\bigcup_{i=0}^b \mathcal D_{a,b}^{(i)}.$$
Also, note that a path in $\mathcal D_{a,b}$ touches $l$ for the first time in $P_i$ if and only if it passed through the point $P_i-(1,0)$ without exiting $\mathcal D_{a-1,i}$, followed by an east step and any path from $P_i$ to $(a+b,b)$. This yields
\begin{eqnarray*}
\sigma_{a,b}=\sum_{i=0}^{b}\mbox{(paths from $(0,0)$ to $P_i-(0,1))$}\cdot \mbox{(paths from $P_i$ to $(a+b,b))$}
\end{eqnarray*}
Note that the amount of paths from $P_i$ to $(a+b,b)$ is $C_{b-i}$, since both points belong to $l$, proving that $\sigma_{a,b}$ verifies the recurrence equation. The border cases $\sigma_{0,\cdot},\sigma_{\cdot,0}$ are straightforward to prove.
\end{proof}

Now let us define the sequence of generating functions with respect to the second variable of $\sigma_{\cdot,\cdot}$ as follows:
$$\begin{array}{cl}
\phi_a:&\RR\to \RR\\
       &x\mapsto \displaystyle \sum_{j=0}^{\infty}\sigma_{a,j}x^j
\end{array}$$
 
\begin{myprop*}
For $x\in [-1/4,1/4]$ and $a\in \NN$ 
$$\phi_a(x)=c(x)^{a+1},$$
where $c(x):=\frac{1-\sqrt{1-4x}}{2x}$ is the generating function of the Catalan numbers.
\end{myprop*}
\begin{proof}
We have $\sigma_{a,\cdot}=\sigma_{a-1,\cdot}\star C_\cdot$, where $\star$ is the convolution operator, therefore $\phi_a(x)=\phi_{a-1}(x)c(x)$. The result follows noting that $\phi_0(x)=c(x)$.
\end{proof}

\begin{myprop*}
For $(a,b)\in\NN^2$,
$$\sigma_{a,b}=\frac{(a+b)(a+2b)!}{b!(a+b+1)!}=\frac{a+1}{a+b+1}{a+2b\choose a+b}.$$
\end{myprop*}
\begin{proof}
Extract the sequence from the Taylor series of $\phi_a(x)$ around 0 and prove by induction.
\end{proof}

\francisco{@Juan: It would be nice to include your reflection argument as a second (or first!) proof.}




