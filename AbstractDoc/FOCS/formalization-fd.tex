%!TEX root = focs.tex

\section{A Game-theoretic Characterization of Bitcoin Mining}
\label{sec-formalization}
The mining game is played by a set $\bP = \{0, 1, , \ldots, m-1\}$ of players, with $m \geq 2$. In this game, each player has some reward depending on the number of blocks she owns. Blocks are placed one on top of another, starting from an initial block called the {\em genesis block}. Thus, the game defines a tree of blocks. Each block is put by one player, called the {\em owner} of this block. Each such tree is called a {\em state of the game}, or just {\em state}, and it represents the knowledge that each player has about the blocks that have been mined thus far.

Formally, in a game played by $m$ players a block is a string $b$ over the alphabet $\{0,1,\ldots, m-1\}$. We denote by $\bB$ the set of all blocks, that is, $\bB = \{0,1,\ldots , m-1\}^*$. Each block apart from $\varepsilon$ has a unique owner, defined by the function $\owner: (\bB \backslash \{\varepsilon\}) \rightarrow \{0,1, \ldots ,m-1\}$ such that $\owner(b)$ is equal to the last symbol of $b$. A state of the game, or just state,  is a finite and non-empty set of blocks $q \subseteq \bB$ that is prefix closed. That is, $q$ is a set of strings over the alphabet $\{0,1,\ldots, m-1\}$ such that if $b\in q$, then every prefix of $b$ (including the empty word $\varepsilon$) also belongs to $q$. Note that a prefix closed subset of $\bB$ uniquely defines a tree with the root $\varepsilon$. 
%
The intuition here is that each element of $q$ corresponds to a block that was put into the state $q$ by some player. The genesis block corresponds to $\varepsilon$. When a player $p$ decides to mine on top of a block $b$, she puts another block into the state defined by the string $b\cdot p$.
%
Let $\bQ$ be the set of all possible states in a game played by $m$ players, and for a state $q \in \bQ$, let $|q|$ be its size, measured as the cardinality of the set $q$. 

Note that in Bitcoin there are several different blocks that a player $p$ can use to extend the current state when mining upon a block $b$ (depending e.g on the ordering of transactions, or the nonce being used to announce the block). Since we are interested primarily in miners behaviour, we just focus on the owner of the block following $b$, and do not consider the possibility of two different blocks belonging to $p$ being added on top of $b$. Alternatively, if we consider the Bitcoin protocol, we could say that all the different blocks that $p$ can put on top of $b$ are considered equivalent, since they give $p$ the same reward. \etienne{We may have to specify that it is under the assumption that there are no fees or that they are negligible.} \marcelo{Etienne is right about this, depending on the transactions included in the block the reward could be different. I don't think we should talk about reward here.}

Given a state $q$, we say that the {\em blockchain} of $q$ is the element $b\in q$ of the biggest length, in the case that this element is unique, in which case we denote it by $\bchain(q)$. If two or more different elements of $q$ are tied for the longest, then we say that the blockchain in $q$ does not exists, and we assume that $\bchain(q)$ is not defined (so that $\bchain(\cdot)$ is a partial function).

On each step, miners looking to maximize their rewards choose a block in the current state, and attempt to mine on top of this block. Thus, in each turn, each of the players race to place the next block in the state, and only one of them succeeds. The probability of succeeding is directly related to the comparative amount of hash power available to this player, the more hash power the likely it is that she will mine the next block before the rest of the players. Once a player places a block, this block is added to the current state, obtaining a different state from which the game continues.

Let $p \in \bP$. Given a block $b \in \bB$ and a state $q \in \bQ$, we denote by $\mine(p,b,q)$ an action played in the mining game in which player $p$ mines on top of block $b$. Such an action $\mine(p,b,q)$ is considered to be valid if $b \in q$ and $b\cdot p \not\in q$. The set of valid actions for player $p$ is collected in the set:
\begin{eqnarray*}
\bA_p & = & \{ \mine(p,b,q) \mid b \in \bB, q \in \bQ \text{ and }\mine(p,b,q) \text{ is a valid action}\}.
\end{eqnarray*}
Moreover, given $a \in \bA_p$ with $a = \mine(p,b,q)$, the result of applying $a$ to $q$, denoted by $a(q)$, is defined as the state $q \cup \{b \cdot p\}$. Finally, we denote by $\bA$ the set of combined actions for the $m$ players, that is, $\bA = \bA_0 \times \bA_1 \times \cdots \times \bA_{m-1}$.

Given a player $p \in \bP$ and a state $q \in \bQ$, the pay-off of player $p$ in $q$ is denoted by $r_p(q)$. Moreover, assuming that there is a  function $r_p$ for each player $p \in \bP$, define $\bR = (r_0, r_1, \ldots, r_{m-1})$ as the pay-off function of the game.

As a last component of the game, we assume that $\pr : \bQ \times \bA \times \bQ \to [0,1]$ is a transition probability function satisfying that for every $q \in \bQ$ and $\ba = (a_0, a_1, \ldots, a_{m-1})$ in $\bA$:
\begin{eqnarray*}\label{eq-prop}
\sum_{p=0}^{m-1} \pr(q, \ba, a_p(q)) & = & 1.
\end{eqnarray*}
Notice that if $p_1$ and $p_2$ are two different players, then for every action $a_1 \in \bA_{p_1}$, every action $a_2 \in \bA_{p_2}$ and every state $q \in \bQ$, it holds that $a_1(q) \neq a_2(q)$. Thus, we can think of $\pr(q, \ba, a_p(q))$ as the probability that player $p$ places the next block, which will generate the state $a_p(q)$. 

Summing up, from now on we consider an infinite stochastic game $\Gamma = (\bP,\bA,\bQ,\bR,\pr)$, where:
\begin{itemize}
	\item $\bP$ is the set of players.
	\item $\bA$ is the set of possible actions.
	\item $\bQ$ is the set of states.
	\item $\bR$ is the pay-off function.
	\item $\pr$ is the transition probability function.
\end{itemize} 

\subsection{A Simplification of the Game}
\label{sec-simp}
As mentioned in the previous section, assuming that $\ba = (a_0, a_1, \ldots, a_{m-1})$ in $\bA$ represents the set of actions that players are willing to execute, the probability that action $a_p$ is indeed executed is given by $\pr(q, \ba, a_p(q))$. Such a probability is directly related with the hash power of player $p$, the more hash power the likely it is that action $a_p$ is executed and $p$ mines the next block before the rest of the players. In what follows, we assume that the hash power of each player does not change during the mining game, which is captured by the following condition:
\begin{itemize}
\item For every $q, q' \in \bQ$, every $\ba, \ba' \in \bA$ such that $\ba = (a_0, a_1, \ldots, a_{m-1})$ and $\ba' = (a'_0, a'_1, \ldots, a'_{m-1})$,  and every player $p \in \bP$, it holds that $\pr(q, \ba, a_p(q)) = \pr(q', \ba', a'_p(q'))$.
\end{itemize}
Thus, we assume from now on that this condition is satisfied. In particular, for each player $p \in \bP$, we assume that that 
$\pr(q, \ba, a_p(q)) = h_p$ for every $q \in \bQ$ and $\ba \in \bA$ with $\ba = (a_0, a_1, \ldots, a_{m-1})$, and we refer to $h_p$ as the hash power of player $p$. Moreover, we define $\bH = (h_0, h_1, \ldots, h_{m-1})$ as the hash power distribution, and we replace $\pr$ by $\bH$ in the definition of an an infinite stochastic game, so that $\Gamma = (\bP,\bA,\bQ,\bR,\bH)$.

\subsection{Stationary equilibrium}
A strategy for a player $p \in \bP$ is a function $s : \bQ \rightarrow \bA_p$. 
We define $\bS_p$ as the set of all strategies for player $p$, and $\bS = \bS_0 \times \bS_{1} \times \cdots \times \bS_{m-1}$ as the set of combined strategies for the game (recall that we are assuming that $\bP = \{0, 1, \ldots, m-1\}$ is the set of players). 

Next we define a notion of how likely is reaching a state using a particular strategy, when starting at some specific state. 
Let $\bs = (s_0, s_1, \ldots, s_{m-1})$ be a strategy in $\bS$. Then given $q \in \bQ$, define $\bs(q)$ as the combined action $(s_0(q), s_1(q), \ldots, s_{m-1}(q))$. Moreover, given an initial state $q_0 \in \bQ$, 
the probability of reaching state $q \in \bQ$ such that $q_0 \subseteq q$ is recursively defined as follows:
\begin{eqnarray*}
\pr^{\bs}(q \mid q_0) & = &
\begin{cases}
1 & \text{if } q =  q_0\\
& \\
{\displaystyle \sum_{\substack{q' \in \bQ \,:\\ q_0 \subseteq q' \text{ and } |q'| - |q_0| = k-1}} \pr^{\bs}(q' \mid q_0) \cdot \pr(q', \bs(q'), q)}
 & \text{if } |q| - |q_0| = k \text{ and } k \geq 1
\end{cases}
\end{eqnarray*}
In this definition, if for a player $p$ we have that $s_p(q') = a$ and $a(q') = q$, then $\pr(q', \bs(q'), q) = h_p$. Otherwise, we have that $\pr(q', \bs(q'), q) = 0$. Hence, consistently with the simplification described in Section \ref{sec-simp}, we can replace the transition probability function $\pr$ by the hash power distribution $\bH$ when computing $\pr^{\bs}(q \mid q_0)$. 

We finally have all the necessary ingredients to define the utility of a player in a mining game given a particular strategy.
\begin{mydef}
Let $p \in \bP$, $q_0 \in \bQ$, $\bs \in \bS$, $\beta \in [0,1]$ and $n \geq 0$. Then the cumulative $\beta$--discounted utility of player $p$ for the strategy $\bs$ from the state $q_0$ in a mining game with $n$ steps, denoted by $u_p^n(\bs \mid q_0)$, is defined as:
\begin{eqnarray*}
u_p^n(\bs \mid q_0) & = & \sum_{i=0}^{n}\beta^{i} \cdot  \bigg(\sum_{\substack{q \in \bQ \,: \\ q_0 \subseteq q \text{ {\rm and} } |q| - |q_0| = i}} r_p(q) \cdot 
\pr^{\bs}(q \mid q_0)\bigg)
\end{eqnarray*}
Moreover, the cumulative $\beta$--discounted utility of player $p$ for the strategy $\bs$ from the state $q_0$ in an infinite mining game, denoted by $u_p(\bs \mid q_0)$, is defined as:
\begin{eqnarray*}
u_p(\bs \mid q_0) & = & \sum_{i=0}^{\infty}\beta^{i} \cdot  \bigg(\sum_{\substack{q \in \bQ \,: \\ q_0 \subseteq q \text{ {\rm and} } |q| - |q_0| = i}} r_p(q) \cdot 
\pr^{\bs}(q \mid q_0)\bigg)
\end{eqnarray*}
\end{mydef}
Given a player $p \in \bP$, a combined strategy $\bs \in \bS$, with $\bs = (s_0,s_1, \ldots, s_{m-1})$, and a strategy $s$ for player $p$ ($s \in \bS_p$), we denote by $(\bs_{-p}, s)$ the strategy $(s_0, s_1, \ldots s_{p-1},s,s_{p+1}, \ldots, s_{m-1})$.
\begin{mydef}
Let $q_0 \in \bQ$, $\bs \in \bS$, $\beta \in [0,1]$ and $n \geq 0$. Then $\bs$ is a $\beta$ discounted stationary equilibrium from the state $q_0$ in a mining game with $n$ steps if for every player $p \in \bP$ and every strategy $s$ for player $p$ $(s \in\bS_p)$, it holds that:
\begin{eqnarray*}
u_p^n(\bs \mid q_0)  & \geq  & u_p^n((\bs_{-p},s) \mid q_0).
\end{eqnarray*}
Moreover, $\bs$ is a $\beta$ discounted stationary equilibrium from the state $q_0$ in  the infinite mining game if for every player $p \in \bP$ and every strategy $s$ for player $p$ $(s \in\bS_p)$, it holds that:
\begin{eqnarray*}u_p(\bs \mid q_0)  & \geq  & u_p ((\bs_{-p},s) \mid q_0).
\end{eqnarray*}
\end{mydef}

