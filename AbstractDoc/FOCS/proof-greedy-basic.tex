\documentclass[11pt]{article}
\usepackage{amsmath,amssymb,amsfonts,amsthm}
\usepackage[english]{babel}
\usepackage{xcolor}
\usepackage[letterpaper, margin=1in]{geometry}
\usepackage{tikz}
\usetikzlibrary{automata, graphs,positioning,chains,arrows,decorations.pathmorphing}
\usepackage{url}

\allowdisplaybreaks

% \let\bfseriesbis=\bfseries \def\bfseries{\sffamily\bfseriesbis}
%
%
% \newenvironment{point}[1]%
% {\subsection*{#1}}%
% {}
%
% \setlength{\parskip}{0.3\baselineskip}


%% USEFUL packages
%\usepackage{mypackages}

%% USEFUL macros
%% Macros Juan

\newcommand{\paths}{\text{PATHS}}

%% Macros comments
\newcommand{\tover}[1]{\textcolor{red}{#1}}
\newcommand{\td}[1]{\textcolor{blue}{[TODO: #1]}}

%% Macros logics
\newcommand{\NN}{\mathbb{N}}
\newcommand{\ZZ}{\mathbb{Z}}
\newcommand{\MM}{\mathbb{M}}
\newcommand{\SE}{\mathbb{S}}
\newcommand{\BB}{\mathbb{B}}
\newcommand{\RR}{\mathbb{R}}
\newcommand{\cF}{\mathcal{F}}
\newcommand{\cI}{\mathcal{I}}
\newcommand{\QFBILIA}{\textsf{QFBILIA}}
\newcommand{\nnf}{\textsf{f}}

\newcommand{\ite}{\textsf{ite}}
\newcommand{\limp}{\Rightarrow}
\newcommand{\flc}{\rightarrow}

\newcommand{\bagone}[1]{\llbracket #1 \rrbracket}
\newcommand{\bsingle}{\textsf{bag}}
\newcommand{\bplus}{\oplus}
\newcommand{\bminus}{\ominus}

%% Macros tools
\newcommand{\spen}{\textsc{spen}}
\newcommand{\zzz}{\textsc{Z3}}

%% Environments
\newtheorem*{myrem}{Remark} %% based on amsthm
\newtheorem{mydef}{Definition}
\newtheorem{myex}{Example}
\newtheorem*{mynota}{Notation}
\newtheorem{myprop}{Proposition}
\newtheorem{mylem}{Lemma}
\newtheorem*{mylem*}{Lemma}
\newtheorem*{myprop*}{Proposition}
\newtheorem*{comp}{Efficiency Study}

\newenvironment{point}[1]
{\subsection*{#1}}%
{}

\newcommand{\flist}{\text{\sc FList}}
\newcommand{\set}{\text{\sc Set}}
\newcommand{\fset}{\text{\sc FSet}}
\newcommand{\B}{\text{\bf B}}
\newcommand{\G}{\mathcal{G}}
\newcommand{\K}{\mathcal{K}}
\newcommand{\LOG}{\text{\sc Log}}
\newcommand{\length}{\text{\rm length}}
\newcommand{\BK}{\text{\sc BK}}
%\newcommand{\cP}{\mathcal{P}}
%\newcommand{\cV}{\mathcal{V}}
%\newcommand{\cC}{\mathcal{C}}
%\newcommand{\cS}{\mathcal{S}}
%\newcommand{\cA}{\mathcal{A}}
%\newcommand{\cR}{\mathcal{R}}
%\newcommand{\cQ}{\mathcal{Q}}
\newcommand{\cG}{\mathcal{G}}
\newcommand{\cT}{\mathcal{T}}
\newcommand{\pr}{\mathbf{Pr}}
\newcommand{\Dyck}{\mathcal{D}}
\newcommand{\expected}{\mathbf{E}}
\newcommand{\bv}{\mathbf{v}}
\newcommand{\bV}{\mathbf{V}}
\newcommand{\bs}{\mathbf{s}}
\newcommand{\bw}{\mathbf{w}}
\newcommand{\ba}{\mathbf{a}}
\newcommand{\bq}{\mathbf{q}}
\newcommand{\bx}{\mathbf{x}}
\newcommand{\by}{\mathbf{y}}


\newcommand{\marcelo}[1]{{\color{red} {\bf Marcelo: #1}}}
\newcommand{\etienne}[1]{{\color{blue} {\bf Etienne: #1}}}
\newcommand{\juan}[1]{{\color{brown} {\bf Juan: #1}}}
\newcommand{\domagoj}[1]{{\color{green} {\bf Domagoj: #1}}}
\newcommand{\francisco}[1]{{\color{magenta} {\bf Francisco: #1}}}
\newcommand{\martin}[1]{{\color{orange} {\bf Martin: #1}}}

\newcommand{\quot}[1]{#1/\!\equiv}

\newcommand{\body}{q}
\newcommand{\bchain}{\text{bc}}

\newcommand{\owner}{\text{\rm owner}}
\newcommand{\pred}{\text{\rm pred}}
\newcommand{\mine}{\text{\rm mine}}
\newcommand{\suc}{\text{\rm succ}}


\newcommand{\bP}{\mathbf{P}}
\newcommand{\bB}{\mathbf{B}}
\newcommand{\bA}{\mathbf{A}}
\newcommand{\bR}{\mathbf{R}}
\newcommand{\bS}{\mathbf{S}}
\newcommand{\bH}{\mathbf{H}}
\newcommand{\bQ}{\mathbf{Q}}

\newcommand{\df}{\text{\rm default}}
\newcommand{\gf}{\text{\rm gen\_fork}}
\newcommand{\mfork}{\text{\rm fork($m$)}}
\newcommand{\last}{\text{\rm last}}
\newcommand{\best}{\text{\rm best}}
\newcommand{\cho}{\text{\rm choose}}

\newcommand{\ie}{i.e.$\!$ }

\newcommand{\longest}{{\text{\rm longest}}}

\newcommand{\subbody}{{\text{\rm sub-body}}}








%% Title
% \title{}
% \author{}
% \date{}

\newcommand{\nb}{\text{\rm nb}}
\newcommand{\lnb}{\text{\rm longest\_nb}}
\newcommand{\meetnb}{\text{\rm meet\_nb}}
\newcommand{\bc}{\text{\rm best\_chain}}

\begin{document}

\section{Compact formalisation for greedy basic}

Let $p \in \bP$ and $q \in \bQ$, and recall that:
\begin{eqnarray*}
\longest(q) & = & \{ b \in q \mid \text{for every } b' \in q: |b'| \leq |b|\}\\
\longest(q,p) & = & \{ b \in q \mid (b = \varepsilon \text{ or } \owner(b) = p),\\
&& \hspace{70pt} \text{ and for every } b' \in q \text{ such that } \owner(b') = p : |b'| \leq |b|\}.
\end{eqnarray*}
Then $\nb(b,p)$ is defined as  $\sum_{i=1}^{|b|} \chi_p(b,i)$, that is, as the number of blocks belonging to a player $p$ in the path in $q$ from the genesis block to $b$. Moreover, $\lnb(q,p)$ is defined as the set of blocks $b \in \longest(q,p)$ such that the path in $q$ from the genesis block to $q$ contains the larger numbers of blocks belonging to $p$:
\begin{eqnarray*}
\lnb(q,p) & = & \{ b \mid b \in \longest(q,p) \wedge \forall b' \in \longest(q,p): \nb(b',p) \leq \nb(b,p)\}.
\end{eqnarray*}
Finally, we define $\meetnb(q)$ and $\bc(q,p)$ as follows:
\begin{eqnarray*}
\meetnb(q) & = & \max_{\preceq} \{ b \in q \mid \forall p \in \bP \ \forall b' \in \lnb(q,p) : b \preceq b' \} \\
\bc(q,p) & = & \{ b \mid b \in \longest(q) \wedge \forall b' \in \longest(q): \nb(b',p) \leq \nb(b,p)\}
\end{eqnarray*}
Notice that $\meetnb(q)$ is always a singleton set, so we assume that $\meetnb(q)$ is a block. On the other hand, $\lnb(q,p)$ and $\bc(q,p)$ may contain more than one block. In the following 
proposition, we show a condition that ensures that for every player $p$ and reachable state $q$, it holds that $\lnb(q,p)$ and $\bc(q,p)$ are singleton sets, so that each one of them can be thought of as a block. More precisely, we say that a combined strategy $\bs$ is almost-greedy if  all but one strategy in $\bs$ are greedy. Then we have that:
%
%As mentioned before $\mathbf{L}(q,p)$ is a set in the general case, however if we restrict ourself to greedy basic strategies for every player, then $|\mathbf{L}(q,p)| = 1$ for each reachable state $q$.
%We define $longest(q,p)$ as a block such that: $longest(q,p) \in \mathbf{L}(q,p)$ and for any $b \in \mathbf{L}(q,p), blocks(b,p) \leq blocks(longest(q,p),p)$. And we define $all-meet(q)$ as the longest block such that $all-meet(q) \in q$ and for any $p$, $all-meet(q) \preceq longest(q,p)$.
%Recall that $longest(q) = \{b\in q \mid \forall b' \in q, |b'| \leq |b| \}$ then we define
%$bestchain_p(q)$ as a block such that: $bestchain_p(q) \in  longest(q)$ and for any $ b \in longest(q) blocks(b,p) \leq blocks(bestchain_p(q),p)$
%
\begin{myprop}
If $\bs$ is a combined almost-greedy strategy, then for every $p \in \bP$ and $q \in \bQ$ such that 
$\Pr^{\bs}(q \mid \varepsilon) > 0$, it holds that $\lnb(q,p)$ and $\bc(q,p)$ are singleton sets.
\end{myprop}
\marcelo{I will include the proof of this proposition.}

\iffalse

\begin{myprop}
Let $\bs$ a combined greedy basic strategy, then for any $p$ and any $q$  such that $\Pr^{\bs}(q \mid \epsilon) \neq 0$ we have $|Longest_0(q,p)| = 1$ and $|Longest_p(q)| = 1$ and ... WIP
\end{myprop}
\begin{proof}
to do.
\end{proof}

\begin{myprop}
Let $\bs$ a combined greedy basic strategy and $\sigma$ a strategy for player $p$, then for any $q$ such that $\Pr^{(\bs_{\lnot p},\sigma)}(q \mid \epsilon) \neq 0$ we have $|Longest_1(q,p)| = 1$ and ... WIP
\end{myprop}
\begin{proof}
to do.
\end{proof}

\fi

From now, we assume that every combined strategy is almost-greedy. Given a combined almost-greedy strategy $\bs = (s_0, \ldots, s_{m-1})$ and a player $p \in \bP$, we say that a function $f : \mathbb{N} \to \mathbb{N}$ specifies $s_p$ given $\bs$ if for every state $q \in \bQ$:
%Let $s$ be a greedy basic strategy for player $p \in \mathbf{P}$, we say that $f : \mathbf{N} \rightarrow \mathbf{N}$ specify $s$ denoted $s_f$ if and only if for any $q \in \mathbf{Q}$ we have :
\begin{eqnarray*}
s_p(q) & = &
\begin{cases}
\mine(p,\lnb(q,p),q) & \text{if } f(|\lnb(q,p)|-|\meetnb(q)|) >\\
& \hspace{100pt}|\bc(q,p)|-|\meetnb(q)|\\
\mine(p,b,q) & \text{otherwise, where } \bc(q,p) = b \cdot v \text{ and } |v| = f(0)\\
\end{cases}
\end{eqnarray*}

\etienne{Whenever one stop forking the function specify where to mine again which is a stronger constraint than general greedy basic strategy.}

One may be tempted to think that every function $f: \mathbb{N} \to \mathbb{N}$ can be used to define some combined almost-greedy strategy $\bs$ such that the strategy $s_p$ is specified by $f$ given $\bs$.
The following lemma shows that this is not the case.
\begin{mylem}
Let $\bs = (s_0, \ldots, s_{m-1})$ be a combined almost-greedy strategy and $p \in \bP$ such that $s_p$ is a greedy strategy.  If a function $f : \mathbb{N} \to \mathbb{N}$ specifies the strategy of player $p$ given $\bs$, then for every $n \in \mathbb{N}$, we have that  $f(n) \geq n + f(0)$.
\end{mylem}
\begin{proof}
By contradiction using the definition of greedy strategy. 
\end{proof}
In this framework, the Default strategy is specified by the function $f$ such that for every $n \in \mathbb{N}$, it holds that $f(n) = n$.  Always fork cannot really be specified so we have to add it, but informally it is $f_\varepsilon$ such that for every $n \in \mathbb{N}$, it holds that $f_\varepsilon(n) = \infty$.



\section{}

In this section, we assume that $m = 2$, so that combined strategies are of the form $(s_0,s_1)$. In order to simplify notation, instead of writing $u_p((s_0,s_1) \mid \varepsilon)$, we omit the inner parenthesis and simply write $u_p(s_0,s_1 \mid \varepsilon)$.
\begin{myprop}
\label{mainprop}
Let $s_1$ be a greedy basic strategy for player 1. Then there exists a greedy basic strategy $s_0$ for player 0 such that, for every strategy $\sigma$ for player 0, it holds that $u_0(s_0,s_1 \mid \varepsilon) \geq u_0(\sigma,s_1 \mid \varepsilon)$.
\end{myprop}
In order to prove the Proposition \ref{mainprop}, we need some technical lemmas. Let $\mathcal{F}_0$ be the set of all functions $f : \mathbb{N} \to \mathbb{N}$ for which there exists a combined almost-greedy strategy.
The first such a lemma states that there exists a reachable maximum for the utility of player $0$ when the strategy of player 1 is fixed.
\begin{mylem}
\label{limlem}
Let $s_1$ be a greedy basic strategy for player 1. Then we have that the following value is well-defined:
$$\max \{u_0(s_0,s_1 \mid \varepsilon) \mid s_0 \text{ is a greedy strategy for player } 0 \text{ specified by a function } f : \mathbb{N} \to \mathbb{N}\}.$$
\end{mylem}
\begin{proof}
\end{proof}

The second states that there exists a sequence from a strategy to an other, for which two subsequent elements only differ from one position.

\begin{mylem}
\label{seqlem}
Let $s$ and $\sigma$ two strategies, then
there exits $(\sigma_i)$ a sequence of strategy such that : $\sigma_0 = s$ and $\underset{n\rightarrow \infty}{lim}\sigma_N = \sigma$. Moreover for any $i < N$, exists a unique $q_{i+1} \in \mathbf{Q}$ such that $\sigma_{i}(q_{i+1}) \neq \sigma_{i+1}(q_{i+1})$ or for all $i' \geq i, \sigma_{i} = \sigma_{i'}$. Moreover for any $i,i' < N$, such that $i < i'$ then $q_{i'} \not \subseteq q_{i}$.
\end{mylem}
\begin{proof}
\end{proof}

Therefore with lemma \ref{limlem} we can conclude that there exists a value $v_{max} = max \{u_0(s_{f_0},s_1 \mid \epsilon) \mid f_0 \in \mathbf{F}_0 \}$. Hence there exits $f_{max}$ such that $u_0(s_{f_{max}},s_1\mid \epsilon) = v_{max}$ and $f_{max} \in \mathbf{F}_0$. We define $s$ as the greedy basic strategy induced from $f_{max}$.

From lemma \ref{seqlem} we know that there exits $(\sigma_i)$ a sequence of strategy such that : $\sigma_0 = s$ and $\underset{n\rightarrow \infty}{lim}\sigma_N = \sigma$. Moreover for any $i < N$, exists a unique $q_{i+1} \in \mathbf{Q}$ such that $\sigma_{i}(q_{i+1}) \neq \sigma_{i+1}(q_{i+1})$ or for all $i' \geq i, \sigma_{i} = \sigma_{i'}$. Moreover for any $i,i' < N$, such that $i < i'$ then $q_{i'} \not \subseteq q_{i}$.


We know prove inductively on the sequence $(\sigma_i)$ that for any $i \geq 0$, $u_0(\sigma_i,s_1 \mid \epsilon) \leq u_0(s,s_1 \mid \epsilon)$. For $i = 0$ it is trivial by definition of $\sigma_0 = s_0$.
Assume for any $i < N$ we have $u_0(\sigma_i,s_1 \mid \epsilon) \leq u_0(s,s_1 \mid \epsilon)$, we know have to distinguish 2 cases.

\bigskip
The first one correspond to "noisy" actions, an action which does not modify the longest chain of the player, hence which will never be rewarding in the future. This can also be seen as non-greedy actions.

\textbf{Case 1: $longest(\sigma_{i+1}(q_{i+1}),0) = longest(q_{i+1},0)$.}

\begin{mylem}
\label{lemdif}
$u_0(\sigma_i,s_1 \mid \epsilon) - u_0(\sigma_{i+1},s_1 \mid \epsilon) = u_0(s,s_1 \mid s(q_{i+1})) - u_0(s,s_1 \mid \sigma_{i+1}(q_{i+1}))$
\end{mylem}
\begin{proof}
\iffalse
For any $i < N$ :

\begin{eqnarray*}
u_p(\bs \mid q) & = & \sum_{q' \in \bQ \,:\, q \subseteq q'} \beta^{|q'|-|q|} \cdot  r_p(q') \cdot \pr^{\bs}(q' \mid q)
\end{eqnarray*}

\begin{eqnarray*}
u_0(\sigma_{i+1},s_1 \mid \epsilon) & = & \sum_{q \in \bQ} \beta^{|q|} \cdot  r_0(q) \cdot \pr^{(\sigma_{i+1},s_1)}(q \mid \epsilon)\\
u_0(\sigma_{i+1},s_1 \mid \epsilon) & = & \sum_{q \in \bQ \,:\, |q| < |q_{i+1}| } \beta^{|q|} \cdot  r_0(q) \cdot \pr^{(\sigma_{i+1},s_1)}(q \mid \epsilon)\\
& & + \sum_{q \in \bQ \,:\, |q| \geq |q_{i+1}| } \beta^{|q|} \cdot  r_0(q) \cdot \pr^{(\sigma_{i+1},s_1)}(q \mid \epsilon)\\
u_0(\sigma_{i+1},s_1 \mid \epsilon) & = & \sum_{q \in \bQ \,:\, |q| < |q_{i+1}| } \beta^{|q|} \cdot  r_0(q) \cdot \pr^{(\sigma_{i+1},s_1)}(q \mid \epsilon)\\
& & + \sum_{q \in \bQ \,:\, |q| = |q_{i+1}| } u_0(\sigma_{i+1},s_1 \mid q) \cdot \pr^{(\sigma_{i+1},s_1)}(q \mid \epsilon)\\
\end{eqnarray*}

\begin{eqnarray*}
u_0(\sigma_{i+1},s_1 \mid \epsilon) & = & \sum_{q \in \bQ} \beta^{|q|} \cdot  r_0(q) \cdot \pr^{(\sigma_{i+1},s_1)}(q \mid \epsilon)\\
& = & \sum_{q \in \bQ \,:\, q_{i+1} \not \subseteq q} \beta^{|q|} \cdot  r_0(q) \cdot \pr^{(\sigma_{i},s_1)}(q \mid \epsilon) \\
&  & +  \sum_{q \in \bQ \,:\, q_{i+1} \subseteq q} \beta^{|q|} \cdot  r_0(q) \cdot \pr^{(\sigma_{i+1},s_1)}(q \mid \epsilon) \\
\end{eqnarray*}

\fi
\end{proof}

\begin{mylem}
\label{lembeta}
If $longest(\sigma_{i+1}(q_{i+1}),0) = longest(q_{i+1},0)$ then  $u_0(s,s_1 \mid \sigma_{i+1}(q_{i+1})) = \beta \cdot u_0(s,s_1 \mid s(q_{i+1}))$
\end{mylem}
\begin{proof}
\end{proof}

By lemma \ref{lemdif} we have that $u_0(\sigma_i,s_1 \mid \epsilon) < u_0(\sigma_{i+1},s_1 \mid \epsilon)$ if and only if $u_0(s,s_1 \mid s(q_{i+1})) < u_0(s,s_1 \mid \sigma_{i+1}(q_{i+1}))$.
Moreover by lemma \ref{lembeta} we have $u_0(s,s_1 \mid \sigma_{i+1}(q_{i+1})) = \beta \cdot u_0(s,s_1 \mid s(q_{i+1}))$.
Hence $u_0(\sigma_{i+1},s_1 \mid \epsilon) \leq u_0(s_0,s_1 \mid \epsilon)$.

\bigskip
The second case corresponds to a different greedy actions, if this new action is strictly better, we can build a strictly better greedy \textbf{basic} strategy for the player which contradict the definition of $s_0$.

\textbf{Case 2: $longest(\sigma_{i+1}(q_{i+1}),0) \neq longest(q_{i+1},0)$.}

Assume that for all $j < i+1$ we have $longest(\sigma_{i+1}(q_{j}),0) = longest(q_{j},0)$ then we define $\mathbf{Q}_{f_{i+1}} = \{q \in \bQ \mid \sigma_{i+1}(q) \preceq longest(\sigma_{i+1}(q_{j}),0) \}$ and  $f_{i+1} : \mathbb{N} \rightarrow \mathbb{N}$ such that for any $a \in \mathbb{N}$:
\begin{eqnarray*}
f_{i+1}(a) =
\begin{cases}
f_{max}(a) & \text{if for any } q \in \mathbf{Q}_{i+1}, |longest(q,0)| - |all-meet(q)| \neq a\\
v & \text{otherwise with } v = max \{|longest(q)| - |all-meet(q)| \mid q \in \mathbf{Q}_{i+1}, \land |longest(q,0)| - |all-meet(q)| = a \}\\
\end{cases}
\end{eqnarray*}

Then using the same argument as the first case (which is the only one that can happen) we can conclude that  $u_0(s_{f_{i+1}},s_1 \mid \epsilon) \geq u_0(\sigma_{i+1},s_1 \mid \epsilon)$. Moreover by definition of $s_0$ we have that
$u_0(s_0,s_1 \mid \epsilon) \geq u_0(s_{f_{i+1}},s_1 \mid \epsilon)$ hence $u_0(s_0,s_1 \mid \epsilon) \geq u_0(\sigma_{i+1},s_1 \mid \epsilon)$.

\bigskip
Assume there exists $j < i+1$ such that $longest(\sigma_{i+1}(q_{j}),0) \neq longest(q_{j},0)$.

\begin{mylem}
If $\Pr^{(s_{f_j},s_1)}(q_{i+1} \mid \epsilon) = \Pr^{(s_{0},s_1)}(q_{i+1} \mid \epsilon)$
\end{mylem}




\end{document}
