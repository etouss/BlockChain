%!TEX root = focs.tex

\section{The Equilibrium in a Two Players Game}
\label{sec-twoplayers}

\section{Games with Constant Payoff}

The first version of the game we analyse is when the payoff function $r_p(q)$ pays each block in the blockchain the same amount $c$. In order to define this function we need 
some notation. 

Recall that a block $b$ is a string over the alphabet $\bP$, and we use notation $|b|$ for the length of $b$ as a string. Moreover, given blocks $b_1, b_2$, we use notation $b_1 \preceq b_2$ to indicate that $b_1$ is a prefix of $b_2$  when considered as strings. Then we define: 
\begin{eqnarray*}
\longest(q) & = & \{ b \in q \mid \text{for every } b' \in q: |b'| \leq |b|\}\\
\meet(q) & = & {\displaystyle \max_{\preceq} \ \{b \in q \mid \text{for every } b' \in \longest(q): b \preceq b'\}}
\end{eqnarray*}
Intuitively, $\longest(q)$ contains the leaves of all branches in the state $q$ that are currently competing for the blockchain, and $\meet(q)$ corresponds to the last block for which all these branches agree on. Notice that $\meet(q)$ is well defined as $\preceq$ is a linear order on the finite and non-empty set $\{b \in q \mid \text{for every } b' \in \longest(q): b \preceq b'\}$.

We need a formal definition of $\meet(q)$ as the pay-off functions considered in this document keep rewarding each player for her blocks in the path from the genesis block to $\meet(q)$. Thus, in the definition of these functions we need to identify who is the owner of each one of these blocks, which is done by considering a function $\chi_p$, for each $p \in \bP$. More precisely, given $b \in \bB$ and $i \in \{1, \ldots, |b|\}$, we have that:
%First, for a player $p$, define the indicator function $\chi_p : \bB \to \{0,1\}$ as follows:
\begin{eqnarray*}
\chi_p(b,i) & = & 
\begin{cases}
1 & \text{the } i\text{-th symbol in } b \text{ is } p\\
0 & \text{otherwise}
\end{cases}
\end{eqnarray*}

We can finally define the payoff function we consider in this section, which we call the \textbf{constant reward}. For a player $p$, we define it as 
\begin{eqnarray*}
r_p(q) & = & 
{\displaystyle c \cdot \sum_{i=1}^{|\meet(q)|} \chi_p(\meet(q),i)} 
\end{eqnarray*}
where $c$ is a positive real number. 
%Here we use notation $b[i]$ for the $i$-th symbol in $b$, where $i \in \{1, \ldots, |b|\}$. 
% and $d \in \mathbb{N}$. Here the number $d$ is the amount of confirmations needed to spend the block (6 in the case of Bitcoin).
 
\subsection{When no player forks}

Let us start with analysing the most obvious strategies for all players: regardless of what everyone else does, keep mining on the blockchain. We call this 
the \emph{default} strategy, as is it reflects the desired behaviour of the miners participating in the Bitcoin network. For a player $p$, let us denote this strategy 
by $\df_p$, and consider the combined strategy $\df = (\df_1,\dots,\df_p)$. 

We can easily calculate the utility of player $p$ under $\df$: we can think that player $p$ receives a fraction $h_p$ of the next block in the blockchain, and thus his 
total utility amounts to $$u_p(\bs \mid q_0) = c\cdot h \cdot \sum_{i=0}^{\infty}i \cdot \beta^{i}$$ (recall that $q_0$ is the genesis block). 

\juan{we need a lemma here with the analytic form}
Can any player do better? Not really, or at least not if we assume that the rest of the players will behave according to $\df$. In the remainder of the section we 
will prove the fact that $\df$ is a $\beta$-discounted stationary equilibrium, under a very mild assumption on the set of strategies available to each player. 
To prove this fact we need to show that the utility of player $p$ under any strategy $(\df_{-p},s)$ is not higher than the utility under $\df$. Note that we only need to focus on 
games for two players, as all the players distinct from $p$ can be combined into a single player with their combined hash power. 

The key lemma we use to prove this result (and several others) is the fact that, in our game, the optimal strategies for a player $p$ are determined only by the portion 
of blocks that appear after the last block owned by $p$ in the blockchain. This is what we show in the following section. 

\subsection{A normal form for states under two-player games} 

In this section we will only consider games with two players, that is, $\bP = \{0,1\}$. Our main objective is to show the Lemma we have informally stated above; to do so 
we need some terminology. Given a player $p \in \{0,1\}$ and a state $q \in \bQ$, define:
\begin{multline*}
\longest(q,p) \ = \ \{ b \in q \mid b = \varepsilon \text{ or } \owner(b) = p,\\
\text{ and for every } b' \in q \text{ such that } \owner(b') = p : |b'| \leq |b|\}
\end{multline*}
Notice that $\varepsilon \in \longest(p,q)$ if and only if there is no $b \in q$ such that $\owner(b) = p$. Moreover, define $\length(q,p)$ as the length of an arbitrary string in $\longest(q,p)$ (all of them have the same length).
\begin{mydef}\label{def-greedy}
Given $p \in \{0,1\}$, $b \in \bB$ and $q \in \bQ$,  an action $\mine(p,b,q)$ is {\em greedy} if $\mine(p,b,q)$ is a valid action and $\length(q,p) \leq |b|$.

Moreover, a strategy $\bs = (s_0, s_1)$ is {\em greedy} if for every $p \in \{0,1\}$ and  $q \in \bQ$ such that $\pr^{\bs}(q \mid \varepsilon) > 0$, it holds that $s_p(q)$ is a greedy action.
\end{mydef}
From now on, we only consider greedy strategies. 

Under greedy strategies, players refrain to fork on top of blocks that appear before their latest block, or their latest block in the blockchain. The consequence of using 
greedy strategies is that every state in a two-player game under greedy strategies cannot have more than two paths contesting for the blockchain. This is captured b the following 
technical lemma: 
\begin{mylem}\label{lem-length-greedy}
Let $\bs$ be a greedy strategy. Then for every $q \in \bQ$ such that $\pr^{\bs}(q \mid \varepsilon) > 0$, the following conditions hold:
\begin{enumerate}
\item For every $p \in \{0,1\}$: $|\longest(q,p)| = 1$ 

\item $1 \leq |\longest(q)| \leq 2$

\item If $|\longest(q)| = 2$, then $\longest(q) = \longest(q,0) \cup \longest(q,1)$
\end{enumerate}
\end{mylem}

\begin{proof}
Let $S = \{ q \in \bQ \mid \pr^{\bs}(q \mid \varepsilon) > 0 \}$. Then we have that $S$ is the smallest subset of $\bQ$ satisfying the following conditions:
\begin{itemize}
\item $\{\varepsilon\} \in S$.

\item If $p \in \{0,1\}$, $b \in \bB$, $q \in S$ and $\mine(p,b,q)$ is a greedy action, then $q \cup \{ b \cdot p\} \in S$.
\end{itemize}
Hence, we have an inductive definition of $S$, and we can prove the proposition by induction on the structure of this set of states. If $q = \{ \varepsilon \}$, then we have that $\longest(q) = \longest(q,1) = \longest(q,2) = \{ \varepsilon \}$ and, thus, we have that  the three conditions in the proposition hold since $|\longest(q)| = |\longest(q,0)| = |\longest(q,1)| = 1$. Assume that the property holds for $q \in S$, and assume that $p \in \{0,1\}$, $b \in \bB$ and $\mine(p,b,q)$ is a greedy action. Then we need to prove that the conditions in the proposition hold for $q ' = q \cup \{b \cdot p \}$, for which we consider the following cases.
\begin{itemize}
\item Assume that $\longest(q,0) = \{b_0\}$,  $\longest(q, 1) = \{b_1\}$ and $\longest(q) = \{b_0,b_1\}$, and without loss of generality assume that $p = 0$. Given that $|b_0| = |b_1|$ and $\mine(p,b,q)$ is a greedy action, we have that either $b = b_0$ or $b = b_1$. If $b = b_0$, then it holds $b_0 \cdot 0 \in q'$, from which we conclude that the three conditions of the proposition hold since $\longest(q',0) = \{b_0 \cdot 0\}$, $\longest(q',1) = \{b_1\}$ and $\longest(q') = \{b_0 \cdot 0\}$.  If $b = b_1$, then it holds $b_1 \cdot 0 \in q'$, from which we conclude that the three conditions of the proposition hold since $\longest(q',0) = \{b_1 \cdot 0\}$, $\longest(q',1) = \{b_1\}$ and $\longest(q') = \{b_1 \cdot 0\}$.

\item Assume that $\longest(q,0) = \{b_0\}$,  $\longest(q, 1) = \{b_1\}$ and $\longest(q) = \{b_0\}$, so that $|b_1| < |b_0|$. Notice that if $p =0$, then we have that $b=b_0$ since $\mine(p,b,q)$ is a greedy action. Hence, it holds $b_0 \cdot 0 \in q'$, from which we conclude that the three conditions of the proposition hold since $\longest(q',0) = \{b_0 \cdot 0\}$, $\longest(q',1) = \{b_1\}$ and $\longest(q') = \{b_0 \cdot 0\}$. Therefore, assume that $p = 1$, from which we have that  $|b_1| \leq |b|$ and $|b| \leq |b_0|$, since $\mine(p,b,q)$ is a greedy action and $\longest(q) = \{b_0\}$. Thus, given that $b \cdot 1 \in q'$, we conclude that $\longest(q',0) = \{b_0\}$ and $\longest(q',1) = \{b \cdot 1\}$. 
Moreover, if $|b \cdot 1| < |b_0|$, then it holds that $\longest(q') = \{b_0\}$ and the three conditions of the proposition are satisfied. If $|b \cdot 1| = |b_0|$, then we have that $\longest(q') = \{b_0, b \cdot 1\}$, from which we conclude that the three conditions of the proposition are satisfied since $|\longest(q')| = 2$ and $\longest(q') = \longest(q',0) \cup \longest(q',1)$. Finally, if $|b \cdot 1| > |b_0|$ (that is, if $b = b_0$), then $\longest(q') = \{b \cdot 1\}$ and again the three conditions of the proposition are satisfied.

\item Assume that $\longest(q,0) = \{b_0\}$,  $\longest(q, 1) = \{b_1\}$ and $\longest(q) = \{b_1\}$. This case is analogous to the previous case, which concludes the proof of the proposition.
\end{itemize}
\end{proof}

\paragraph{Longest blocks and optimal strategies.}
For a state $q$ and a block $b \in q$, let us denote by $\subbody(q,b)$ the state
given by $\{u \mid b\cdot u$ is a block in $q\}$, that is, the subtree of $q$ rooted at $b$, but in which $b$ is renamed 
$\epsilon$ and all its descendants are renamed accordingly. 

%Furthermore, let us denote by $\meet(q,p)$ the greatest block in the set $\{b \in q \mid b$ is a prefix all nodes in $\longest(q)\}$, the greatest common block owned by $p$ that is a prefix of all 
%blocks in $\longest(q)$. 

The following Lemma tells us that an optimal strategy for player $p$ can only differentiate the portion of 
a state that goes after $\meet(q,p)$: 

\begin{mylem}
Let $s = (s_1,s_2)$ be a $\beta$ discounted stationary equilibrium in an infinite mining game with two players. 
Then there is a $\beta$ discounted stationary equilibrium such that $u_p(s \mid q_0) = u_p(s' \mid q_0)$ for 
any player $p$ and for every pair $q$ and $q'$ of 
bodies of knowledge in which $\subbody(q,\meet(q,p)) = \subbody(q',\meet(q',p))$ we have that 
$s_p(q) = s_p(q')$. 
\end{mylem}

\begin{proof}
\end{proof}

\subsection{Equilibrium}

We can now formally state and prove our first main result

\begin{mythm}
For any $0 \leq \beta \leq 1$, the strategy $\df$ is a $\beta$-discounted stationary equilibrium under greedy strategies. 
\end{mythm} 

\begin{proof}
recall two players, greedy, use lemma above. 
\end{proof}
